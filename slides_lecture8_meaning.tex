%%%%%%%%%%%%%%%%%%%%%%%%%%%%%%%%%%%%%%%%%
% Beamer Presentation
% LaTeX Template
% Version 1.0 (10/11/12)
%
% This template has been downloaded from:
% http://www.LaTeXTemplates.com
%
% License:
% CC BY-NC-SA 3.0 (http://creativecommons.org/licenses/by-nc-sa/3.0/)
%
%%%%%%%%%%%%%%%%%%%%%%%%%%%%%%%%%%%%%%%%%

%----------------------------------------------------------------------------------------
%	PACKAGES AND THEMES
%----------------------------------------------------------------------------------------

\documentclass[aspectratio=169,usenames,dvipsnames]{beamer}

\usepackage[utf8]{inputenc}
\usepackage{booktabs}
\usepackage{tabularx}
\usepackage[authordate,bibencoding=auto,strict,backend=biber,natbib]{biblatex-chicago}
\addbibresource{bib.bib}
\usepackage{graphicx}
% \hypersetup{
%     colorlinks,
%     %citecolor=black,
%     linkcolor=black
% }
\usepackage{array}
\usepackage{caption}
\usepackage{threeparttable}
\usepackage{epigraph} 
\usepackage{lscape}
\usepackage{adjustbox}
\newcommand*{\Scale}[2][4]{\scalebox{#1}{\ensuremath{#2}}}%
\usepackage{import}
\newenvironment{wideitemize}{\itemize\addtolength{\itemsep}{10pt}}{\enditemize}
\usepackage{amsmath}
\usepackage{csvsimple}
\usepackage{siunitx}
\usepackage{filecontents}
\usepackage{rotating}
\usepackage{multirow}
\usepackage{amsmath}
\usepackage{subcaption}
\usepackage{appendixnumberbeamer}
\usepackage{float}
\usepackage{amsmath}
\usepackage{csvsimple}
\usepackage{hyperref}
\newtheorem{proposition}{Proposition}
\usepackage{xcolor}
\def\boxit#1#2{%
    \smash{\color{red}\fboxrule=1pt\relax\fboxsep=2pt\relax%
    \llap{\rlap{\fbox{\phantom{\rule{#1}{#2}}}}~}}\ignorespaces
}
\newenvironment{variableblock}[3]{%
  \setbeamercolor{block body}{#2}
  \setbeamercolor{block title}{#3}
  \begin{block}{#1}}{\end{block}}
\usepackage{appendixnumberbeamer}
\usepackage{tikz,pgfplots}
\usepackage{tkz-fct}
\usepackage{amsthm}
\pgfplotsset{compat=1.10}
\usepgfplotslibrary{fillbetween}
\mode<presentation> {
\AtBeginSection[]
{
    \begin{frame}
        \frametitle{Table of Contents}
        \tableofcontents[currentsection]
    \end{frame}
}
% The Beamer class comes with a number of default slide themes
% which change the colors and layouts of slides. Below this is a list
% of all the themes, uncomment each in turn to see what they look like.

\usetheme{default}
%\usetheme{AnnArbor}
%\usetheme{Antibes} -
%\usetheme{Bergen}
%\usetheme{Berkeley}
%\usetheme{Berlin}
%\usetheme{Boadilla}
%\usetheme{CambridgeUS}
%\usetheme{Copenhagen} -
%\usetheme{Darmstadt}
%\usetheme{Dresden}
%\usetheme{Frankfurt}
%\usetheme{Goettingen}
%\usetheme{Hannover}
%\usetheme{Ilmenau}
%\usetheme{JuanLesPins}
%\usetheme{Luebeck}
%\usetheme{Madrid}
%\usetheme{Malmoe}
%\usetheme{Marburg}
%\usetheme{Montpellier}
%\usetheme{PaloAlto}
%\usetheme{Pittsburgh}
%\usetheme{Rochester} -
%\usetheme{Singapore}
%\usetheme{Szeged}
%\usetheme{Warsaw}

% As well as themes, the Beamer class has a number of color themes
% for any slide theme. Uncomment each of these in turn to see how it
% changes the colors of your current slide theme.

%\usecolortheme{albatross}
%\usecolortheme{beaver}
%\usecolortheme{beetle}
%\usecolortheme{crane}
%\usecolortheme{dolphin}
%\usecolortheme{dove}
%\usecolortheme{fly}
%\usecolortheme{lily}
%\usecolortheme{orchid}
%\usecolortheme{rose}
%\usecolortheme{seagull}
%\usecolortheme{seahorse}
%\usecolortheme{whale}
%\usecolortheme{wolverine}

%\setbeamertemplate{footline} % To remove the footer line in all slides uncomment this line
%\setbeamertemplate{footline}[frame number] % To replace the footer line in all slides with a simple slide count uncomment this line
\setbeamertemplate{theorems}[numbered]
\setbeamertemplate{navigation symbols}{} % To remove the navigation symbols from the bottom of all slides uncomment this line
}
\setbeamertemplate{caption}{\raggedright\insertcaption\par}
  \setbeamertemplate{enumerate items}[default]
  %\setbeamertemplate{page number in head/foot}{\insertframenumber}
\usepackage{graphicx} % Allows including images
\usepackage{booktabs} % Allows the use of \toprule, \midrule and \bottomrule in tables
%\usepackage {tikz}
\newtheorem*{theorem*}{Theorem}
\newtheorem*{lemma*}{Lemma}
\newtheorem*{proposition*}{Proposition}
\newtheorem*{corollary*}{Corollary}
\newtheorem*{definition*}{Definition}
\DeclareMathOperator*{\argmin}{arg\,min}
\newtheorem*{assumption}{Assumption}
\usetikzlibrary {positioning}
\renewcommand{\arraystretch}{1.5}
\newcommand\hideit[1]{%
  \only<0| handout:1>{\mbox{}}%
  \invisible<0| handout:1>{#1}}
\usepackage[default]{lato}

\setbeamercolor{block body alerted}{bg=alerted text.fg!10}
\setbeamercolor{block title alerted}{bg=alerted text.fg!20}
\setbeamercolor{block body}{bg=structure!10}
\setbeamercolor{block title}{bg=structure!20}
\setbeamercolor{block body example}{bg=green!10}
\setbeamercolor{block title example}{bg=green!20}


\makeatletter
\let\save@measuring@true\measuring@true
\def\measuring@true{%
  \save@measuring@true
  \def\beamer@sortzero##1{\beamer@ifnextcharospec{\beamer@sortzeroread{##1}}{}}%
  \def\beamer@sortzeroread##1<##2>{}%
  \def\beamer@finalnospec{}%
}
\makeatother
%\usepackage {xcolor}

%----------------------------------------------------------------------------------------
%	TITLE PAGE
%----------------------------------------------------------------------------------------

\title[diss]{Lecture 8: Meaning and Performance} % The short title appears at the bottom of every slide, the full title is only on the title page
\author{Compensation in Organizations} % Your name
\institute[shortinst]{Jacob Kohlhepp}
\date{\today} % Date, can be changed to a custom date

\begin{document}

\begin{frame}
\titlepage % Print the title page as the first slide

\end{frame}

\begin{frame}{Discussion}
\centering
    \huge  Ashraf, Bandiera, Minni, and Zingales (2025)

    \normalsize (Not Yet Published)
\end{frame}

\begin{frame}{Meaning and Karl Marx?}

    \begin{quote}
                “What, then, constitutes the alienation of labor? First, the fact that labor is external to the worker; that in his work, he does not feel content but unhappy, does not develop freely his physical and mental energy...The worker therefore only feels himself outside his work, and in his work feels outside himself. He feels at home when he is not working, and when he is working he does not feel at home.”\\
\hfill — Marx, Karl, 1844. Estranged labor.
    \end{quote}

\end{frame}

\begin{frame}{Recall Effort Based Pay Model (Lecture 3)}
    \begin{wideitemize}
        \item The only benefit workers derive from work is the wage.
        \item The costs are several, but most notably the cost of effort.
        \item We often specify this as:
        \[c(e) = e^2/2\]
        \pause 
        \item Three questions:
        \begin{wideitemize}
            \item Should this vary across people?
            \pause 
            \item Should it vary across jobs for the same person?
            \pause
            \item Can it be changed?
        \end{wideitemize}
    \end{wideitemize}
\end{frame}

\begin{frame}{Adding Meaning to Our Model}
\begin{wideitemize}
    \item Change the cost of effort:
    \[c(e) = \frac{e^2}{2(1+\lambda_i m_{ij})} \]

    \item The cost of effort now varies across people and jobs!
    \item Meaning is measured by $m_{ij}$ because:
    \[\lim_{m_{ij}\to \infty} c(e)= \lim_{m_{ij}\to \infty}\frac{e^2}{2(1+\lambda_i m_{ij})}=0\]
    \item ``If you love your job, you will never work a day in your life."
    
\end{wideitemize}

    \hfill \footnotesize Source: Ashraf, Bandiera, Minni, and Zingales (2025)
\end{frame}

\begin{frame}{Adding Meaning to Our Model}
\begin{wideitemize}
    \item A worker ``sees meaning" if $\lambda_i$ is large.
    \item If a worker does not see meaning, $\lambda_i=0$, and we return to our original model!
    \item We add just what we need to capture ``meaning."
    \begin{wideitemize}
        \item We drop risk aversion: assume the worker is risk-neutral.
            \item Does dropping risk aversion matter?
    \end{wideitemize}

\end{wideitemize}

    \hfill \footnotesize Source: Ashraf, Bandiera, Minni, and Zingales (2025)
\end{frame}


\begin{frame}{Some Other Updates}
\begin{wideitemize}
    \item The worker's job at the firm is labeled $j=p$.
    
    
    \item The worker's outside option is a job labeled $j=a$.
    \item We will solve this model once for general $j$ treating base pay as exogenous.
    \item Finally, let output be: $y=\theta_i e +\epsilon$.
    \item The contract is based on output: $w_{ij}(y_{ij})= \alpha_{ij} + \beta_{ij} y_{ij}=\alpha_{ij} + \beta_{ij} \theta_i e_{ij} $

\end{wideitemize}

    \hfill \footnotesize Source: Ashraf, Bandiera, Minni, and Zingales (2025)
\end{frame}


\begin{frame}{Solving the Model: Effort and Bonus}

\begin{proposition}
    The worker's equilibrium effort at job $j$ is $e^*_{ij}=\beta \theta_i(1+\lambda_i m_{ij})$, the firm's equilibrium bonus is $\beta^*=1/2$.
\end{proposition}

    
\end{frame}


\begin{frame}{Solving the Model}

\begin{proposition}
    The worker will take the job $p$ rather than the outside option job $a$ if:
    \[\alpha_p -\alpha_a \geq \frac{\lambda_i \theta_i^2}{8} (m_{ia}-m_{ip}) \]
\end{proposition}
Two types of workers at the firm:
\begin{wideitemize}
    \item Those there for money: $\alpha_p-\alpha_a>>0$.
    \item Those there for meaning: $m_{ip}-m_{ia}>>0$.
\end{wideitemize}

    
\end{frame}

\begin{frame}{The Experiment Increases $\lambda_i$ from $\lambda_C\to \lambda_T$ }
The model replicates two impacts of the experiment:
\begin{wideitemize}
    \item[1.] Workers there for the salary tend to exit, workers there for meaning tend to stay, and this raises productivity.
    \[\text{ average } m_{ip} \uparrow \implies \text{ average }  c(e) \downarrow \implies \text{ average } y \uparrow  \]

    Similar to the selection effect in Lazear (2000)!
    
    \item[2.] Workers who stay become more productive.
\[ e^*_T=\frac{1}{2} \theta_i(1+\lambda_T m_{ij})>\frac{1}{2} \theta_i(1+\lambda_C m_{ij})=e^*_{C} \]
       Similar to the productivity effect in Lazear (2000)!
     
\end{wideitemize}

    
\end{frame}

\begin{frame}{Verification 1: Exit Increases Productivity}
\centering
 \includegraphics[width=0.8\textwidth]{pictures/meaing_exit.png}

    
\end{frame}
\begin{frame}{Verification 2: Stayers Productivity Increased}
\centering
 \includegraphics[width=0.8\textwidth]{pictures/meaning_performance.png}

    
\end{frame}


\begin{frame}{The Experiment Increases $\lambda_i$ from $\lambda_C\to \lambda_T$ }
The model has several additional implications:
\begin{wideitemize}
    \item[1.] The meaning-pay frontier will flatten.

    High salary, low meaning workers leave because they start to value meaning.
    
    \item[2.] Treated workers that stay have increased utility.

Compute utility on the board.
     
\end{wideitemize}

    
\end{frame}

\begin{frame}{Additional Implication 1: Flattening the Meaning-Pay Frontier}
\centering
 \includegraphics[width=0.7\textwidth]{pictures/meaing_frontier.png}

    
\end{frame}
\begin{frame}{Additional Implication 2: Increased Utility for Treated Stayers}
\centering
 \includegraphics[width=0.7\textwidth]{pictures/meaing_utility.png}

    
\end{frame}




\begin{frame}{Who Chose to Find Their Meaning?}

\centering

 \includegraphics[width=0.6\textwidth]{pictures/meaning_compliers.png}

    
\end{frame}

\end{document}







