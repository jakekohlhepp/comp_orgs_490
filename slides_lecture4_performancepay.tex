%%%%%%%%%%%%%%%%%%%%%%%%%%%%%%%%%%%%%%%%%
% Beamer Presentation
% LaTeX Template
% Version 1.0 (10/11/12)
%
% This template has been downloaded from:
% http://www.LaTeXTemplates.com
%
% License:
% CC BY-NC-SA 3.0 (http://creativecommons.org/licenses/by-nc-sa/3.0/)
%
%%%%%%%%%%%%%%%%%%%%%%%%%%%%%%%%%%%%%%%%%

%----------------------------------------------------------------------------------------
%	PACKAGES AND THEMES
%----------------------------------------------------------------------------------------

\documentclass[aspectratio=169,usenames,dvipsnames]{beamer}

\usepackage[utf8]{inputenc}
\usepackage{booktabs}
\usepackage{tabularx}
\usepackage[authordate,bibencoding=auto,strict,backend=biber,natbib]{biblatex-chicago}
\addbibresource{bib.bib}
\usepackage{graphicx}
% \hypersetup{
%     colorlinks,
%     %citecolor=black,
%     linkcolor=black
% }
\usepackage{array}
\usepackage{caption}
\usepackage{threeparttable}
\usepackage{epigraph} 
\usepackage{lscape}
\usepackage{adjustbox}
\newcommand*{\Scale}[2][4]{\scalebox{#1}{\ensuremath{#2}}}%
\usepackage{import}
\newenvironment{wideitemize}{\itemize\addtolength{\itemsep}{10pt}}{\enditemize}
\usepackage{amsmath}
\usepackage{csvsimple}
\usepackage{siunitx}
\usepackage{filecontents}
\usepackage{rotating}
\usepackage{multirow}
\usepackage{amsmath}
\usepackage{subcaption}
\usepackage{appendixnumberbeamer}
\usepackage{float}
\usepackage{amsmath}
\usepackage{csvsimple}
\usepackage{hyperref}
\newtheorem{proposition}{Proposition}
\usepackage{xcolor}
\def\boxit#1#2{%
    \smash{\color{red}\fboxrule=1pt\relax\fboxsep=2pt\relax%
    \llap{\rlap{\fbox{\phantom{\rule{#1}{#2}}}}~}}\ignorespaces
}
\newenvironment{variableblock}[3]{%
  \setbeamercolor{block body}{#2}
  \setbeamercolor{block title}{#3}
  \begin{block}{#1}}{\end{block}}
\usepackage{appendixnumberbeamer}
\usepackage{tikz,pgfplots}
\usepackage{tkz-fct}
\usepackage{amsthm}
\pgfplotsset{compat=1.10}
\usepgfplotslibrary{fillbetween}
\mode<presentation> {
\AtBeginSection[]
{
    \begin{frame}
        \frametitle{Table of Contents}
        \tableofcontents[currentsection]
    \end{frame}
}
% The Beamer class comes with a number of default slide themes
% which change the colors and layouts of slides. Below this is a list
% of all the themes, uncomment each in turn to see what they look like.

\usetheme{default}
%\usetheme{AnnArbor}
%\usetheme{Antibes} -
%\usetheme{Bergen}
%\usetheme{Berkeley}
%\usetheme{Berlin}
%\usetheme{Boadilla}
%\usetheme{CambridgeUS}
%\usetheme{Copenhagen} -
%\usetheme{Darmstadt}
%\usetheme{Dresden}
%\usetheme{Frankfurt}
%\usetheme{Goettingen}
%\usetheme{Hannover}
%\usetheme{Ilmenau}
%\usetheme{JuanLesPins}
%\usetheme{Luebeck}
%\usetheme{Madrid}
%\usetheme{Malmoe}
%\usetheme{Marburg}
%\usetheme{Montpellier}
%\usetheme{PaloAlto}
%\usetheme{Pittsburgh}
%\usetheme{Rochester} -
%\usetheme{Singapore}
%\usetheme{Szeged}
%\usetheme{Warsaw}

% As well as themes, the Beamer class has a number of color themes
% for any slide theme. Uncomment each of these in turn to see how it
% changes the colors of your current slide theme.

%\usecolortheme{albatross}
%\usecolortheme{beaver}
%\usecolortheme{beetle}
%\usecolortheme{crane}
%\usecolortheme{dolphin}
%\usecolortheme{dove}
%\usecolortheme{fly}
%\usecolortheme{lily}
%\usecolortheme{orchid}
%\usecolortheme{rose}
%\usecolortheme{seagull}
%\usecolortheme{seahorse}
%\usecolortheme{whale}
%\usecolortheme{wolverine}

%\setbeamertemplate{footline} % To remove the footer line in all slides uncomment this line
%\setbeamertemplate{footline}[frame number] % To replace the footer line in all slides with a simple slide count uncomment this line
\setbeamertemplate{theorems}[numbered]
\setbeamertemplate{navigation symbols}{} % To remove the navigation symbols from the bottom of all slides uncomment this line
}
\setbeamertemplate{caption}{\raggedright\insertcaption\par}
  \setbeamertemplate{enumerate items}[default]
  %\setbeamertemplate{page number in head/foot}{\insertframenumber}
\usepackage{graphicx} % Allows including images
\usepackage{booktabs} % Allows the use of \toprule, \midrule and \bottomrule in tables
%\usepackage {tikz}
\newtheorem*{theorem*}{Theorem}
\newtheorem*{lemma*}{Lemma}
\newtheorem*{proposition*}{Proposition}
\newtheorem*{corollary*}{Corollary}
\newtheorem*{definition*}{Definition}
\DeclareMathOperator*{\argmin}{arg\,min}
\newtheorem*{assumption}{Assumption}
\usetikzlibrary {positioning}
\renewcommand{\arraystretch}{1.5}
\newcommand\hideit[1]{%
  \only<0| handout:1>{\mbox{}}%
  \invisible<0| handout:1>{#1}}
\usepackage[default]{lato}

\setbeamercolor{block body alerted}{bg=alerted text.fg!10}
\setbeamercolor{block title alerted}{bg=alerted text.fg!20}
\setbeamercolor{block body}{bg=structure!10}
\setbeamercolor{block title}{bg=structure!20}
\setbeamercolor{block body example}{bg=green!10}
\setbeamercolor{block title example}{bg=green!20}


\makeatletter
\let\save@measuring@true\measuring@true
\def\measuring@true{%
  \save@measuring@true
  \def\beamer@sortzero##1{\beamer@ifnextcharospec{\beamer@sortzeroread{##1}}{}}%
  \def\beamer@sortzeroread##1<##2>{}%
  \def\beamer@finalnospec{}%
}
\makeatother
%\usepackage {xcolor}

%----------------------------------------------------------------------------------------
%	TITLE PAGE
%----------------------------------------------------------------------------------------

\title[diss]{Lecture 4: Performance Pay} % The short title appears at the bottom of every slide, the full title is only on the title page
\author{Compensation in Organizations} % Your name
\institute[shortinst]{Jacob Kohlhepp}
\date{\today} % Date, can be changed to a custom date

\begin{document}

\begin{frame}
\titlepage % Print the title page as the first slide

\end{frame}

\begin{frame}
\centering
    \huge Discussion: Loyalka et. al. (2019)
\end{frame}

\begin{frame}{The Principal-Agent Model}
\begin{block}{Players}
    \begin{wideitemize}
    \item There is a firm (the principal) who is risk neutral (exponential utility with parameter $r=0$).
    \item There is a worker (the agent) who is risk averse (exponential utility with parameter $r\geq 0$).
\end{wideitemize}
\end{block}
\begin{block}{Actions}
    \begin{wideitemize}
    \item Firm chooses a linear wage which depends on effort ($w(e)$) or output ($w(y)$)
    \item After seeing the wage, the worker either accepts or rejects the job.
    \item If they accept, worker chooses effort $e$ at an increasing, convex cost $c(e)$
    % \begin{wideitemize}
    %     \item Increasing means $c'(e)>0$, convex means $c''(e)>0$
    % \end{wideitemize}
\end{wideitemize}
\end{block}
\end{frame}
\begin{frame}{The Principal-Agent Model}
\begin{block}{Output}
    \begin{wideitemize}
    \item Output is effort ($e$) plus noise/luck ($\epsilon$): $y=e+\epsilon$ where $\epsilon\sim N(0,\sigma^2)$
    \item This implies output is normal with mean $e$ and variance $\sigma^2$
\end{wideitemize}
\end{block}
\begin{block}{Payoffs}
    \begin{wideitemize}
    \item If accepted, firm's payoff $\pi$ is expected output minus expected wages: $E[y-w|e]$
    \item If accepted, worker's payoff is expected utility of the wage minus effort cost: $E[u(w) -c(e)|e]$
    \item If rejected, worker has ``outside option" of $\bar u$ and firm has ``outside option" of 0.\footnote{We will assume throughout that the firm prefers to hire the worker ex-ante.}
\end{wideitemize}
\end{block}
\end{frame}
\begin{frame}{Timing}
\centering
    \huge See the board!
\end{frame}

\section{Recap: Effort-Based Pay}

\begin{frame}{Recap: Effort-Based Pay}
    \begin{wideitemize}
        \item Suppose the firm can pay based on the worker's effort.
        \item Then wage is a linear function of effort: $w(e)=\alpha + \beta e$
        \item We now go to the board to solve!
    \end{wideitemize}
\end{frame}
\begin{frame}{Recap: Effort-Based Pay}
    \begin{theorem}
        When wages depend directly on effort, effort is $e^*$ which solves $c'(e^*)=1$ and $\beta^*=1, \alpha^*=\bar u+c(e^*)-1$
    \end{theorem}

\end{frame}

\section{Performance-Based Pay}

\begin{frame}{Performance-Based Pay}
    \begin{wideitemize}
        \item Suppose the firm can pay based ONLY on output $y$
        \item Then wage is a linear function of output: $w(y)=\alpha + \beta y$
        \item We now go to the board to solve!
    \end{wideitemize}
\end{frame}

\begin{frame}{Performance-Based Pay}
    \begin{theorem}
        When wages depend only on output, effort is $e_{p}$ which solves 
        \[c'(e_{p})= \frac{1}{1+r \sigma^2 c''(e_{p})}\]
        and $\beta_{p} =c'(e_{p}),\alpha_{p} =\bar u - \beta_{p} e_{p}+r \beta^2\sigma^2/2+c(e_{p})$.
    \end{theorem}

    \begin{wideitemize}
        \item Notice that $\frac{1}{1+r \sigma^2 c''(e_{p})}<1$.
        \item Therefore we are getting less than surplus maximizing effort: $e_{p}<e^*$
        \item Performance pay generates inefficiency relative to effort based pay!
    \end{wideitemize}
\end{frame}

\begin{frame}{Performance-Based Pay: Explicit Cost Function}
    \begin{wideitemize}
        \item Suppose that $c(e)=e^2/2$
        \item Let's work it out on the board!
        \pause
        \item Under this quadratic effort cost:
        \[e_p =\beta_p = \frac{1}{1+r \sigma^2}  \]
        \[\alpha_p =  \frac{r \sigma^2-1}{2} \bigg ( \frac{1}{1+r \sigma^2} \bigg )^2-\bar u \]
    \end{wideitemize}
\end{frame}

\begin{frame}{Interpreting Results}
    \begin{wideitemize}
        \item $\beta$ is the average amount of money paid to the worker per unit of effort.
        \item $\beta$ represents the strength of incentives (question: why?)
        \item We say incentives are high-powered when $\beta$ is high (close to 1)
        \item Because $\beta=c'(e)$ we have that:
        \[\beta_p =c'(e_{p})= \frac{1}{1+r \sigma^2 c''(e_{p})}\]
    \end{wideitemize}
\end{frame}
\begin{frame}{Interpreting Results}
    \begin{wideitemize}
        \item Because $\beta=c'(e)$ we have that:
        \[\beta_p =c'(e_{p})= \frac{1}{1+r \sigma^2 c''(e_{p})}\]
        \item The strength of incentives rises when:
        \begin{wideitemize}
            \item risk-aversion decreases $\downarrow r$ (question: what if $r=0$?)
            \item noise/luck becomes less important $\downarrow \sigma^2$ (question: what if $\sigma^2=0$?)
            \item the marginal-marginal cost of effort decreases $\downarrow c''(e_p)$
            \begin{wideitemize}
                \item a high $c''(e_p)$ means working one more hour after already working 8 hours is much harder than working one more hour after working 1 hour.
            \end{wideitemize}
        \end{wideitemize}
    \end{wideitemize}
\end{frame}

\begin{frame}
\centering
    \huge Discussion: Drago and Garvey (1998)
\end{frame}


\end{document}







