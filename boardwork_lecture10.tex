\documentclass{article}
\usepackage{graphicx} % Required for inserting images
\usepackage{amsmath} 
\newtheorem{lemma}{lemma}
\title{Board Work for Lecture 10: Wells Fargo and Multitasking}
\author{Jacob Kohlhepp}
\date{\today}

\begin{document}

\maketitle


\section{Destructive Effort}

\subsection{Setup}
\begin{itemize}
    \item Output depends on a productive and destructive task:
    \[y = 1+  ae_1 -b e_2, a>0, b>0\]
    \item The $1$ is latent profit from operating at all.
    \item The cost of effort is:
    \[c(e_1, e_2) = (e_1^2+e_2^2)/2 \]
    \item We measure and pay only based on total effort:
    \[m=e_1+e_2\]
    \[w(m)=\alpha + \beta m\]
\end{itemize}

\subsection{Solution}

As always, we start with the agent's choice of effort given the wages are already set:
\[\max_{e_1, e_2} \alpha + \beta (e_1+e_2)-(e_1^2+e_2^2)/2 \]
\[[e_1:] \beta -e_1=0 \implies e_1=\beta \]
\[[e_2:] \beta -e_2=0 \implies e_2=\beta \]
Now we plug this in to get the utility the agent expects if they accept the wage:
\[u(accept) = \alpha + \beta (\beta +\beta)-(\beta^2+\beta^2)/2\]
It is helpful to simplify here a bit:
\[u(accept) = \alpha + 2\beta^2-\beta^2= \alpha + \beta^2\]
The firm always sets $\alpha$ to make the worker indifferent to accepting or rejecting:
\[\alpha + \beta^2=0 \leftrightarrow \alpha = -\beta^2\]
The firm's profit is:
\[\pi = 1+  ae_1 -b e_2 -\alpha -\beta m\]
plugging everything in:
\[\pi = 1+  a\beta -b \beta +\beta^2 -2\beta^2=1+  (a-b) \beta -\beta^2\]
The firm maximizes this with respect to $\beta$. Let's suppose everything is okay and we can take the FOC:
\[[\beta:] (a-b) -2\beta=0 \leftrightarrow \beta^*= \frac{a-b}{2}\]
But wait. This entire stream of logic only works if $a-b>0$. What if $a-b<0$? Then the marginal destructive effort we get with more incentives outweighs any productive effort. If that is true, more incentives only hurts profit. So the firm should set $\beta^* =0$. This is the final result!

Profit is then given in two cases:
\[\pi^* = \begin{cases}1+(a-b)^2/4\text{ if } a-b>0\\
1 \text{ if } a-b\leq 0\end{cases}\]
If the firm could do everything itself (first-best) what would it do? To find out, maximize surplus allowing the firm to pick efforts directly:
\[\max_{e_1, e_2} 1+ ae_1 -be_2 -(e_1^2+e_2^2)/2\]
Notice that $e_2$ is wasteful: it takes costly effort and only hurts surplus. Thus set $e_2^{FB}=0$. Then solve for $e_1$:
\[[e_1:] a -e_1=0 \leftrightarrow e_1^{FB}=a\]
Choose the amount of $e_1$ where marginal cost is equal to marginal benefit! profit under the first-best is:
\[\pi^{FB} = 1+ ae_1 -be_2 -(e_1^2+e_2^2)/2 = 1+a^2/2\]
Even if $b$ is very close to 0, we are still losing $a^2/4$ in profit from multitasking inefficiency. One way the firm could get this money back is to prohibit task 2. Then the firm can just set $\beta=a$ and everything is resolved. But is this realistic? Do firms always have the power to prohibit an activity entirely?



\end{document}