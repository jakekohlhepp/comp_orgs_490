%%%%%%%%%%%%%%%%%%%%%%%%%%%%%%%%%%%%%%%%%
% Beamer Presentation
% LaTeX Template
% Version 1.0 (10/11/12)
%
% This template has been downloaded from:
% http://www.LaTeXTemplates.com
%
% License:
% CC BY-NC-SA 3.0 (http://creativecommons.org/licenses/by-nc-sa/3.0/)
%
%%%%%%%%%%%%%%%%%%%%%%%%%%%%%%%%%%%%%%%%%

%----------------------------------------------------------------------------------------
%	PACKAGES AND THEMES
%----------------------------------------------------------------------------------------

\documentclass[aspectratio=169,usenames,dvipsnames]{beamer}

\usepackage[utf8]{inputenc}
\usepackage{booktabs}
\usepackage{tabularx}
\usepackage[authordate,bibencoding=auto,strict,backend=biber,natbib]{biblatex-chicago}
\addbibresource{bib.bib}
\usepackage{graphicx}
% \hypersetup{
%     colorlinks,
%     %citecolor=black,
%     linkcolor=black
% }
\usepackage{array}
\usepackage{caption}
\usepackage{threeparttable}
\usepackage{epigraph} 
\usepackage{lscape}
\usepackage{adjustbox}
\newcommand*{\Scale}[2][4]{\scalebox{#1}{\ensuremath{#2}}}%
\usepackage{import}
\newenvironment{wideitemize}{\itemize\addtolength{\itemsep}{10pt}}{\enditemize}
\usepackage{amsmath}
\usepackage{csvsimple}
\usepackage{siunitx}
\usepackage{filecontents}
\usepackage{rotating}
\usepackage{multirow}
\usepackage{amsmath}
\usepackage{subcaption}
\usepackage{appendixnumberbeamer}
\usepackage{float}
\usepackage{amsmath}
\usepackage{csvsimple}
\usepackage{hyperref}
\newtheorem{proposition}{Proposition}
\usepackage{xcolor}
\def\boxit#1#2{%
    \smash{\color{red}\fboxrule=1pt\relax\fboxsep=2pt\relax%
    \llap{\rlap{\fbox{\phantom{\rule{#1}{#2}}}}~}}\ignorespaces
}
\newenvironment{variableblock}[3]{%
  \setbeamercolor{block body}{#2}
  \setbeamercolor{block title}{#3}
  \begin{block}{#1}}{\end{block}}
\usepackage{appendixnumberbeamer}
\usepackage{tikz,pgfplots}
\usepackage{tkz-fct}
\usepackage{amsthm}
\pgfplotsset{compat=1.10}
\usepgfplotslibrary{fillbetween}
\mode<presentation> {
\AtBeginSection[]
{
    \begin{frame}
        \frametitle{Table of Contents}
        \tableofcontents[currentsection]
    \end{frame}
}
% The Beamer class comes with a number of default slide themes
% which change the colors and layouts of slides. Below this is a list
% of all the themes, uncomment each in turn to see what they look like.

\usetheme{default}
%\usetheme{AnnArbor}
%\usetheme{Antibes} -
%\usetheme{Bergen}
%\usetheme{Berkeley}
%\usetheme{Berlin}
%\usetheme{Boadilla}
%\usetheme{CambridgeUS}
%\usetheme{Copenhagen} -
%\usetheme{Darmstadt}
%\usetheme{Dresden}
%\usetheme{Frankfurt}
%\usetheme{Goettingen}
%\usetheme{Hannover}
%\usetheme{Ilmenau}
%\usetheme{JuanLesPins}
%\usetheme{Luebeck}
%\usetheme{Madrid}
%\usetheme{Malmoe}
%\usetheme{Marburg}
%\usetheme{Montpellier}
%\usetheme{PaloAlto}
%\usetheme{Pittsburgh}
%\usetheme{Rochester} -
%\usetheme{Singapore}
%\usetheme{Szeged}
%\usetheme{Warsaw}

% As well as themes, the Beamer class has a number of color themes
% for any slide theme. Uncomment each of these in turn to see how it
% changes the colors of your current slide theme.

%\usecolortheme{albatross}
%\usecolortheme{beaver}
%\usecolortheme{beetle}
%\usecolortheme{crane}
%\usecolortheme{dolphin}
%\usecolortheme{dove}
%\usecolortheme{fly}
%\usecolortheme{lily}
%\usecolortheme{orchid}
%\usecolortheme{rose}
%\usecolortheme{seagull}
%\usecolortheme{seahorse}
%\usecolortheme{whale}
%\usecolortheme{wolverine}

%\setbeamertemplate{footline} % To remove the footer line in all slides uncomment this line
%\setbeamertemplate{footline}[frame number] % To replace the footer line in all slides with a simple slide count uncomment this line
\setbeamertemplate{theorems}[numbered]
\setbeamertemplate{navigation symbols}{} % To remove the navigation symbols from the bottom of all slides uncomment this line
}
\setbeamertemplate{caption}{\raggedright\insertcaption\par}
  \setbeamertemplate{enumerate items}[default]
  %\setbeamertemplate{page number in head/foot}{\insertframenumber}
\usepackage{graphicx} % Allows including images
\usepackage{booktabs} % Allows the use of \toprule, \midrule and \bottomrule in tables
%\usepackage {tikz}
\newtheorem*{theorem*}{Theorem}
\newtheorem*{lemma*}{Lemma}
\newtheorem*{proposition*}{Proposition}
\newtheorem*{corollary*}{Corollary}
\newtheorem*{definition*}{Definition}
\DeclareMathOperator*{\argmin}{arg\,min}
\newtheorem*{assumption}{Assumption}
\usetikzlibrary {positioning}
\renewcommand{\arraystretch}{1.5}
\newcommand\hideit[1]{%
  \only<0| handout:1>{\mbox{}}%
  \invisible<0| handout:1>{#1}}
\usepackage[default]{lato}

\setbeamercolor{block body alerted}{bg=alerted text.fg!10}
\setbeamercolor{block title alerted}{bg=alerted text.fg!20}
\setbeamercolor{block body}{bg=structure!10}
\setbeamercolor{block title}{bg=structure!20}
\setbeamercolor{block body example}{bg=green!10}
\setbeamercolor{block title example}{bg=green!20}


\makeatletter
\let\save@measuring@true\measuring@true
\def\measuring@true{%
  \save@measuring@true
  \def\beamer@sortzero##1{\beamer@ifnextcharospec{\beamer@sortzeroread{##1}}{}}%
  \def\beamer@sortzeroread##1<##2>{}%
  \def\beamer@finalnospec{}%
}
\makeatother
%\usepackage {xcolor}

%----------------------------------------------------------------------------------------
%	TITLE PAGE
%----------------------------------------------------------------------------------------

\title[diss]{Lecture 3: The Principal-Agent Model} % The short title appears at the bottom of every slide, the full title is only on the title page
\author{Compensation in Organizations} % Your name
\institute[shortinst]{Jacob Kohlhepp}
\date{\today} % Date, can be changed to a custom date

\begin{document}

\begin{frame}
\titlepage % Print the title page as the first slide

\end{frame}

\begin{frame}
\centering
    \huge Discussion: Lazear (2000)
\end{frame}

\begin{frame}{Why a Model? (Clarity of Thought)}

\begin{wideitemize}
    \item ``All models are wrong, but some are useful." -George Box
    \item Maps are useful precisely because they ignore some features.
    \item A good model has what we need to study one question, and nothing more.
    \item Under the model's assumptions, we know the truth.
    \item If the outcome is unrealistic, we can argue about which precise assumptions led to it.
\end{wideitemize}
    
\end{frame}

\begin{frame}{Why THIS Model?}

\begin{wideitemize}
    \item The model we study in this lecture will be the main model of this class.
    \item We will slightly change it to study different questions.
    \item It was developed in the 1970s and has framed theoretical and empirical research for decades.
\end{wideitemize}
    
\end{frame}

\begin{frame}{The Principal-Agent Model}
\begin{block}{Players}
    \begin{wideitemize}
    \item There is a firm (the principal) who is risk neutral (exponential utility with parameter $r=0$).
    \item There is a worker (the agent) who is risk averse (exponential utility with parameter $r=r\geq 0$).
\end{wideitemize}
\end{block}
\begin{block}{Actions}
    \begin{wideitemize}
    \item Firm chooses a linear wage which depends on effort ($w(e)$) or output ($w(y)$)
    \item After seeing the wage, the worker either accepts or rejects the job.
    \item If they accept, worker chooses effort $e$ at an increasing, convex cost $c(e)$
    % \begin{wideitemize}
    %     \item Increasing means $c'(e)>0$, convex means $c''(e)>0$
    % \end{wideitemize}
\end{wideitemize}
\end{block}
\end{frame}
\begin{frame}{The Principal-Agent Model}
\begin{block}{Output}
    \begin{wideitemize}
    \item Output is effort ($e$) plus noise/luck ($\epsilon$): $y=e+\epsilon$ where $\epsilon\sim N(0,\sigma^2)$
    \item This implies output is normal with mean $e$ and variance $\sigma^2$
\end{wideitemize}
\end{block}
\begin{block}{Payoffs}
    \begin{wideitemize}
    \item If accepted, firm's payoff $\pi$ is expected output minus expected wages: $E[y-w|e]$
    \item If accepted, worker's payoff is expected utility of the wage minus effort cost: $E[u(w) -c(e)|e]$
    \item If rejected, worker has ``outside option" of $\bar u$ and firm has ``outside option" of 0
\end{wideitemize}
\end{block}
\end{frame}
\begin{frame}{Timing}
\centering
    \huge See the board!
\end{frame}

\begin{frame}{Effort-Based Pay}
    \begin{wideitemize}
        \item Suppose the firm can pay based on the worker's effort.
        \item Then wage is a linear function of effort: $w(e)=\alpha + \beta e$
        \item We now go to the board to solve!
    \end{wideitemize}
\end{frame}
\begin{frame}{Effort-Based Pay}
   \begin{theorem}
        When wages depend directly on effort, effort is $e^*$ which solves $c'(e^*)=1$ and $\beta^*=1, \alpha^*=\bar u+c(e^*)-1$
    \end{theorem}

    \begin{wideitemize}
        \item We reward the worker for more effort.
        \item Question: What is the worker's expected payoff?
        \item Question: Why is the marginal cost set equal to 1?
        \item Let's compute the firm's payoff and total surplus on the board.
    \end{wideitemize}
\end{frame}

\begin{frame}{Effort-Based Pay}
    \begin{wideitemize}
        \item We showed that this is the first-best: the firm could not do better even if they did the work themselves!
        \item The firm gets everything, but if the worker proposed they would get everything
        \item Question: What assumption did we bake in that got us here?
        \item Question: Is this assumption realistic?
    \end{wideitemize}
\end{frame}

\begin{frame}{Selling the Firm}
 \begin{theorem}
        When wages depend directly on effort, effort is $e^*$ which solves $c'(e^*)=1$ and $\beta^*=1, \alpha^*=\bar u+c(e^*)-1$
    \end{theorem}
    \begin{wideitemize}
        \item Consider our result: $\beta=1$ and the average return to effort is $1$
        \item The firm gives the worker everything they produce.
        \item This is called selling the firm: the worker pays $\alpha$ but keeps everything they make!
        \item These are extremely strong incentives (why?)
    \end{wideitemize}
\end{frame}


\begin{frame}{Paying to Work}
    \begin{wideitemize}
        \item If the worker's outside option is low enough we will have that $\alpha$ is negative.
        \pause
        \item This means the worker is paying to work!
         \pause
        \item Does this ever happen?
         \pause
        \item Question: Can you give some examples?
        %hair salons booth renters, franchises, food stands in stadiums
    \end{wideitemize}
\end{frame}



\end{document}







