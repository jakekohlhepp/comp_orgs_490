



\documentclass{article}
\usepackage{graphicx} % Required for inserting images
\usepackage{amsmath} 
\newtheorem{lemma}{lemma}
\usepackage{geometry}
\geometry{margin=1in}
\usepackage{hyperref}
\hypersetup{
    colorlinks=true,
    linkcolor=blue,
    filecolor=magenta,      
    urlcolor=cyan,
    pdftitle={Overleaf Example},
    pdfpagemode=FullScreen,
    }
    
\title{Problem Set 0}
\author{Jacob Kohlhepp}
\date{\today}

\begin{document}

\maketitle


The purpose of this problem set is to practice using Kritik.

\section{Setup}

Create an account on Kritik. You should have received an email with a link to setup an account. If you did not, check your spam. If you still did not, submit a request through the Logistics google form: \url{https://forms.gle/baxnFKnunuSzoDbs9}

For all four problem sets, the submission process will be as follows:

\begin{enumerate}
    \item I will post a PDF of the questions on the Canvas course website.
    \item You will work on the assignment alone.
    \item If you need help, you can then work with others in the class.
    \item Each person must write up their answers individually. You can type them or hand write and take pictures. Whatever way you choose, the submission file must be a PDF. There are many ways online to convert pictures or other files to PDFs.
    \item DO NOT WRITE YOUR NAME ANYWHERE ON THE SUBMISSION DOCUMENT. This is critical for peer evaluation.
    \item Return to the Kritik website and submit your assignment.
    \item After the deadline, I will upload an answer key to the Canvas website.
    \item Each person will be assigned three problem sets to peer evaluate on Kritik
    \item After you complete your evaluation, the person you evaluated can evaluate your evaluation.
\end{enumerate}

In order to familiarize everyone with the system, I have designed problem set 0. For this problem set, go through all the steps in the last section (including peer evaluation) for the question described below. When peer evaluating, follow the rubric.

\section{Question}

\begin{enumerate}
    \item In 3 or 4 sentences, describe your career plans. If your plans are not certain, describe one possibility.
\end{enumerate}






\end{document}