%%%%%%%%%%%%%%%%%%%%%%%%%%%%%%%%%%%%%%%%%
% Beamer Presentation
% LaTeX Template
% Version 1.0 (10/11/12)
%
% This template has been downloaded from:
% http://www.LaTeXTemplates.com
%
% License:
% CC BY-NC-SA 3.0 (http://creativecommons.org/licenses/by-nc-sa/3.0/)
%
%%%%%%%%%%%%%%%%%%%%%%%%%%%%%%%%%%%%%%%%%

%----------------------------------------------------------------------------------------
%	PACKAGES AND THEMES
%----------------------------------------------------------------------------------------

\documentclass[aspectratio=169,usenames,dvipsnames]{beamer}

\usepackage[utf8]{inputenc}
\usepackage{booktabs}
\usepackage{tabularx}
\usepackage{amsmath}
\usepackage[authordate,bibencoding=auto,strict,backend=biber,natbib]{biblatex-chicago}
\addbibresource{bib.bib}
\usepackage{graphicx}
% \hypersetup{
%     colorlinks,
%     %citecolor=black,
%     linkcolor=black
% }
\usepackage{array}
\usepackage{caption}
\usepackage{threeparttable}
\usepackage{epigraph} 
\usepackage{lscape}
\usepackage{adjustbox}
\newcommand*{\Scale}[2][4]{\scalebox{#1}{\ensuremath{#2}}}%
\usepackage{import}
\newenvironment{wideitemize}{\itemize\addtolength{\itemsep}{10pt}}{\enditemize}
\usepackage{amsmath}
\usepackage{csvsimple}
\usepackage{siunitx}
\usepackage{filecontents}
\usepackage{rotating}
\usepackage{multirow}
\usepackage{amsmath}
\usepackage{subcaption}
\usepackage{appendixnumberbeamer}
\usepackage{float}
\usepackage{amsmath}
\usepackage{csvsimple}
\usepackage{hyperref}
\newtheorem{proposition}{Proposition}
\usepackage{xcolor}
\def\boxit#1#2{%
    \smash{\color{red}\fboxrule=1pt\relax\fboxsep=2pt\relax%
    \llap{\rlap{\fbox{\phantom{\rule{#1}{#2}}}}~}}\ignorespaces
}
\newenvironment{variableblock}[3]{%
  \setbeamercolor{block body}{#2}
  \setbeamercolor{block title}{#3}
  \begin{block}{#1}}{\end{block}}
\usepackage{appendixnumberbeamer}
\usepackage{tikz,pgfplots}
\usepackage{tkz-fct}
\usepackage{amsthm}
\pgfplotsset{compat=1.10}
\usepgfplotslibrary{fillbetween}
\mode<presentation> {
\AtBeginSection[]
{
    \begin{frame}
        \frametitle{Table of Contents}
        \tableofcontents[currentsection]
    \end{frame}
}
% The Beamer class comes with a number of default slide themes
% which change the colors and layouts of slides. Below this is a list
% of all the themes, uncomment each in turn to see what they look like.

\usetheme{default}
%\usetheme{AnnArbor}
%\usetheme{Antibes} -
%\usetheme{Bergen}
%\usetheme{Berkeley}
%\usetheme{Berlin}
%\usetheme{Boadilla}
%\usetheme{CambridgeUS}
%\usetheme{Copenhagen} -
%\usetheme{Darmstadt}
%\usetheme{Dresden}
%\usetheme{Frankfurt}
%\usetheme{Goettingen}
%\usetheme{Hannover}
%\usetheme{Ilmenau}
%\usetheme{JuanLesPins}
%\usetheme{Luebeck}
%\usetheme{Madrid}
%\usetheme{Malmoe}
%\usetheme{Marburg}
%\usetheme{Montpellier}
%\usetheme{PaloAlto}
%\usetheme{Pittsburgh}
%\usetheme{Rochester} -
%\usetheme{Singapore}
%\usetheme{Szeged}
%\usetheme{Warsaw}

% As well as themes, the Beamer class has a number of color themes
% for any slide theme. Uncomment each of these in turn to see how it
% changes the colors of your current slide theme.

%\usecolortheme{albatross}
%\usecolortheme{beaver}
%\usecolortheme{beetle}
%\usecolortheme{crane}
%\usecolortheme{dolphin}
%\usecolortheme{dove}
%\usecolortheme{fly}
%\usecolortheme{lily}
%\usecolortheme{orchid}
%\usecolortheme{rose}
%\usecolortheme{seagull}
%\usecolortheme{seahorse}
%\usecolortheme{whale}
%\usecolortheme{wolverine}

%\setbeamertemplate{footline} % To remove the footer line in all slides uncomment this line
%\setbeamertemplate{footline}[frame number] % To replace the footer line in all slides with a simple slide count uncomment this line
\setbeamertemplate{theorems}[numbered]
\setbeamertemplate{navigation symbols}{} % To remove the navigation symbols from the bottom of all slides uncomment this line
}
\setbeamertemplate{caption}{\raggedright\insertcaption\par}
  \setbeamertemplate{enumerate items}[default]
  %\setbeamertemplate{page number in head/foot}{\insertframenumber}
\usepackage{graphicx} % Allows including images
\usepackage{booktabs} % Allows the use of \toprule, \midrule and \bottomrule in tables
%\usepackage {tikz}
\newtheorem*{theorem*}{Theorem}
\newtheorem*{lemma*}{Lemma}
\newtheorem*{proposition*}{Proposition}
\newtheorem*{corollary*}{Corollary}
\newtheorem*{definition*}{Definition}
\DeclareMathOperator*{\argmin}{arg\,min}
\newtheorem*{assumption}{Assumption}
\usetikzlibrary {positioning}
\renewcommand{\arraystretch}{1.5}
\newcommand\hideit[1]{%
  \only<0| handout:1>{\mbox{}}%
  \invisible<0| handout:1>{#1}}
\usepackage[default]{lato}

\setbeamercolor{block body alerted}{bg=alerted text.fg!10}
\setbeamercolor{block title alerted}{bg=alerted text.fg!20}
\setbeamercolor{block body}{bg=structure!10}
\setbeamercolor{block title}{bg=structure!20}
\setbeamercolor{block body example}{bg=green!10}
\setbeamercolor{block title example}{bg=green!20}


\makeatletter
\let\save@measuring@true\measuring@true
\def\measuring@true{%
  \save@measuring@true
  \def\beamer@sortzero##1{\beamer@ifnextcharospec{\beamer@sortzeroread{##1}}{}}%
  \def\beamer@sortzeroread##1<##2>{}%
  \def\beamer@finalnospec{}%
}
\makeatother
%\usepackage {xcolor}

%----------------------------------------------------------------------------------------
%	TITLE PAGE
%----------------------------------------------------------------------------------------

\title[diss]{Lecture 22: Pay Transparency} % The short title appears at the bottom of every slide, the full title is only on the title page
\author{Compensation in Organizations} % Your name
\institute[shortinst]{Jacob Kohlhepp}
\date{\today} % Date, can be changed to a custom date

\begin{document}

\begin{frame}
\titlepage % Print the title page as the first slide

\end{frame}

\begin{frame}{What You Are Paid vs. Who Knows What You Are Paid?}
    \begin{wideitemize}
        \item Most of this class focuses on what people are paid.
        \item In this last lecture, we think about how much people know about how others are paid.
        \item Why might this matter?
    \end{wideitemize}
\end{frame}

\begin{frame}{Discussion}

\huge How much will you be paid in your first job?
% How much am I paid? How much was I paid?
%https://uncdm.northcarolina.edu/salaries/index.php#
\end{frame}
\begin{frame}{The Pay of Economics Graduates}
\centering 
\includegraphics[width=0.6\textwidth]{pictures/major_salaries.png}

\hfill Source:  NACE SALARY SURVEY (2022)
\end{frame}

\section{Pay Transparency in the UNC System}
\begin{frame}{Discussion}

\Large 

The UNC System maintains a public salary database that is current to December 31, 2024.
% How much am I paid? How much was I paid?
%https://uncdm.northcarolina.edu/salaries/index.php#
\end{frame}

\begin{frame}[shrink=4]{Pay Distribution by Job Class in Econ. Departments System-Wide}
\centering
    \input{tex_pieces/top15}
     \hfill The top 15 campus-job class pairs by median salary.
\end{frame}


\begin{frame}[shrink=4]{Pay Distribution Econ Assistant Professors}
\centering
    \input{tex_pieces/assistants}
\end{frame}

\begin{frame}[shrink=4]{Salary Inversion Across Institutions}
\centering
    \input{tex_pieces/inversion}
\end{frame}


\section{Pay Transparency in General}

\begin{frame}{Discussion}

\huge Cullen (2024)
\end{frame}


\begin{frame}{Partial Equilibrium vs. General Equilibrium}
\begin{wideitemize}
    \item \textbf{Decision Problems:} Given everything is fixed, what happens if I do x vs. y?
    \begin{itemize}
        \item Most everyday decisions.
        \item Many people (including policymakers) consider.
        \item Intuition/common sense works well here.
    \end{itemize}
    \item \textbf{Games/Partial Equilibrium:} Given you respond to me, what happens if I do x vs. y if we fix most other things?
    \begin{itemize}
        \item Most of this class (employer designs contract understanding worker will react)
        \item Some people think this way for big decisions (college, job offers, etc.)
        \item Some government policy consider this (Ledbetter case)
        \item Intuition/common sense sometimes works well here.
    \end{itemize}
    \item \textbf{General Equilibrium:} Given you respond to me and everything around us also responds, what happens if I do x vs. y?
     \begin{itemize}
        \item This class largely ignored this.
        \item People almost never think this way (and this is largely rational).
        \item With the exception of macroeconomists, many economists abstract from this.
        \item Intuition/common sense are often wrong here.
        \item But this matters for pay transparency, because we change the info. everyone has.
    \end{itemize}
\end{wideitemize}
\end{frame}


\begin{frame}{The Ledbetter Case and ``Decision Problem" Thinking}

\begin{wideitemize}
    \item Ledbetter learns the salary of a coworker doing similar work.
    \item It turns out this was true of the other (male) coworkers.
    \item Ledbetter presses charges against her company.
    \item Decision problem thinking works here, because this is a single company and one employee.
    \item This is also ``horizontal" pay transparency.
\end{wideitemize}
    
\end{frame}


\begin{frame}{The 2009 Lilly Ledbetter Fair Pay Act}

\begin{wideitemize}
    \item People have tried to use Ledbetter as an example for larger reforms.
    \item But scaling the logic from one case can be dangerous.
    \item There can be partial and general equilibrium effects.
    \item Question: What does this mean?
\end{wideitemize}
    
\end{frame}

\begin{frame}{Horizontal Pay Transparency Has Unintended Consequences}
\centering
    \includegraphics[width=0.65\textwidth]{pictures/pay_transparency.png}

\end{frame}

\begin{frame}{Unintended Consequence 1: A Partial Equilibrium Effect }
\begin{wideitemize}
    \item One explanation relates directly to this class.
    \item When a worker sees they are paid less than peers, they become discouraged.
    \item They then exert less effort and are less productive.
    \item The firm then pays even lower wages.
    \item In what ways it this similar to the logic of relational contracts?
    \item How is it not?
    % hint behavioral/fairness element. outside option did not necessarily change.
    % crucially, if discrimination is economy-wide then learning a male coworker doing the same work is paid more should have no relational contract influence.
\end{wideitemize}
\end{frame}

\begin{frame}{Unintended Consequence 2: A General Equilibrium Effect }
\begin{wideitemize}
    \item Consider an organization with $100$ workers with wage $x$ before pay transparency.
    \item If one of the workers asks for a raise of $x+1$, the firm can grant the raise at cost of $1$.
    \item After pay transparency, if the worker asks for a raise of $1$ the cost is now $100$!
    \item The firm will now negotiate more aggressively against requests for raises.
    \item And importantly, the firm can credibly say ``If I give it to you, I must give it to everyone."
    \item Finally, workers will not negotiate aggressively because they could be fired, and the benefit of a raise is shared.
\end{wideitemize}
\end{frame}

\begin{frame}{Discussion}

\huge What is the first-year total compensation of an economics PhD at a consulting firm?
% How much am I paid? How much was I paid?
%https://uncdm.northcarolina.edu/salaries/index.php#
\end{frame}

\begin{frame}{Vertical Pay Transparency (Cullen and Perez-Truglia (2022))}

\begin{wideitemize}
    \item  Employees underestimated the average salary of their manager's position by 14\%
    \item This inflates future expectations about their own earnings.
    \item Employees responded by increasing sales by 1.1\%, emails by 1.3\%, and hours by 1.5\% for each 10\% increase in the perceived manager's pay.
    \item This is a productivity improving, career-concerns-like-effect.
\end{wideitemize}
    
\end{frame}



\end{document}




