%%%%%%%%%%%%%%%%%%%%%%%%%%%%%%%%%%%%%%%%%
% Beamer Presentation
% LaTeX Template
% Version 1.0 (10/11/12)
%
% This template has been downloaded from:
% http://www.LaTeXTemplates.com
%
% License:
% CC BY-NC-SA 3.0 (http://creativecommons.org/licenses/by-nc-sa/3.0/)
%
%%%%%%%%%%%%%%%%%%%%%%%%%%%%%%%%%%%%%%%%%

%----------------------------------------------------------------------------------------
%	PACKAGES AND THEMES
%----------------------------------------------------------------------------------------

\documentclass[aspectratio=169,usenames,dvipsnames]{beamer}

\usepackage[utf8]{inputenc}
\usepackage{booktabs}
\usepackage{tabularx}
\usepackage[authordate,bibencoding=auto,strict,backend=biber,natbib]{biblatex-chicago}
\addbibresource{bib.bib}
\usepackage{graphicx}
% \hypersetup{
%     colorlinks,
%     %citecolor=black,
%     linkcolor=black
% }
\usepackage{array}
\usepackage{caption}
\usepackage{threeparttable}
\usepackage{epigraph} 
\usepackage{lscape}
\usepackage{adjustbox}
\newcommand*{\Scale}[2][4]{\scalebox{#1}{\ensuremath{#2}}}%
\usepackage{import}
\newenvironment{wideitemize}{\itemize\addtolength{\itemsep}{10pt}}{\enditemize}
\usepackage{amsmath}
\usepackage{csvsimple}
\usepackage{siunitx}
\usepackage{filecontents}
\usepackage{rotating}
\usepackage{multirow}
\usepackage{amsmath}
\usepackage{subcaption}
\usepackage{appendixnumberbeamer}
\usepackage{float}
\usepackage{amsmath}
\usepackage{csvsimple}
\usepackage{hyperref}
\newtheorem{proposition}{Proposition}
\usepackage{xcolor}
\def\boxit#1#2{%
    \smash{\color{red}\fboxrule=1pt\relax\fboxsep=2pt\relax%
    \llap{\rlap{\fbox{\phantom{\rule{#1}{#2}}}}~}}\ignorespaces
}
\newenvironment{variableblock}[3]{%
  \setbeamercolor{block body}{#2}
  \setbeamercolor{block title}{#3}
  \begin{block}{#1}}{\end{block}}
\usepackage{appendixnumberbeamer}
\usepackage{tikz,pgfplots}
\usepackage{tkz-fct}
\usepackage{amsthm}
\pgfplotsset{compat=1.10}
\usepgfplotslibrary{fillbetween}
\mode<presentation> {
\AtBeginSection[]
{
    \begin{frame}
        \frametitle{Table of Contents}
        \tableofcontents[currentsection]
    \end{frame}
}
% The Beamer class comes with a number of default slide themes
% which change the colors and layouts of slides. Below this is a list
% of all the themes, uncomment each in turn to see what they look like.

\usetheme{default}
%\usetheme{AnnArbor}
%\usetheme{Antibes} -
%\usetheme{Bergen}
%\usetheme{Berkeley}
%\usetheme{Berlin}
%\usetheme{Boadilla}
%\usetheme{CambridgeUS}
%\usetheme{Copenhagen} -
%\usetheme{Darmstadt}
%\usetheme{Dresden}
%\usetheme{Frankfurt}
%\usetheme{Goettingen}
%\usetheme{Hannover}
%\usetheme{Ilmenau}
%\usetheme{JuanLesPins}
%\usetheme{Luebeck}
%\usetheme{Madrid}
%\usetheme{Malmoe}
%\usetheme{Marburg}
%\usetheme{Montpellier}
%\usetheme{PaloAlto}
%\usetheme{Pittsburgh}
%\usetheme{Rochester} -
%\usetheme{Singapore}
%\usetheme{Szeged}
%\usetheme{Warsaw}

% As well as themes, the Beamer class has a number of color themes
% for any slide theme. Uncomment each of these in turn to see how it
% changes the colors of your current slide theme.

%\usecolortheme{albatross}
%\usecolortheme{beaver}
%\usecolortheme{beetle}
%\usecolortheme{crane}
%\usecolortheme{dolphin}
%\usecolortheme{dove}
%\usecolortheme{fly}
%\usecolortheme{lily}
%\usecolortheme{orchid}
%\usecolortheme{rose}
%\usecolortheme{seagull}
%\usecolortheme{seahorse}
%\usecolortheme{whale}
%\usecolortheme{wolverine}

%\setbeamertemplate{footline} % To remove the footer line in all slides uncomment this line
%\setbeamertemplate{footline}[frame number] % To replace the footer line in all slides with a simple slide count uncomment this line
\setbeamertemplate{theorems}[numbered]
\setbeamertemplate{navigation symbols}{} % To remove the navigation symbols from the bottom of all slides uncomment this line
}
\setbeamertemplate{caption}{\raggedright\insertcaption\par}
  \setbeamertemplate{enumerate items}[default]
  %\setbeamertemplate{page number in head/foot}{\insertframenumber}
\usepackage{graphicx} % Allows including images
\usepackage{booktabs} % Allows the use of \toprule, \midrule and \bottomrule in tables
%\usepackage {tikz}
\newtheorem*{theorem*}{Theorem}
\newtheorem*{lemma*}{Lemma}
\newtheorem*{proposition*}{Proposition}
\newtheorem*{corollary*}{Corollary}
\newtheorem*{definition*}{Definition}
\DeclareMathOperator*{\argmin}{arg\,min}
\newtheorem*{assumption}{Assumption}
\usetikzlibrary {positioning}
\renewcommand{\arraystretch}{1.5}
\newcommand\hideit[1]{%
  \only<0| handout:1>{\mbox{}}%
  \invisible<0| handout:1>{#1}}
\usepackage[default]{lato}

\setbeamercolor{block body alerted}{bg=alerted text.fg!10}
\setbeamercolor{block title alerted}{bg=alerted text.fg!20}
\setbeamercolor{block body}{bg=structure!10}
\setbeamercolor{block title}{bg=structure!20}
\setbeamercolor{block body example}{bg=green!10}
\setbeamercolor{block title example}{bg=green!20}


\makeatletter
\let\save@measuring@true\measuring@true
\def\measuring@true{%
  \save@measuring@true
  \def\beamer@sortzero##1{\beamer@ifnextcharospec{\beamer@sortzeroread{##1}}{}}%
  \def\beamer@sortzeroread##1<##2>{}%
  \def\beamer@finalnospec{}%
}
\makeatother
%\usepackage {xcolor}

%----------------------------------------------------------------------------------------
%	TITLE PAGE
%----------------------------------------------------------------------------------------

\title[diss]{Lecture 7: Relative Performance Evaluation} % The short title appears at the bottom of every slide, the full title is only on the title page
\author{Compensation in Organizations} % Your name
\institute[shortinst]{Jacob Kohlhepp}
\date{\today} % Date, can be changed to a custom date

\begin{document}

\begin{frame}
\titlepage % Print the title page as the first slide

\end{frame}

\begin{frame}
\centering
    \huge Discussion: Gibbons and Murphy (1990)
\end{frame}






\begin{frame}{Relative Performance Evaluation is Not Teamwork}

\begin{wideitemize}
    \item Sometimes multiple people's effort goes into the final product
    \item When we only observe total output ($y=e_1+e_2$) and we cannot tell how much each person contributed
    \item We call this teamwork and study it after the midterm but not now.
    \item We care about when we observe output for each but with uncertainty
    \item For example: $y_1=e_1 + \epsilon_1$, $y_2=e_2 + \epsilon_2$
    \item Question: what is the point of grouping the workers at all?
\end{wideitemize}
    
\end{frame}


\begin{frame}{Relative Performance Evaluation: A Model}

\begin{wideitemize}
    \item Suppose there are two workers labeled 1 and 2 with the same cost of effort $c(e_i)$.
    \item Output for each $y_1=e_1 + \epsilon_1$, $y_2=e_2 + \epsilon_2$
    \item The noise terms are distributed:
    \begin{wideitemize}
        \item $\epsilon_1= v_s + v_1$
        \item $\epsilon_2= v_s + v_2$
        \item where $v_s \sim N(0,\sigma^2_{s})$, $v_1 \sim N(0,\sigma^2_{1})$ and $v_2 \sim N(0,\sigma^2_{2})$\footnote{Technical note: All are also jointly independent.}
    \end{wideitemize}
    \item Let's focus just on worker 1.
    \item The firm can offer linear wages:
    \begin{wideitemize}
        \item $w(y_1, y_{2}) = \alpha + \beta (y_1 - \gamma y_2)$
    \end{wideitemize}
\end{wideitemize}
    
\end{frame}

\begin{frame}{Interpreting the Model}

\[\epsilon_1= v_s + v_1 \qquad \epsilon_2= v_s + v_2\]

\huge How can we interpret $v_s, v_1, v_2$ when workers 1 and 2 are at the same company?


    
\end{frame}

\begin{frame}{Interpreting the Model}

\[\epsilon_1= v_s + v_1 \qquad \epsilon_2= v_s + v_2\]

\huge How can we interpret $v_s, v_1, v_2$ when workers 1 and 2 are at different companies?
    
\end{frame}

\begin{frame}{Interpreting the Model}

\[\epsilon_1= v_s + v_1 \qquad \epsilon_2= v_s + v_2\]

\huge How can we interpret the variances of $v_s, v_1, v_2$? i.e. what does it mean if $\sigma_s^2>\sigma_1^2$?
\end{frame}

\begin{frame}{Some Observations}
\[w(y_1, y_{2}) = \alpha + \beta (y_1 - \gamma y_2)\]
\begin{wideitemize}
    \item[1.] If $\gamma=0$ we have no relative performance evaluation (back to base model)
    \begin{wideitemize}
        \item Since $\gamma$ is a choice, access to relative performance evaluations must weakly improve profit!
    \end{wideitemize}
    \pause
    \item[2.] $\gamma$ does not influence effort.
    \begin{wideitemize}
        \item Question: Why?
        % Because the worker 1 cannot influence Y_2
    \end{wideitemize}
    \pause
    \item[3.] $Y_2$ contains information about $Y_1$
    \begin{wideitemize}
        \item Question: Why?
        % Because of $v_s$
    \end{wideitemize}
\end{wideitemize}
    
\end{frame}

\begin{frame}{Solving the Model}
    \huge See the board!
\end{frame}

\begin{frame}{Solving the Model}
      \begin{theorem}
       Under relative performance evaluation, worker 1's wage is $w(y_1, y_2)=\alpha_{rel} + \beta_{rel} (y_1 - \gamma_{rel} y_2)$ where
      \[\gamma_{rel}=\frac{\sigma^2_s}{\sigma^2_s + \sigma^2_2}\]
 \[  \beta_{rel}=\frac{1}{1+r  ( \sigma_1^2+ (1-\gamma_{rel})^2\sigma_s^2 +\gamma_{rel}^2 \sigma_2^2 )c''(e_1)}\]
 \[\alpha_{rel} = \bar u - \beta_{rel}(e_1-\gamma_{rel} e_2) + \frac{r\beta_{rel}^2[\sigma_1^2 + (1-\gamma_{rel})^2\sigma^2_s + \gamma_{rel}^2\sigma^2_2] }{2}   \]
    \end{theorem}

    \footnotesize A nice comprehension check is to figure out how these would change for worker 2.
\end{frame}



\section{The Informativeness Principle}

\begin{frame}{Thinking More Generally}

\begin{wideitemize}
    \item We thought of $Y_2$ as the output of a coworker or comparable worker.
    \item But we just showed that it does not matter for incentives.
    \item The firm only uses it to reduce the noise in performance evaluations.
    \item What if we think of $Y_2$ as just some extra information?
    \begin{wideitemize}
        \item Question: What are some examples?
        %weather, stock market performance, unemployment rate.
    \end{wideitemize}
\end{wideitemize}
\end{frame}

\begin{frame}{Working It Out}

\huge See the board!
    
\end{frame}

\begin{frame}{The Informativeness Principle}

    \begin{theorem}
        The firm should use additional information $Y_2$ to set pay for worker 1 whenever the information is informative about worker 1's output: $\sigma_s^2>0$.
    \end{theorem}

    \begin{wideitemize}
    \item The firm uses $Y_2$ to purge $Y_1$ of noise/luck/etc.
    \item This reduces the effective variance the worker faces for each level of bonus $\beta$.
    \item This relieves the risk-incentive trade-off.
    \item Therefore it improves profit and total surplus!
\end{wideitemize}


\end{frame}



\end{document}







