%%%%%%%%%%%%%%%%%%%%%%%%%%%%%%%%%%%%%%%%%
% Beamer Presentation
% LaTeX Template
% Version 1.0 (10/11/12)
%
% This template has been downloaded from:
% http://www.LaTeXTemplates.com
%
% License:
% CC BY-NC-SA 3.0 (http://creativecommons.org/licenses/by-nc-sa/3.0/)
%
%%%%%%%%%%%%%%%%%%%%%%%%%%%%%%%%%%%%%%%%%

%----------------------------------------------------------------------------------------
%	PACKAGES AND THEMES
%----------------------------------------------------------------------------------------

\documentclass[aspectratio=169,usenames,dvipsnames]{beamer}

\usepackage[utf8]{inputenc}
\usepackage{booktabs}
\usepackage{tabularx}
\usepackage[authordate,bibencoding=auto,strict,backend=biber,natbib]{biblatex-chicago}
\addbibresource{bib.bib}
\usepackage{graphicx}
% \hypersetup{
%     colorlinks,
%     %citecolor=black,
%     linkcolor=black
% }
\usepackage{array}
\usepackage{caption}
\usepackage{threeparttable}
\usepackage{epigraph} 
\usepackage{lscape}
\usepackage{adjustbox}
\newcommand*{\Scale}[2][4]{\scalebox{#1}{\ensuremath{#2}}}%
\usepackage{import}
\newenvironment{wideitemize}{\itemize\addtolength{\itemsep}{10pt}}{\enditemize}
\usepackage{amsmath}
\usepackage{csvsimple}
\usepackage{siunitx}
\usepackage{filecontents}
\usepackage{rotating}
\usepackage{multirow}
\usepackage{amsmath}
\usepackage{subcaption}
\usepackage{appendixnumberbeamer}
\usepackage{float}
\usepackage{amsmath}
\usepackage{csvsimple}
\usepackage{hyperref}
\newtheorem{proposition}{Proposition}
\usepackage{xcolor}
\def\boxit#1#2{%
    \smash{\color{red}\fboxrule=1pt\relax\fboxsep=2pt\relax%
    \llap{\rlap{\fbox{\phantom{\rule{#1}{#2}}}}~}}\ignorespaces
}
\newenvironment{variableblock}[3]{%
  \setbeamercolor{block body}{#2}
  \setbeamercolor{block title}{#3}
  \begin{block}{#1}}{\end{block}}
\usepackage{appendixnumberbeamer}
\usepackage{tikz,pgfplots}
\usepackage{tkz-fct}
\usepackage{amsthm}
\pgfplotsset{compat=1.10}
\usepgfplotslibrary{fillbetween}
\mode<presentation> {
\AtBeginSection[]
{
    \begin{frame}
        \frametitle{Table of Contents}
        \tableofcontents[currentsection]
    \end{frame}
}
% The Beamer class comes with a number of default slide themes
% which change the colors and layouts of slides. Below this is a list
% of all the themes, uncomment each in turn to see what they look like.

\usetheme{default}
%\usetheme{AnnArbor}
%\usetheme{Antibes} -
%\usetheme{Bergen}
%\usetheme{Berkeley}
%\usetheme{Berlin}
%\usetheme{Boadilla}
%\usetheme{CambridgeUS}
%\usetheme{Copenhagen} -
%\usetheme{Darmstadt}
%\usetheme{Dresden}
%\usetheme{Frankfurt}
%\usetheme{Goettingen}
%\usetheme{Hannover}
%\usetheme{Ilmenau}
%\usetheme{JuanLesPins}
%\usetheme{Luebeck}
%\usetheme{Madrid}
%\usetheme{Malmoe}
%\usetheme{Marburg}
%\usetheme{Montpellier}
%\usetheme{PaloAlto}
%\usetheme{Pittsburgh}
%\usetheme{Rochester} -
%\usetheme{Singapore}
%\usetheme{Szeged}
%\usetheme{Warsaw}

% As well as themes, the Beamer class has a number of color themes
% for any slide theme. Uncomment each of these in turn to see how it
% changes the colors of your current slide theme.

%\usecolortheme{albatross}
%\usecolortheme{beaver}
%\usecolortheme{beetle}
%\usecolortheme{crane}
%\usecolortheme{dolphin}
%\usecolortheme{dove}
%\usecolortheme{fly}
%\usecolortheme{lily}
%\usecolortheme{orchid}
%\usecolortheme{rose}
%\usecolortheme{seagull}
%\usecolortheme{seahorse}
%\usecolortheme{whale}
%\usecolortheme{wolverine}

%\setbeamertemplate{footline} % To remove the footer line in all slides uncomment this line
%\setbeamertemplate{footline}[frame number] % To replace the footer line in all slides with a simple slide count uncomment this line
\setbeamertemplate{theorems}[numbered]
\setbeamertemplate{navigation symbols}{} % To remove the navigation symbols from the bottom of all slides uncomment this line
}
\setbeamertemplate{caption}{\raggedright\insertcaption\par}
  \setbeamertemplate{enumerate items}[default]
  %\setbeamertemplate{page number in head/foot}{\insertframenumber}
\usepackage{graphicx} % Allows including images
\usepackage{booktabs} % Allows the use of \toprule, \midrule and \bottomrule in tables
%\usepackage {tikz}
\newtheorem*{theorem*}{Theorem}
\newtheorem*{lemma*}{Lemma}
\newtheorem*{proposition*}{Proposition}
\newtheorem*{corollary*}{Corollary}
\newtheorem*{definition*}{Definition}
\DeclareMathOperator*{\argmin}{arg\,min}
\newtheorem*{assumption}{Assumption}
\usetikzlibrary {positioning}
\renewcommand{\arraystretch}{1.5}
\newcommand\hideit[1]{%
  \only<0| handout:1>{\mbox{}}%
  \invisible<0| handout:1>{#1}}
\usepackage[default]{lato}

\setbeamercolor{block body alerted}{bg=alerted text.fg!10}
\setbeamercolor{block title alerted}{bg=alerted text.fg!20}
\setbeamercolor{block body}{bg=structure!10}
\setbeamercolor{block title}{bg=structure!20}
\setbeamercolor{block body example}{bg=green!10}
\setbeamercolor{block title example}{bg=green!20}


\makeatletter
\let\save@measuring@true\measuring@true
\def\measuring@true{%
  \save@measuring@true
  \def\beamer@sortzero##1{\beamer@ifnextcharospec{\beamer@sortzeroread{##1}}{}}%
  \def\beamer@sortzeroread##1<##2>{}%
  \def\beamer@finalnospec{}%
}
\makeatother
%\usepackage {xcolor}

%----------------------------------------------------------------------------------------
%	TITLE PAGE
%----------------------------------------------------------------------------------------

\title[diss]{Lecture 13: Career Concerns} % The short title appears at the bottom of every slide, the full title is only on the title page
\author{Compensation in Organizations} % Your name
\institute[shortinst]{Jacob Kohlhepp}
\date{\today} % Date, can be changed to a custom date

\begin{document}

\begin{frame}
\titlepage % Print the title page as the first slide

\end{frame}

\begin{frame}{Roadmap}
\begin{wideitemize}
    \item In the first half of the class we discussed explicit performance pay.
    \item Last lecture we discussed one alternative: relational contracts.
    \item I may work hard because I don't want to lose a good job.
    \item This lecture we consider another alternative: career concerns.
    \item I work hard at my current job to get a better future job at another company.
\end{wideitemize}
    
\end{frame}
\begin{frame}
\centering
    \huge Discussion: Fee and Hadlock (2003)

\end{frame}

\begin{frame}{Model}
    \begin{wideitemize}
        \item There are two firms and one worker.
        \item The worker has a skill level $a$ that no one knows.
        \item However, everyone knows that skills are distributed uniformly between $[0,A]$. That is, $a\sim U[0,A]$
        \item The worker exerts unobserved, costly effort: $c(e)=e^2/2$
        \item Revenue is equal to effort plus skill: $y=e+a$
        \item The worker is hired and exerts effort in two periods.
        \item The worker is hired in each period by the firm that posts the highest wage, and if there is a tie they randomly pick a firm (Bertrand style)
        \item All outside options are 0.
    \end{wideitemize}
\end{frame}

\begin{frame}{Thinking About First-Best}

\begin{wideitemize}
    \item A helpful fact: the mean of a uniform random variable is just the average of the lower and the upper bound, so: $E[a]=(0+A)/2=A/2$\pause
    \item In a single period, the first-best level of effort solves:

    \[\max_{e} E[a+e -c(e)] =  \max_{e} A/2+e-e^2/2  \]
    \pause
    \item Taking the FOC wrt $e$:
    \[e_{FB}= 1\]
    
\end{wideitemize}
    
\end{frame}

\begin{frame}{Diagram of the Model}
\centering
    \huge See the board!

\end{frame}

\begin{frame}{Solving the Model}
\centering
    \huge See the board!

\end{frame}

\begin{frame}{Solution}

\begin{theorem}
    If uncertainty about skill is large enough ($A>>0$), career concerns motivate the worker to provide first-best effort in the first period, $e_1^*=1$.
\end{theorem}
    \begin{wideitemize}
        \item The worker works hard to prove his/herself in period 1
        \item The worker reaps the reward for this in period 2
        \item The strength of career concerns depends on how uncertain the market is about a worker's skill.
        \item This is captured by $A$.
    \end{wideitemize}
\end{frame}

\begin{frame}{Wages Over Time}

\begin{wideitemize}
    \item Skill ($a$) is initially unknown and everyone is paid the same wage: $A/2+1$
    \item Skill ($a$) is revealed when revenue realizes.
    \item Then each worker is paid their skill.
    \item This implies two things:
    \begin{wideitemize}
        \item Wages of two workers become more dispersed over time
        \item Wages may go up or down, with most going down.
    \end{wideitemize}
\end{wideitemize}
    
\end{frame}

\begin{frame}{Effort Over Time}

\begin{wideitemize}
    \item Effort in the first period is high (the first-best level)
    \item Intuition: prove yourself when young, take it easy while old.
\end{wideitemize}
    
\end{frame}



\begin{frame}{Crucial Ingredients for Career Concerns}

\begin{wideitemize}
    \item There must be competition for the worker.
    \item Revenue in period 1 must be observed by everyone.
    \item Discussion: in what occupations is this true/not true?
    \item These are crucial because they allow the worker to internalize the benefits of effort/skill through higher wages in period 2.
\end{wideitemize}
    
\end{frame}

\end{document}







