\documentclass{article}
\usepackage{graphicx} % Required for inserting images
\usepackage{amsmath} 
\newtheorem{lemma}{lemma}
\title{Board Work for Lecture 16}
\author{Jacob Kohlhepp}
\date{\today}

\begin{document}

\maketitle


\section{Free Riding Problem}

To get a feel for the problem associated with teamwork, consider the intuitive wage scheme where we all split output evenly. That is: $w_i(y)=y/N=(e_1+e_2+...e_N)/N$. The marginal benefit to the worker of additional effort is:
\[\frac{\partial w_i(y)}{\partial e_i} = \frac{\partial }{\partial e_i} \frac{e_1+e_2+...e_i+...+e_N}{N} = \frac{1}{N} \]
The worker equates this to the marginal cost to find the effort they provide:
\[c_i'(e_i)=\frac{1}{N} \]
In general this is too little effort. To see this consider when $c_i(e_i)=e_i^2/2$. In this case individual output less effort cost looks like our effort-based pay model ($e_i-e_i^2/2$). And the first-best effort is $e_i^*=1$. However, what actually happens is given by: $c_i'(e_i)=1/N \implies e_i=1/N$. Each worker provides a fraction of the first-best effort, and the effort provided gets lower as the group gets bigger.

Effectively people are free riding: they know that the benefits of their effort are split across $N-1$ other people but they bare the full costs of their effort directly. So what do they do? They provide too little effort. Since everyone does this, we collectively get a total effort of 1 instead of the first-best total effort $\sum_{i=1}^N e^*_i = N$.

\section{First-Best}

The first-best benchmark is the amount of effort from each worker that maximizes total surplus. Total surplus is all the real benefits less all the real costs (wages are transfers from the firm to the worker so they wash out). The real benefits are output and the real costs are each worker's effort cost, so the first-best effort solves:

\[\max_{e} y(e) -\sum_{i=1}^N c_i(e_i)\]
where remember that $e$ contains each effort of each worker. Remember that $y(e)=e_1+e_2+...+e_N$. Take the FOC for worker $i$'s effort:
\[\frac{\partial y(e)}{\partial e_i} -c_i'(e_i)=0 \leftrightarrow \frac{\partial e_1+e_2+...+e_N}{\partial e_i}-c_i'(e_i)=0\leftrightarrow 1 -c_i'(e_i)=0\]
Thus the first-best effort is defined as the $e_i^*$ that solve:
\[ c_i'(e_i^*)=1\]
for all $i$. This is one equation for each worker, so we can find $e_i^*$ for everyone. We call the list of all of the first-best effort $e^*$.

\section{Partnerships}

% \begin{definition}
%     A partnership is a wage scheme where $w_i(y(e))\geq 0$ and:
%     \[\sum_{i=1}^N w_i(y(e))=y(e)\]
%     for every output $y(e)$.
% \end{definition}

We now want to know if there exists a wage scheme that is a partnership that also achieves first-best effort $e^*$. Under a wage scheme $w_i(e)$ worker $i$ chooses effort based on the FOC:
\[\frac{\partial w_i(y(e))}{\partial e_i}-c_i'(e_i)=0\]
We can expand the first term using the chain rule:
\[\frac{\partial w_i(y(e))}{\partial e_i}=w_i'(y) \frac{\partial y(e)}{\partial e_i} \]
Remember that output is just the sum of everyone's effort so:
\[\frac{\partial y(e)}{\partial e_i} = \frac{\partial e_1..+e_i+...e_N}{\partial e_i}= 1 \]
Therefore:
\[\frac{\partial w_i(y)}{\partial e_i}=w_i'(y) \frac{\partial y(e)}{\partial e_i} = w_i'(y(e))\]
Putting everything together, we have that worker $i$'s effort solves:
\[ w_i'(y(e)) =c_i'(e_i) \]
If we want to get first-best effort, we need $e^*$ to solve this equation for everyone. Therefore each of the $N$ equations must hold if we plug in $e^*$:
\[ w_i'(y(e^*)) =c_i'(e_i^*) \]
But we also have that $c_i(e^*_i)=1$, so:
\[ w_i'(y(e^*)) =c_i'(e_i^*)=1 \]
So we must be paying each worker a dollar for their marginal effort. is this possible with a partnership? To find out, remember that a partnership is a wage scheme where the what is paid out to everyone is equal to the output produced:
\[\sum_{i=1}^N w_i(y(e))=y(e)\]
Take the derivative of both sides of this equation with respect to total output/effort $y(e)$ (not individual effort):
\[ \frac{\partial } {\partial y(e)} \bigg ( \sum_{i=1}^N w_i(y(e)) \bigg )=\frac{\partial } {\partial y(e)} \bigg ( y(e) \bigg ) \]
\[  \sum_{i=1}^N \frac{\partial w_i(y(e))} {\partial y(e)}  =1 \]
\[  \sum_{i=1}^N w_i'(y(e))  =1 \]
This means that the marginal dollar paid to all workers must sum up to 1. This is the contradiction: to get the first-best effort everyone must get the marginal dollar produced: $w_i'(y(e^*))=1$. But if we plug this in, we get a contradiction:
\[  \sum_{i=1}^N w_i'(y(e^*)) = \sum_{i=1}^N 1=N  \neq 1 \]
This is because under a partnership we must split the marginal dollar! Therefore a partnership cannot implement the first-best level of effort. This was a proof by contradiction: we assumed the wage scheme accomplished the first-best effort and it was a partnership. We then showed that this is impossible.


\section{Group Bonus}

We will start by guessing that $\bar y = y(e^*)$ and $b_i\geq c_i(e_i^*)$. Why might these properties be necessary to get the first-best? We will proceed by showing that under this scheme everyone chooses $e_i^*$ in equilibrium. We will end by asking if these properties are actually feasible. That is, is there enough money to pay bonuses that satisfy these properties?

The first thing to remember is that in equilibrium, the worker assumes everyone else is contributing first-best effort. The question is then given this, does the worker want to follow suit and contribute first-best effort? or do they want to free ride or work too hard? Notice that if everyone, including the worker contributes exactly the first-best level of effort each person makes a positive amount of money because everyone gets the bonus, and the bonus is greater than the effort cost at $e_i^*$ (because of the property $b_i> c_i(e_i^*)$).


If the worker overworks ($e_i>e_i^*$) the worker earns no additional money (the bonus is fixed at $b_i$) but pays an additional effort cost. This is clearly worse than just contributing first-best effort $e_i^*$. If the worker free rides and contributes too little effort ($e_i<e_i^*$) then $y(e)<y(e^*)$ and everyone including the worker we are considering does not get the bonus. The worker's payoff goes from something positive to something that is at best 0 (if the worker chooses $e_i=0$) and potentially negative (if $e_i^*>e_i>0$). Combining these arguments we have that each worker wants to contribute exactly first-best effort under a group bonus with the properties we guessed.

Now, is our dream bonus feasible? Is there money to pay for it? Yes! We first consider the case when $e_i=e_i^*$. Remember that first-best effort is defined as the level of effort which solves:
\[\max_e y(e) - \sum_i c_i(e_i)\]
We assumed that if everyone contributes 0 effort, this is zero cost but also zero output. Since $e^*$ is first-best it achieves more total surplus than 0 effort from everyone. This means that:
\[y(e^*)- \sum_i c_i(e_i)>0\]
In words, output is greater than the total effort cost. Thus, there is enough money to pay each worker for their effort cost. There are many bonuses that can work: they just need to cover each worker's cost of effort  and add up to the total output.

For example, if there are two workers which both contribute 2 to output, and the cost of effort for both is 1, any bonus that adds up to 4 and gives each person at least 1 works. For example, $(2.9, 1.1)$ and $(2,2)$ both work.


We also must consider when $y(e)<y(e^*)$. In this case bonuses are 0, so it is trivially true that we can afford to pay them out. Taken together we have shown that a specific type of group bonus is feasible and implements first-best effort.





\section{Money Burning}

Consider the group bonus. When everyone performs first-best effort, all output is paid out in bonuses because $\sum_i b_i = y(e^*)$. However, whenever too little effort is provided by anyone, nothing is paid out. Even if only one person slacks off, or even if everyone only slightly slacks off, nothing gets paid out. Yet, except for some particular output functions, something will be produced.

Implementing the first-best using group bonuses thus requires credibly destroying or getting rid of output when people slack. It is pretty hard to believe that a partnership, where there is no third party boss or shareholder, would destroy output rather than paying it out. This provides a reason for the existence of a monitor or a firm: someone to enforce punishments/incentives.


\end{document}