\documentclass{article}
\usepackage{graphicx} % Required for inserting images
\usepackage{amsmath} 
\newtheorem{lemma}{lemma}
\usepackage{geometry}
\geometry{margin=1in}
\title{Final Exam: Econ 490 Compensation in Organizations}
\author{Instructor: Jacob Kohlhepp}
\date{Section 3: May 3, 2024 at 12pm}

\begin{document}

\maketitle

Name: \underline{\qquad \qquad \qquad \qquad\qquad \qquad \qquad\qquad \qquad\qquad \qquad} \quad \quad \quad  PID:  \underline{\qquad \qquad \qquad \qquad\qquad \qquad\quad \quad }
\\

You have 3 hours to complete this exam. Please stop writing when told to do so. Write all answers in the space provided, and show work where possible. If you run out of room, make a note and use the additional pages attached at the end of the exam. This is a closed book exam. The only materials you may use are a pen and paper. By taking this exam, you agree to follow the UNC Chapel Hill honor code, in particular the standards of academic integrity. All academic dishonesty will be reported to the Office of Student Conduct and the Student Attorney General. Each reading question is worth 5 points. Each model question is worth 3 points. There are a total of 100 points.

\section{Readings}
Answer these questions in 3 sentences or less.
\begin{enumerate}

    \item Describe the policy from Alexander (2020) ``How do Doctors Respond to Incentives?" How did doctors respond?

\vspace{6cm}  

    \item In Gong, Zhang and Zhou (2023) ``Retention Effects of Employee Stock Options," what was the main finding?

 \vspace{6cm}

    \item What does Macleod and Urquiola (2021) ``Why Does the United States Have the Best Research Universities?" use to measure the rise of US universities?

 \vspace{6cm}    

    \item Describe the method used in ``Team Incentives and Performance: Evidence from a Retail Chain" by Friebel, Heniz Krueger and Zubanov (2017).

\vspace{6cm}    


    \item In Blair and Chung (2022) ``Job Market Signaling through Occupational Licensing," in what types of states does occupational licensing reduce the racial wage gap the most? Use one sentence.

\vspace{6cm}  

  
  
\end{enumerate}

\newpage

\section{Multitasking}


\subsection*{Setup}
\begin{itemize}
    \item Output is $y=a e_1+b e_2$, where $a>b, a>0$
    \item Cost of effort is:
       \[c(e_1, e_2) = \begin{cases}
            0 & \text{ if }  e_1+e_2 \leq 2 \bar e \\
            (e_1+e_2-2\bar e)^2/2 & \text{ if } e_1+e_2 > 2 \bar e 
        \end{cases}\]
      \item We assume that without incentives the worker supplies all 0 cost effort and splits effort evenly:
      \[e_1=e_2=\bar e\]
    \item Only task 1 effort is measured: $m=e_1$
    \item The firm can only pay based on task 1: $w(m)=\alpha + \beta m =\alpha + \beta e_1$
    \item The firm's outside option is 0, the worker's is $\bar u$
\end{itemize}

\subsection*{Questions}

\begin{enumerate}

    \item Solve for the first-best $e_1,e_2$. For this problem only assume that $a>b$.

\vspace{7cm}

        
 \item From now on we are solving for equilibrium, meaning the firm cannot choose effort directly but just chooses a compensation scheme. Setup the worker's effort choice problem.

\vspace{6cm}


 \item Solve for worker's choice of effort assuming for now until told otherwise that $\beta>0$.


\vspace{7cm}


 \item Write down the inequality that determines whether the worker takes the job. Argue that it must be an equality.

\vspace{6cm}



 \item Setup the firm's profit maximization problem. Substitute past work in so that it is only a function of $\beta$.


\vspace{7cm}

 \item Solve for the profit-maximizing $\beta, e_1,e_2$.


\vspace{7cm}

 \item Now, solve for $e_1, e_2$ when $\beta=0$. You may use the same steps we just did or do it your own way.

\vspace{7cm}

\item Assume that $a=2, b=1,\bar e=2$. Using the work you have already done, should the firm set $\beta=0$ or $\beta>0$? Find $\beta, e_1, e_2$.

  \vspace{6cm}
  

\end{enumerate}


\newpage


\section{Relational Contracts}
\subsection*{Setup}
\begin{itemize}
    \item A firm and a worker both have discount rate $\delta$ and interact for many periods ($t=1,...,\infty$)
    \item At each period $t$ the following occur:
    \begin{itemize}
        \item First the firm offers a flat wage $w_t$
        \item Second the worker chooses high (H) or low (L) effort $e_t$
    \end{itemize}
    \item High effort has cost $c>0$, low effort has cost 0.
    \item High effort yields revenue $v \geq 0$, low effort yields revenue 0.
    \item Firm outside option is 0, worker outside option is $\bar u > 0$.
    \item Assume the firm wants to motivate high effort.
\end{itemize}

\subsection*{Questions}

\begin{enumerate}
    \item Guess an equilibrium strategy for the firm in words. Guess an equilibrium strategy for the worker in words. (Hint: guess the same strategy as in class)

  \vspace{6cm}  
  

    \item Call the high wage $w_H$ and the low wage $w_L$. Assume the strategy you guessed is being played. What value of $w_L$ will the firm choose? What is the worker's payoff in any period where the firm posts a wage of $w_L$? Justify your answers.

  \vspace{6cm} 

  



\item[***] For the next two subquestions, suppose the worker is considering one shot deviations when they slacked in the past (so are now being offered $w_L$ forever).


    \item  Consider two deviations: taking the job and exerting low effort and taking the job and exerting high effort. Argue either mathematically or verbally that one is a more attractive deviation.


  \vspace{6cm} 

  
\item Write down one inequality that captures when the worker has no incentive to make the more attractive deviation in the last subquestion. Make sure to simplify. When does it hold?


  \vspace{7cm} 
  


\item[***] For the next two subquestions, suppose the worker is considering one shot deviations when they never slacked in the past (so are currently being offered $w_H$ each period).

    
    \item Consider two deviations: taking the job and exerting low effort and not taking the job. Argue either mathematically or verbally that one is a more attractive deviation.


  \vspace{6cm} 
  

\item Write down one inequality that captures when the worker has no incentive to make the more attractive deviation in the last subquestion. Make sure to simplify.


  \vspace{8cm}
  



\item Using the inequality you just derived, find the firm's optimal choice of $w_H$ and profit.

\vspace{6cm}

    \item Suppose initially $\delta = 0.5, v=4, \bar u = 1, c=1$. Then, conditions change and $\delta = 0.5, v=7, \bar u = 4, c=1$. Explain what happens to profit and provide an economic example.

\vspace{7cm}


    
\end{enumerate}

\newpage

\section{Career Concerns}

\subsection*{Setup}
\begin{itemize}
        \item There are two firms and one worker.
        \item The worker has a skill level $a$ that no one knows.
        \item However, everyone knows that skills are distributed uniformly between $[0,A]$. That is, $a\sim U[0,A]$
        \item The worker exerts unobserved, costly effort: $c(e)=e^2/2$
        \item Revenue is equal to effort plus skill: $y=e+a$
        \item The worker is hired and exerts effort in two periods.
        \item The worker is hired in each period by the firm that posts the highest wage, and if there is a tie they randomly pick a firm (Bertrand style)
        \item All outside options are 0.
\end{itemize}

\subsection*{Questions}
\begin{enumerate}
    \item What is the first-best level of effort for a single period? That is, the $e_{FB}$ that maximizes output less the cost of effort?

\vspace{7cm}

    \item How much effort will the worker exert in period 2? Justify your answer.


\vspace{6cm}


    \item Denote the effort the firm believes the worker exerts in period 1 $\tilde e_1$. How can the firm recover the worker's skill using $\tilde e_1$ and output $y_1$?

\vspace{6cm}


    \item What output levels $y_1$ will the firm never observe if the worker does the effort that is expected ($\tilde e_1$)?

\vspace{7cm}


    \item Suppose the firms believe skill is $a$ in period 2. What wage will they bid in period 2? Justify your answer.


\vspace{7cm}



    \item Solve for the worker's optimal effort.


\vspace{6cm}


    \item What wage do the firms in period 1 bid? Justify your answer.


\vspace{5cm}


\item How does this effort compare to the effort in sub question 1? Why is the worker working hard?

\vspace{5cm}

    \item Suppose $A=20$. If a worker has skill $12$, how does their wage change from period 1 to period 2?


\vspace{5cm}

    

    
\end{enumerate}

\newpage

\,

\newpage

\,

\newpage

\,

\newpage

\,


\end{document}