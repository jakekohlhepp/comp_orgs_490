%%%%%%%%%%%%%%%%%%%%%%%%%%%%%%%%%%%%%%%%%
% Beamer Presentation
% LaTeX Template
% Version 1.0 (10/11/12)
%
% This template has been downloaded from:
% http://www.LaTeXTemplates.com
%
% License:
% CC BY-NC-SA 3.0 (http://creativecommons.org/licenses/by-nc-sa/3.0/)
%
%%%%%%%%%%%%%%%%%%%%%%%%%%%%%%%%%%%%%%%%%

%----------------------------------------------------------------------------------------
%	PACKAGES AND THEMES
%----------------------------------------------------------------------------------------

\documentclass[aspectratio=169,usenames,dvipsnames]{beamer}

\usepackage[utf8]{inputenc}
\usepackage{booktabs}
\usepackage{tabularx}
\usepackage[authordate,bibencoding=auto,strict,backend=biber,natbib]{biblatex-chicago}
\addbibresource{bib.bib}
\usepackage{graphicx}
% \hypersetup{
%     colorlinks,
%     %citecolor=black,
%     linkcolor=black
% }
\usepackage{array}
\usepackage{caption}
\usepackage{threeparttable}
\usepackage{epigraph} 
\usepackage{lscape}
\usepackage{adjustbox}
\newcommand*{\Scale}[2][4]{\scalebox{#1}{\ensuremath{#2}}}%
\usepackage{import}
\newenvironment{wideitemize}{\itemize\addtolength{\itemsep}{10pt}}{\enditemize}
\usepackage{amsmath}
\usepackage{csvsimple}
\usepackage{siunitx}
\usepackage{filecontents}
\usepackage{rotating}
\usepackage{multirow}
\usepackage{amsmath}
\usepackage{subcaption}
\usepackage{appendixnumberbeamer}
\usepackage{float}
\usepackage{amsmath}
\usepackage{csvsimple}
\usepackage{hyperref}
\newtheorem{proposition}{Proposition}
\usepackage{xcolor}
\def\boxit#1#2{%
    \smash{\color{red}\fboxrule=1pt\relax\fboxsep=2pt\relax%
    \llap{\rlap{\fbox{\phantom{\rule{#1}{#2}}}}~}}\ignorespaces
}
\newenvironment{variableblock}[3]{%
  \setbeamercolor{block body}{#2}
  \setbeamercolor{block title}{#3}
  \begin{block}{#1}}{\end{block}}
\usepackage{appendixnumberbeamer}
\usepackage{tikz,pgfplots}
\usepackage{tkz-fct}
\usepackage{amsthm}
\pgfplotsset{compat=1.10}
\usepgfplotslibrary{fillbetween}
\mode<presentation> {
\AtBeginSection[]
{
    \begin{frame}
        \frametitle{Table of Contents}
        \tableofcontents[currentsection]
    \end{frame}
}
% The Beamer class comes with a number of default slide themes
% which change the colors and layouts of slides. Below this is a list
% of all the themes, uncomment each in turn to see what they look like.

\usetheme{default}
%\usetheme{AnnArbor}
%\usetheme{Antibes} -
%\usetheme{Bergen}
%\usetheme{Berkeley}
%\usetheme{Berlin}
%\usetheme{Boadilla}
%\usetheme{CambridgeUS}
%\usetheme{Copenhagen} -
%\usetheme{Darmstadt}
%\usetheme{Dresden}
%\usetheme{Frankfurt}
%\usetheme{Goettingen}
%\usetheme{Hannover}
%\usetheme{Ilmenau}
%\usetheme{JuanLesPins}
%\usetheme{Luebeck}
%\usetheme{Madrid}
%\usetheme{Malmoe}
%\usetheme{Marburg}
%\usetheme{Montpellier}
%\usetheme{PaloAlto}
%\usetheme{Pittsburgh}
%\usetheme{Rochester} -
%\usetheme{Singapore}
%\usetheme{Szeged}
%\usetheme{Warsaw}

% As well as themes, the Beamer class has a number of color themes
% for any slide theme. Uncomment each of these in turn to see how it
% changes the colors of your current slide theme.

%\usecolortheme{albatross}
%\usecolortheme{beaver}
%\usecolortheme{beetle}
%\usecolortheme{crane}
%\usecolortheme{dolphin}
%\usecolortheme{dove}
%\usecolortheme{fly}
%\usecolortheme{lily}
%\usecolortheme{orchid}
%\usecolortheme{rose}
%\usecolortheme{seagull}
%\usecolortheme{seahorse}
%\usecolortheme{whale}
%\usecolortheme{wolverine}

%\setbeamertemplate{footline} % To remove the footer line in all slides uncomment this line
%\setbeamertemplate{footline}[frame number] % To replace the footer line in all slides with a simple slide count uncomment this line
\setbeamertemplate{theorems}[numbered]
\setbeamertemplate{navigation symbols}{} % To remove the navigation symbols from the bottom of all slides uncomment this line
}
\setbeamertemplate{caption}{\raggedright\insertcaption\par}
  \setbeamertemplate{enumerate items}[default]
  %\setbeamertemplate{page number in head/foot}{\insertframenumber}
\usepackage{graphicx} % Allows including images
\usepackage{booktabs} % Allows the use of \toprule, \midrule and \bottomrule in tables
%\usepackage {tikz}
\newtheorem*{theorem*}{Theorem}
\newtheorem*{lemma*}{Lemma}
\newtheorem*{proposition*}{Proposition}
\newtheorem*{corollary*}{Corollary}
\newtheorem*{definition*}{Definition}
\DeclareMathOperator*{\argmin}{arg\,min}
\newtheorem*{assumption}{Assumption}
\usetikzlibrary {positioning}
\renewcommand{\arraystretch}{1.5}
\newcommand\hideit[1]{%
  \only<0| handout:1>{\mbox{}}%
  \invisible<0| handout:1>{#1}}
\usepackage[default]{lato}

\setbeamercolor{block body alerted}{bg=alerted text.fg!10}
\setbeamercolor{block title alerted}{bg=alerted text.fg!20}
\setbeamercolor{block body}{bg=structure!10}
\setbeamercolor{block title}{bg=structure!20}
\setbeamercolor{block body example}{bg=green!10}
\setbeamercolor{block title example}{bg=green!20}


\makeatletter
\let\save@measuring@true\measuring@true
\def\measuring@true{%
  \save@measuring@true
  \def\beamer@sortzero##1{\beamer@ifnextcharospec{\beamer@sortzeroread{##1}}{}}%
  \def\beamer@sortzeroread##1<##2>{}%
  \def\beamer@finalnospec{}%
}
\makeatother
%\usepackage {xcolor}

%----------------------------------------------------------------------------------------
%	TITLE PAGE
%----------------------------------------------------------------------------------------

\title[diss]{Lecture 12: Relational Contracts} % The short title appears at the bottom of every slide, the full title is only on the title page
\author{Compensation in Organizations} % Your name
\institute[shortinst]{Jacob Kohlhepp}
\date{\today} % Date, can be changed to a custom date

\begin{document}

\begin{frame}
\titlepage % Print the title page as the first slide

\end{frame}


\section{Relational Contracts}


\begin{frame}
\centering
    \huge Discussion: Cheveleir and Ellison (1998)

\end{frame}

\begin{frame}{Aside: Discounting}
\begin{wideitemize}
    \item The discount rate $\delta$ captures how much a dollar tomorrow is worth (to someone) today.
    \item If $\delta=0.9$, a dollar tomorrow is worth 90 cents today.
    \item If $\delta=0.99$, a dollar tomorrow is worth 99 cents today.
    \item Higher $\delta \implies$ I am more patient.
    \item We can also think of this as the probability we meet again tomorrow.
    \item Then the probability we meet again $T$ times (assuming independence) is just $\delta^T$
\end{wideitemize}
    
\end{frame}


\begin{frame}{Aside: Discounting}
\begin{wideitemize}
    \item Suppose I receive a payment (or utility) $u$ for $T$ periods. The present value of this stream of payments is:
    \[\sum_{t=0}^T \delta^t u=u+\delta u +\delta^2 u+...+\delta^Tu\]
    \pause
    \item Suppose $T\to \infty$. Then:
    \begin{align*}
        \sum_{t=0}^\infty \delta^t u&=u+\delta u +\delta^2 u+...\\
        &= u + \delta (u + \delta u + \delta^2 u...)\\
        &=  u + \delta  \sum_{t=0}^\infty \delta^t u\\
        \sum_{t=0}^\infty \delta^t u = u + \delta  \sum_{t=0}^\infty \delta^t u &\leftrightarrow (1-\delta)\sum_{t=0}^\infty \delta^t u  = u \leftrightarrow \sum_{t=0}^\infty \delta^t u = \frac{u}{1-\delta}
    \end{align*}
\end{wideitemize}
\end{frame}

\begin{frame}{Model}
\begin{wideitemize}
    \item A firm and a worker both have discount rate $\delta$ and interact for many periods ($t=1,...,\infty$)
    \item At each period $t$ the following occur:
    \begin{wideitemize}
        \item First the firm offers a flat wage $w$
        \item Second the worker chooses high (H) or low (L) effort
    \end{wideitemize}
    \item High effort has cost $c$, low effort has cost 0.
    \item High effort yields revenue $v$, low effort yields revenue 0.
    \item Firm outside option is 0, worker outside option is $\bar u$.
    \item Assume the firm wants to motivate high effort.
\end{wideitemize}
    
\end{frame}

\begin{frame}{Quick Tutorial: Infinitely Repeated Games}
\begin{wideitemize}
    \item We will not fully cover how to solve infinitely repeated games.
    \item For this class you only need to be able to solve variants of the exact problem in this lecture.
    \item The procedure is as follows:
    \begin{wideitemize}
        \item We guess a simple strategy for the firm and the worker.
        \item We verify that there are no one-shot deviations.
    \end{wideitemize}
    \item For more information on infinitely repeated games see the supplemental slides 
    %``infinitely_repeated_notes.pdf"
\end{wideitemize}
    
\end{frame}


\begin{frame}{Step 1: Guess a simple strategy}
\begin{wideitemize}
    \item Nothing stops the firm and worker from choosing different wages and efforts at each point in time.
    \item They can even condition their choices on the past in complicated ways!
    \item We will look for equilibria where strategies are simple.
    \item We guess that the firm pays a wage $w_H$ as long as the worker exerts high effort, and a wage $w_L$ forever after the worker does not exert high effort (outside option or low effort).    
    \item We guess that the worker exerts high effort as long as they are paid $w_H$. As soon as they are paid anything else, they either exert low effort or take the outside option.
\end{wideitemize}
    
\end{frame}

\begin{frame}{Step 2: Verify}
\begin{wideitemize}
    \item We now need to verify that our guess is an equilibrium.
    \item This means we need to check that both the firm or the worker cannot gain from using some other strategy.
    \item We will focus on the worker's incentives to deviate.
    \item In general there are many other possible strategies, many of which can be complex.
    \item We have a shortcut: the one shot deviation principle.
\end{wideitemize}
    
\end{frame}
\begin{frame}{Step 2: Verify}
\begin{definition}
    The \textbf{one-shot deviation principle} states that a strategy profile is a subgame-perfect Nash equilibrium if and only if no player can increase their payoff by changing a single decision in a single period. 
    \end{definition}

    \begin{wideitemize}
        \item Our guess generates a very simply set of outcomes.
        \item On path: the worker exerts high effort and is paid $w_H$ forever.
        \item Off path: the worker slacked off in the past, is paid $w_L$ forever and exerts low effort forever.
        \item the one-shot-deviation principle says we only need to check that the worker does not want to change course for a single period.
        \item If they don't, our guess is an equilibrium!
    \end{wideitemize}
    
\end{frame}

\begin{frame}{Solving the Model}
\centering
    \huge See the board!

\end{frame}

\begin{frame}{Model Solution}

\begin{theorem}
    If $\delta(v-\bar u) \geq c$, there is an equilibrium where the firm offers a wage of $w_H^*=\frac{c}{\delta}+\bar u$ as long as the worker exerts high effort, and a wage of $w_L^*=0$ forever after the worker exerts low effort once.
\end{theorem}
    \begin{wideitemize}
        \item We say ``there is" because this is only one of many equilibria.
        \item Notice that whenever the firm offers $w_L^*=0$ the worker takes the outside option.
    \end{wideitemize}
\end{frame}

\begin{frame}{Why is this ``Relational"?}

\begin{wideitemize}
    \item The firm pays the worker a high wage and ``trusts" the worker will work hard.
    \item The worker then works hard because they value the future relationship with the firm.
    \item Suppose one party breaks this trust (by exerting low effort or not paying a high wage).
    \item Both players stop working together forever after.
    \item In this way the value of the employment relationships encourages high effort.
\end{wideitemize}
    
\end{frame}

\begin{frame}{Working Hard to Keep a Good Job}

\begin{wideitemize}
    \item The firm does not use performance pay in this model.
    \item There is a fixed wage that is paid regardless of output.
    \item The worker works hard because they want to keep their job.
    \item But the worker only wants to keep their job because it pays better than ``the market"
    \item Thus high salaries paired with the possibility of termination can work like performance pay!
    \item I would argue most US workers are motivated this way.
\end{wideitemize}
    
\end{frame}




\begin{frame}{When Do Relational Contracts Work?}
Recall that our result only holds when:
\[\delta(v-\bar u) \geq c\]
Relational contracts are more likely when...
\begin{wideitemize}
    \item  everyone is more patient ($\uparrow \delta $)
    \item the value of working together is higher ($\uparrow v $)
    \item the worker's outside option is worse ($\downarrow \bar u $)
    \item effort is less costly ($\downarrow c $)
\end{wideitemize}
    
\end{frame}

\begin{frame}{Other Equilibria}

\begin{wideitemize}
    \item The firm's strategy we studied is rather harsh: if the worker slacks, they are essentially fired forever.
    \item Sometimes there are other equilibria with less severe or less eternal consequences.
    \item For example: after low effort pay the low wage for some $T<\infty$ periods, then revert to high wage.
    \item However these work ``less of the time" (for fewer values of $c,\delta, v,\bar u$)
    \item Our harsh strategy works ``more of the time" (for many values of $c,\delta, v,\bar u$)
    \item It is a grim trigger strategy (discuss this).
\end{wideitemize}
    
\end{frame}

\begin{frame}{Efficiency Wages}
\begin{definition}
    Efficiency wages refers to the practice of paying workers above the market rate in order to improve productivity.
\end{definition}
\begin{wideitemize}
    \item Technically speaking, in our model the worker never shirks (exerts low effort)
    \item However, if they do, they are paid a lower wage forever.
    \item Thus the worker is more ``efficient" when wages are higher.
    \item This is a microfoundation (discuss this word) for efficiency wages.
\end{wideitemize}
    
\end{frame}



\end{document}







