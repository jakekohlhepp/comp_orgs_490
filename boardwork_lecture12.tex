\documentclass[14pt]{article}
\usepackage{graphicx} % Required for inserting images
\usepackage{amsmath} 
\usepackage[left=3cm,right=3cm,top=2.5cm,bottom=2.5cm]{geometry}
\newtheorem{lemma}{lemma}
\title{Board Work for Lecture 12}
\author{Jacob Kohlhepp}
\date{\today}

\begin{document}

\maketitle


\section{Relational Contracts}

\subsection{Guessing a Strategy}

We guess two strategies, one for the firm and one for the worker. Our strategies must specify what each player does for any possible history of play. This is just one of infinitely many equilibria. It is easier to analyze because strategies are simple. But this simplicity comes at a cost: if you think about what the worker and the firm do when either deviates, it is rather extreme.

For the worker, we guess they take the job and exert high effort as long as the firm pays a high wage $w_t=w_H$. The first time they  observe any wage other than $w_H$, they never take the job again regardless of what the firm offers in the future.

For the firm, we guess they first post a high wage $w_H$. They continue to post this wage as long as they observe the worker providing high effort in the prior period, $e_{t-1}=H$. The first time the worker does not exert high effort (takes the outside option or exerts low effort) the firm post a low wage $w_L$ forever after regardless of what the worker does in the future.

If the firm and the worker do this strategy, the outcome of the game is simple: the firm posts a high wage, the worker exerts high effort, and this repeats forever. This is called ``on path" because it is what actually happens on the equilibrium path. This is sustained by ``off path" consequences: if either party fails to uphold their end, both sides start ``punishing" the other: the firm posts low wages and the worker never takes the job forever. Notice that these are extreme punishments: the worker gets their outside option $\bar u$ forever, while the firm gets $0$ profit forever.




\subsection{No Incentive to Deviate: Checking One-Shot Deviations}

To show our guessed strategies form a subgame perfect Nash equilibrium, we must check all one shot deviations. We will check the worker's one shot deviations thoroughly, and only briefly consider the firm. This is mainly to make this model less of a burden for tests/homework. In graduate school if you were to use this model you would want to check the firm's side as well. The deviations we need to check are as follows:

\begin{enumerate}
    \item Taking the job and exerting low effort when the worker is supposed to be not taking the job (because they slacked in the past and are being offered $w_L$).
    \item Taking the job and exerting high effort when the worker is supposed to be not taking the job (because they slacked in the past and are being offered $w_L$).
    \item Not taking the job when the worker is supposed to take the job (because they have never slacked before and are being offered $w_H$)
    \item Taking the job and exerting low effort when they are supposed to take the job and exert high effort (because they have never slacked before and are being offered $w_H$).
\end{enumerate}

Think about 1 and 2 for a second. Notice that if the worker does deviation 1 or 2, the consequence starting tomorrow is the same: the low wage forever after. However, deviation 2 also incurs an effort cost of $c$ today. Thus we can see that as long as deviation 1 is not profitable, deviation 2 will also not be profitable. In the same way, deviations 3 and 4 have the same consequence tomorrow: trust is broken and the low wage is offered forever. However, deviation 4 is slightly better because the worker gets the high wage for one period instead of the outside option. Since the high wage must be at least $\bar u +c$, this is always the more attractive deviation. So we if 4 is not profitable, 3 will also not be profitable. This leaves only two deviations we need to check:

\begin{enumerate}
    \item Taking the job and exerting low effort when the worker is supposed to be not taking the job (because they slacked in the past and are being offered $w_L$).
    \item Taking the job and exerting low effort when they are supposed to take the job and exert high effort (because they have never slacked before and are being offered $w_H$).
\end{enumerate}

Before we check for deviations, we need to think about what low wage $w_L$ the firm will choose. If the wage $w_L$ is above $\bar u$ the worker takes the wage and their payoff is just $w_L$. However, if the wage is less than $\bar u$ the worker takes the outside option and gets $\bar u$. Will the firm offer a wage above $\bar u$? Well, a wage above $\bar u$ only makes exerting low effort more tempting, and it also costs the firm money. So it will not offer a wage above $\bar u$. Will it offer exactly $w_L=\bar u$? Well, this gets the worker to take the job, but the worker exerts low effort, so the firm gets $0$ revenue from this and then has to pay out a positive wage. This generates negative profit, so the firm will not do it. 

Therefore the firm will offer any wage where $w_L<\bar u$. The one I suggest specifying is $w_L^*=0$. Any time the firm offers such a low wage, the worker will want to take the outside option. As a result, the worker's payoff in a single period whenever the firm posts a low wage is $\bar u$. 

 With this in hand, let's consider deviation (1): Taking the job and exerting low effort when the worker is supposed to be not taking the job (because they slacked in the past and are being offered $w_L$). Not deviating in this case means never taking the job, which gives the worker $\bar u$ forever: 
\[u(\textit{don't deviate, slacked in past})=\sum_{t=0}^\infty w_L=\frac{\bar u}{1-\delta}\]

Deviating and taking the job and exerting low effort in this case yields the low wage today ($w_L=0$) and no effort cost, and then $\bar u$ starting tomorrow until the end of time:
 \[u(\textit{deviate to accept and low effort, slacked in past}) =  0+\delta \sum_{t=0}^\infty \delta^t \bar u=   \frac{\delta \bar u}{1-\delta}\]

Comparing deviating to not deviating we have that:
   \[u(\textit{don't deviate, slacked in past})  \geq u(\textit{deviate to accept and low effort, slacked in past}) \]
   \[\leftrightarrow \frac{\bar u}{1-\delta} \geq \frac{\delta \bar u}{1-\delta} \leftrightarrow  \bar u \geq 0 \]
The outside option is always weakly positive, so this is always true. There is no incentive to deviate in this way. 

Now consider deviation (2): taking the job and exerting low effort when they are supposed to take the job and exert high effort (because they have never slacked before and are being offered $w_H$). Let's consider first the worker's payoff from doing what they are supposed to do (never slacking off). The worker receives $w_H$ forever and pays the effort cost $c$:
\[u(\textit{don't deviate, never slacked})=\sum_{t=0}^\infty (w_H-c)=\frac{w_H-c}{1-\delta}\]
The worker compares this to a one-shot-deviation: exerting low effort today (at 0 cost), still getting paid $w_H$ today, but then getting offered $w_L$ forever after. Remember, the worker changes action only in one period and is assumed to follow through on our guessed strategy everywhere else. The gain today from this is $w_H$ without any effort cost. The loss is that the firm now posts $w_L$ forever. Putting this all together, we have:
\[u(\textit{deviate, never slacked})=w_H+\delta \sum_{t=0}^\infty w_L = w_H+ \frac{\delta \bar u}{1-\delta}\]
The worker has no incentive to deviate if:
\[u(\textit{don't deviate, never slacked}) \geq u(\textit{deviate, never slacked}) \leftrightarrow \frac{w_H-c}{1-\delta}\geq w_H+ \frac{\delta \bar u}{1-\delta}\]

This is a substantive inequality: it will not always be true for reasonable parameter values ($\delta, c, v, \bar u$). Also, it involves a choice variable: $w_H$.

\subsection{Finding the High Wage}

First, simplify the inequality we derived:

\[\frac{w_H-c}{1-\delta}\geq w_H+ \frac{\delta \bar u}{1-\delta}\]
\[w_H-c \geq (1-\delta) w_H +\delta \bar u\]
\[\delta w_H \geq c +\delta \bar u\]
\[w_H \geq   \frac{c}{\delta}+\bar u\]
We see that this inequality is more likely to be satisfied the higher $w_H$ is. However, the firm maximizes profit by paying the lowest wage that satisfies the inequality, changing the inequality into an equality:
\[w_H^*=\frac{c}{\delta}+\bar u\]
The firm's profit is the revenue less the high wage, because on path the worker never slacks:
\[\pi = \sum_{t=0}^\infty \delta^t (v-w_H^*)=\frac{v-\frac{c}{\delta}-\bar u}{1-\delta}\]
If you stare at this you will notice that profit is increasing in $\delta$. We can also see that a relational contract is profitable if:
\[\delta(v-\bar u) \geq c\]
The more patient the worker and the firm are the more likely this is to hold. The higher the value of working together the more likely this is to hold. The higher the cost of effort and the higher the outside option the more unlikely this is to hold.


\end{document}