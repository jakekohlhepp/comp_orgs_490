%%%%%%%%%%%%%%%%%%%%%%%%%%%%%%%%%%%%%%%%%
% Beamer Presentation
% LaTeX Template
% Version 1.0 (10/11/12)
%
% This template has been downloaded from:
% http://www.LaTeXTemplates.com
%
% License:
% CC BY-NC-SA 3.0 (http://creativecommons.org/licenses/by-nc-sa/3.0/)
%
%%%%%%%%%%%%%%%%%%%%%%%%%%%%%%%%%%%%%%%%%

%----------------------------------------------------------------------------------------
%	PACKAGES AND THEMES
%----------------------------------------------------------------------------------------

\documentclass[aspectratio=169,usenames,dvipsnames]{beamer}

\usepackage[utf8]{inputenc}
\usepackage{booktabs}
\usepackage{tabularx}
\usepackage[authordate,bibencoding=auto,strict,backend=biber,natbib]{biblatex-chicago}
\addbibresource{bib.bib}
\usepackage{graphicx}
% \hypersetup{
%     colorlinks,
%     %citecolor=black,
%     linkcolor=black
% }
\usepackage{array}
\usepackage{caption}
\usepackage{threeparttable}
\usepackage{epigraph} 
\usepackage{lscape}
\usepackage{adjustbox}
\newcommand*{\Scale}[2][4]{\scalebox{#1}{\ensuremath{#2}}}%
\usepackage{import}
\newenvironment{wideitemize}{\itemize\addtolength{\itemsep}{10pt}}{\enditemize}
\usepackage{amsmath}
\usepackage{csvsimple}
\usepackage{siunitx}
\usepackage{filecontents}
\usepackage{rotating}
\usepackage{multirow}
\usepackage{amsmath}
\usepackage{subcaption}
\usepackage{appendixnumberbeamer}
\usepackage{float}
\usepackage{amsmath}
\usepackage{csvsimple}
\usepackage{hyperref}
\newtheorem{proposition}{Proposition}
\usepackage{xcolor}
\def\boxit#1#2{%
    \smash{\color{red}\fboxrule=1pt\relax\fboxsep=2pt\relax%
    \llap{\rlap{\fbox{\phantom{\rule{#1}{#2}}}}~}}\ignorespaces
}
\newenvironment{variableblock}[3]{%
  \setbeamercolor{block body}{#2}
  \setbeamercolor{block title}{#3}
  \begin{block}{#1}}{\end{block}}
\usepackage{appendixnumberbeamer}
\usepackage{tikz,pgfplots}
\usepackage{tkz-fct}
\usepackage{amsthm}
\pgfplotsset{compat=1.10}
\usepgfplotslibrary{fillbetween}
\mode<presentation> {
\AtBeginSection[]
{
    \begin{frame}
        \frametitle{Table of Contents}
        \tableofcontents[currentsection]
    \end{frame}
}
% The Beamer class comes with a number of default slide themes
% which change the colors and layouts of slides. Below this is a list
% of all the themes, uncomment each in turn to see what they look like.

\usetheme{default}
%\usetheme{AnnArbor}
%\usetheme{Antibes} -
%\usetheme{Bergen}
%\usetheme{Berkeley}
%\usetheme{Berlin}
%\usetheme{Boadilla}
%\usetheme{CambridgeUS}
%\usetheme{Copenhagen} -
%\usetheme{Darmstadt}
%\usetheme{Dresden}
%\usetheme{Frankfurt}
%\usetheme{Goettingen}
%\usetheme{Hannover}
%\usetheme{Ilmenau}
%\usetheme{JuanLesPins}
%\usetheme{Luebeck}
%\usetheme{Madrid}
%\usetheme{Malmoe}
%\usetheme{Marburg}
%\usetheme{Montpellier}
%\usetheme{PaloAlto}
%\usetheme{Pittsburgh}
%\usetheme{Rochester} -
%\usetheme{Singapore}
%\usetheme{Szeged}
%\usetheme{Warsaw}

% As well as themes, the Beamer class has a number of color themes
% for any slide theme. Uncomment each of these in turn to see how it
% changes the colors of your current slide theme.

%\usecolortheme{albatross}
%\usecolortheme{beaver}
%\usecolortheme{beetle}
%\usecolortheme{crane}
%\usecolortheme{dolphin}
%\usecolortheme{dove}
%\usecolortheme{fly}
%\usecolortheme{lily}
%\usecolortheme{orchid}
%\usecolortheme{rose}
%\usecolortheme{seagull}
%\usecolortheme{seahorse}
%\usecolortheme{whale}
%\usecolortheme{wolverine}

%\setbeamertemplate{footline} % To remove the footer line in all slides uncomment this line
%\setbeamertemplate{footline}[frame number] % To replace the footer line in all slides with a simple slide count uncomment this line
\setbeamertemplate{theorems}[numbered]
\setbeamertemplate{navigation symbols}{} % To remove the navigation symbols from the bottom of all slides uncomment this line
}
\setbeamertemplate{caption}{\raggedright\insertcaption\par}
  \setbeamertemplate{enumerate items}[default]
  %\setbeamertemplate{page number in head/foot}{\insertframenumber}
\usepackage{graphicx} % Allows including images
\usepackage{booktabs} % Allows the use of \toprule, \midrule and \bottomrule in tables
%\usepackage {tikz}
\newtheorem*{theorem*}{Theorem}
\newtheorem*{lemma*}{Lemma}
\newtheorem*{proposition*}{Proposition}
\newtheorem*{corollary*}{Corollary}
\newtheorem*{definition*}{Definition}
\DeclareMathOperator*{\argmin}{arg\,min}
\newtheorem*{assumption}{Assumption}
\usetikzlibrary {positioning}
\renewcommand{\arraystretch}{1.5}
\newcommand\hideit[1]{%
  \only<0| handout:1>{\mbox{}}%
  \invisible<0| handout:1>{#1}}
\usepackage[default]{lato}

\setbeamercolor{block body alerted}{bg=alerted text.fg!10}
\setbeamercolor{block title alerted}{bg=alerted text.fg!20}
\setbeamercolor{block body}{bg=structure!10}
\setbeamercolor{block title}{bg=structure!20}
\setbeamercolor{block body example}{bg=green!10}
\setbeamercolor{block title example}{bg=green!20}


\makeatletter
\let\save@measuring@true\measuring@true
\def\measuring@true{%
  \save@measuring@true
  \def\beamer@sortzero##1{\beamer@ifnextcharospec{\beamer@sortzeroread{##1}}{}}%
  \def\beamer@sortzeroread##1<##2>{}%
  \def\beamer@finalnospec{}%
}
\makeatother
%\usepackage {xcolor}

%----------------------------------------------------------------------------------------
%	TITLE PAGE
%----------------------------------------------------------------------------------------

\title[diss]{Lecture 16: Teamwork} % The short title appears at the bottom of every slide, the full title is only on the title page
\author{Compensation in Organizations} % Your name
\institute[shortinst]{Jacob Kohlhepp}
\date{\today} % Date, can be changed to a custom date

\begin{document}

\begin{frame}
\titlepage % Print the title page as the first slide

\end{frame}

\begin{frame}{Teamwork vs. Relative Performance Pay}

\begin{wideitemize}
    \item We have studied when and why it can be helpful to pay workers based on their performance relative to others.
    \item That is, when should wages depend on more than just own output?
    \item But we always assumed workers produced separately.
    \begin{wideitemize}
            \item i.e. $y_1=e_1+\epsilon_1$ and $y_2=e_2+\epsilon_2$

    \end{wideitemize}
    \item But what if workers produce together?
    \begin{wideitemize}
            \item i.e. $y=e_1+e_2$
    \end{wideitemize}
    \item We call this teamwork.
\end{wideitemize}
    
\end{frame}

\begin{frame}
\centering
    \huge Discussion: Friebel, Heinz, Krueger, and Zubanov (2017)

\end{frame}

\begin{frame}{Model (``Moral Hazard in Teams," Holmstrom (1982))}

\begin{wideitemize}
    \item There are $N$ workers, indexed by $i=1,...,N$
    \item Each worker can exert effort $e_i$ at cost $c_i(e_i)$
    \item We will refer to $e=(e_1,...,e_N)$ as a list which contains everyone's effort.
    \item Output is the sum of everyone's effort: $y(e)=e_1+e_2+...+e_N$
    \item The firm can pay a wage to each worker based only on team output $w_i(y(e))$
    \item For technical reasons we assume all $c_i(e_i)$ are convex, increasing  and differentiable with $c_i(0)=0$.
\end{wideitemize}
\end{frame}


\begin{frame}{Understanding Free Riding}

\begin{wideitemize}
    \item Consider the case where we use the intuitive wage $w_i(y(e))=y(e)/N$
    \item That is everyone splits everything evenly.
    \item We will see that people free ride.
    \item This is similar to the struggles of group projects in school.
\end{wideitemize}

\begin{definition}
    Free riding is the under supply of effort because the marginal benefits of effort are shared.
\end{definition}
    
\end{frame}


\begin{frame}{Understanding Free Riding: Solution}

\Huge See the board!
    
\end{frame}

\begin{frame}{The First-Best Benchmark}

\begin{wideitemize}
    \item Consider the case where the firm can choose effort levels directly.
    \item Suppose the firm maximizes total surplus (output minus total effort costs)
\end{wideitemize}
    
\end{frame}


\begin{frame}{The First-Best Benchmark: Solution}

\Huge See the board!
    
\end{frame}

\begin{frame}{Is There Any Way to Get First-Best Effort Using Wages?}
\begin{wideitemize}
    \item Now return to the actual model where the firm can only control effort via wages.
    \item Wages can only depend on total output.
    \item We consider several types of wage schemes.
    \item We ask: can the wage scheme achieve $e^*$?
    \item We ignore individual rationality/outside options.
\end{wideitemize}
    
\end{frame}

\begin{frame}{Partnerships}

\begin{definition*}
    A partnership is a wage scheme where $w_i(y(e))\geq 0$ and:
    \[\sum_{i=1}^N w_i(y(e))=y(e)\]
    for every output $y(e)$.
\end{definition*}

    \begin{wideitemize}
        \item This is also called budget balanced because everything that is produced is paid out.
        \item It is called a partnership because we are choosing the share each person gets.
        \item Assume for this case only that wages are differentiable, so derivatives are well defined.
    \end{wideitemize}
\end{frame}

\begin{frame}{Can Partnerships Achieve the First-Best?}

\Huge See the board
    
\end{frame}


\begin{frame}{Can Partnerships Achieve the First-Best?}

\begin{theorem*}
    There does not exist a partnership which achieves the first-best level of effort $e^*$.
\end{theorem*}
    \begin{wideitemize}
        \item There is inherent free-riding with teamwork.
        \item To overcome free-riding we must pay each person the marginal dollar produced
        \item But because the budget must balance, there is only one marginal dollar!
    \end{wideitemize}
\end{frame}


\begin{frame}{Can Group Bonuses Achieve the First-Best?}
We now relax budget balance.
\begin{definition*}
    A group bonus is a wage scheme where:
    \[w_i(y)=\begin{cases}b_i \text{ if } y(e)\geq \bar y \\
    0 \text{ else }
    \end{cases}\]
\end{definition*}
    \begin{wideitemize}
        \item This is clearly not budget balanced because if output is below $\bar y$ but not 0:
        \[\sum_i w_i(y(e))=0 < y(e)\]
        \item If there was an outside option for the workers pay would look like a flat wage plus a bonus if a group target is achieved.
    \end{wideitemize}
\end{frame}

\begin{frame}{Can Group Bonuses Achieve the First-Best?}

\Huge See the board
    
\end{frame}

\begin{frame}{Can Group Bonuses Achieve the First-Best?}
\begin{theorem*}
    A group bonus with $\sum_i b_i=y(e^*)$, $b_i>c_i(e_i^*)$ and $\bar y = y(e^*)$ achieves the first-best level of effort.
\end{theorem*}
    
\end{frame}

% \begin{frame}{Can Group Bonds Achieve the First-Best?}
% We now relax budget balance and allow wages to be negative.
% \begin{definition}
%     A group bond is a wage scheme where:
%     \[w_i(y)=-b_i+\begin{cases}  b_i \text{ if } y(e)\geq \bar y\\
%     0 \text{ else }
%     \end{cases}\]
% \end{definition}
%     \begin{wideitemize}
%         \item This is called a bond because like in finance you pay upfront and get something back later.
%         \item This is never budget balanced. 
%     \end{wideitemize}
% \end{frame}

% \begin{frame}{Can A Group Bond Achieve the First-Best?}

% \Huge See the board
    
% \end{frame}
% \begin{frame}{Can A Group Bond Achieve the First-Best?}
% \begin{theorem}
%     A group bond with $b_i=y(e^*)$ and $\bar y = y(e^*)$ achieves the first-best level of effort.
% \end{theorem}
    
% \end{frame}

\begin{frame}{Money Burning}

\begin{wideitemize}
    \item We showed that organizations which give out everything in wages cannot achieve the first-best.
    
    \item We showed that there are organizations that do not balance the budget that can!

    \item Specifically group bonuses require us to commit to burn money.
    \item That is, some of the output must be destroyed or given to someone else.
    \item Discussion: if the firm is owned by the workers is money burning credible?
    \pause

    \item No: once output is produced we will want to pay it out.
\end{wideitemize}
    
\end{frame}

\begin{frame}{Worker Co-Op vs. Corporations vs. Partnerships}

\huge Discussion: separation of ownership and control
    
\end{frame}
\end{document}




