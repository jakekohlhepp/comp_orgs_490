\documentclass{article}
\usepackage{graphicx} % Required for inserting images
\usepackage{amsmath} 
\newtheorem{lemma}{lemma}
\usepackage{geometry}
\geometry{margin=1in}
\title{Board Work for Lecture 8: Meaning and Performance}
\author{Jacob Kohlhepp}
\date{\today}

\begin{document}

\maketitle

\section{Setup}

\begin{enumerate}
    \item Change the cost of effort:
    \[c(e_{ij}) = \frac{e_{ij}^2}{2(1+\lambda_i m_{ij})} \]

    \item The cost of effort now varies across people and jobs!
    \item Meaning is measured by $m_{ij}$ because:
    \[\lim_{m_{ij}\to \infty} c(e_{ij})= \lim_{m_{ij}\to \infty}\frac{e_{ij}^2}{2(1+\lambda_i m_{ij})}=0\]
    \item Output is given by $y_{ij}=\theta_i e_{ij} + \epsilon$
    \item The worker is risk neutral.
    
\end{enumerate}

\section{Solution}

We can begin to solve this model in a way similar to effort-based pay (lecture 3). Specifically, we proceed by backwards induction. Consider first the worker $i$'s choice of effort after wages are set and the worker has taken job $j$:

\[\max_{e_{ij}} \alpha_{ij}+ \beta_{ij} \theta_i e_{ij} - c(e_i)\]
which given the modified cost of effort is:
\[\max_{e_{ij}} \alpha_{ij}+ \beta_{ij}\theta_i  e_{ij} - \frac{e_{ij}^2}{2(1+\lambda_i m_{ij})}\]
Taking the FOC:
\[[e_{ij}]: \beta_{ij}\theta_i -\frac{e_{ij}}{(1+\lambda_i m_{ij})} = 0 \leftrightarrow e_{ij} =  \beta_{ij}\theta_i (1+\lambda_i m_{ij} ) \]
Notice that everything is worker and job specific because of the differences across workers and jobs in meaning. Now comes the first big difference: in lecture 3 and 4, we would next consider the worker's decision to take the job and make the argument that the firm extracts the worker's value from the job via base pay so that:
\[ \alpha_{ij} = \bar u_i -\beta_{ij} \theta_i e_{ij} + \frac{e_{ij}^2}{2(1+\lambda_i m_{ij})} \]
However, this step does not make sense in this paper because they are running an experiment and ``shocking" the meaning workers get from the job, while ostensibly trying to hold other personnel policies of the firm fixed (which would include base pay). So they do not assume the firm has extracted all surplus. Instead, they observe that the profit of the firm will be the fraction of output the firm keeps ($1-\beta_{ij}$) less an exogenous (fixed) salary $\alpha_{ij}$. So they skip the middle step and jump straight to profit maximization. The firm's profit is:
\[\max_{\beta_{ij}} \theta_i (1-\beta_{ij}) e_{ij} - \alpha_{ij} \]
Plugging in $e_{ij}$:
\[\max_{\beta_{ij}} \theta_i (1-\beta_{ij}) \beta_{ij}\theta_i (1+\lambda_i m_{ij} ) - \alpha_{ij}= \max_{\beta_{ij}} \theta_i (\beta_{ij}-\beta_{ij}^2) \beta_{ij}\theta_i (1+\lambda_i m_{ij} ) - \alpha_{ij}\]
Taking the FOC:
\[[\beta_{ij}]: 1-2\beta_{ij}=0 \leftrightarrow \beta_{ij}=\frac{1}{2} \]
The authors then return to the decision to take the job afterwards. Worker $i$'s utility at job $j$ is given by:
\[\alpha_{ij}+ \beta_{ij}\theta_i  \beta_{ij}\theta_i (1+\lambda_i m_{ij} ) - \frac{(\beta_{ij}\theta_i (1+\lambda_i m_{ij} ))^2}{2(1+\lambda_i m_{ij})}\]
Simplifying and plugging in $\beta_{ij}=1/2$:
\[\alpha_{ij}+ \frac{1}{4}\theta_i^2 (1+\lambda_i m_{ij} ) - \frac{\theta_i^2 (1+\lambda_i m_{ij} )}{8}=\alpha_{ij}+ \frac{1}{8}\theta_i^2 (1+\lambda_i m_{ij} )\]
Suppose the worker's outside option is a job $j=a$ and the job they are considering is a job $j=p$, then the worker will work job $p$ if:
\[\alpha_{ip}+ \frac{1}{8}\theta_i^2 (1+\lambda_i m_{ip} ) \geq \alpha_{ia}+ \frac{1}{8}\theta_i^2 (1+\lambda_i m_{ia} )\]
Manipulation of the inequality yields:
\[\alpha_{ip} -\alpha_{ia} \geq \frac{\lambda_i \theta_i^2}{8} (m_{ia}-m_{ip}) \]
To stay at the job they are in, the worker must have a meaning of at least:
\[m_{ia}^* = m_{ia}-\frac{8(\alpha_{ip}-\alpha_{ia})}{\lambda_i \theta_i} \]

\end{document}