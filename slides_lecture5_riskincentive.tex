%%%%%%%%%%%%%%%%%%%%%%%%%%%%%%%%%%%%%%%%%
% Beamer Presentation
% LaTeX Template
% Version 1.0 (10/11/12)
%
% This template has been downloaded from:
% http://www.LaTeXTemplates.com
%
% License:
% CC BY-NC-SA 3.0 (http://creativecommons.org/licenses/by-nc-sa/3.0/)
%
%%%%%%%%%%%%%%%%%%%%%%%%%%%%%%%%%%%%%%%%%

%----------------------------------------------------------------------------------------
%	PACKAGES AND THEMES
%----------------------------------------------------------------------------------------

\documentclass[aspectratio=169,usenames,dvipsnames]{beamer}

\usepackage[utf8]{inputenc}
\usepackage{booktabs}
\usepackage{tabularx}
\usepackage[authordate,bibencoding=auto,strict,backend=biber,natbib]{biblatex-chicago}
\addbibresource{bib.bib}
\usepackage{graphicx}
% \hypersetup{
%     colorlinks,
%     %citecolor=black,
%     linkcolor=black
% }
\usepackage{array}
\usepackage{caption}
\usepackage{threeparttable}
\usepackage{epigraph} 
\usepackage{lscape}
\usepackage{adjustbox}
\newcommand*{\Scale}[2][4]{\scalebox{#1}{\ensuremath{#2}}}%
\usepackage{import}
\newenvironment{wideitemize}{\itemize\addtolength{\itemsep}{10pt}}{\enditemize}
\usepackage{amsmath}
\usepackage{csvsimple}
\usepackage{siunitx}
\usepackage{filecontents}
\usepackage{rotating}
\usepackage{multirow}
\usepackage{amsmath}
\usepackage{subcaption}
\usepackage{appendixnumberbeamer}
\usepackage{float}
\usepackage{amsmath}
\usepackage{csvsimple}
\usepackage{hyperref}
\newtheorem{proposition}{Proposition}
\usepackage{xcolor}
\def\boxit#1#2{%
    \smash{\color{red}\fboxrule=1pt\relax\fboxsep=2pt\relax%
    \llap{\rlap{\fbox{\phantom{\rule{#1}{#2}}}}~}}\ignorespaces
}
\newenvironment{variableblock}[3]{%
  \setbeamercolor{block body}{#2}
  \setbeamercolor{block title}{#3}
  \begin{block}{#1}}{\end{block}}
\usepackage{appendixnumberbeamer}
\usepackage{tikz,pgfplots}
\usepackage{tkz-fct}
\usepackage{amsthm}
\pgfplotsset{compat=1.10}
\usepgfplotslibrary{fillbetween}
\mode<presentation> {
\AtBeginSection[]
{
    \begin{frame}
        \frametitle{Table of Contents}
        \tableofcontents[currentsection]
    \end{frame}
}
% The Beamer class comes with a number of default slide themes
% which change the colors and layouts of slides. Below this is a list
% of all the themes, uncomment each in turn to see what they look like.

\usetheme{default}
%\usetheme{AnnArbor}
%\usetheme{Antibes} -
%\usetheme{Bergen}
%\usetheme{Berkeley}
%\usetheme{Berlin}
%\usetheme{Boadilla}
%\usetheme{CambridgeUS}
%\usetheme{Copenhagen} -
%\usetheme{Darmstadt}
%\usetheme{Dresden}
%\usetheme{Frankfurt}
%\usetheme{Goettingen}
%\usetheme{Hannover}
%\usetheme{Ilmenau}
%\usetheme{JuanLesPins}
%\usetheme{Luebeck}
%\usetheme{Madrid}
%\usetheme{Malmoe}
%\usetheme{Marburg}
%\usetheme{Montpellier}
%\usetheme{PaloAlto}
%\usetheme{Pittsburgh}
%\usetheme{Rochester} -
%\usetheme{Singapore}
%\usetheme{Szeged}
%\usetheme{Warsaw}

% As well as themes, the Beamer class has a number of color themes
% for any slide theme. Uncomment each of these in turn to see how it
% changes the colors of your current slide theme.

%\usecolortheme{albatross}
%\usecolortheme{beaver}
%\usecolortheme{beetle}
%\usecolortheme{crane}
%\usecolortheme{dolphin}
%\usecolortheme{dove}
%\usecolortheme{fly}
%\usecolortheme{lily}
%\usecolortheme{orchid}
%\usecolortheme{rose}
%\usecolortheme{seagull}
%\usecolortheme{seahorse}
%\usecolortheme{whale}
%\usecolortheme{wolverine}

%\setbeamertemplate{footline} % To remove the footer line in all slides uncomment this line
%\setbeamertemplate{footline}[frame number] % To replace the footer line in all slides with a simple slide count uncomment this line
\setbeamertemplate{theorems}[numbered]
\setbeamertemplate{navigation symbols}{} % To remove the navigation symbols from the bottom of all slides uncomment this line
}
\setbeamertemplate{caption}{\raggedright\insertcaption\par}
  \setbeamertemplate{enumerate items}[default]
  %\setbeamertemplate{page number in head/foot}{\insertframenumber}
\usepackage{graphicx} % Allows including images
\usepackage{booktabs} % Allows the use of \toprule, \midrule and \bottomrule in tables
%\usepackage {tikz}
\newtheorem*{theorem*}{Theorem}
\newtheorem*{lemma*}{Lemma}
\newtheorem*{proposition*}{Proposition}
\newtheorem*{corollary*}{Corollary}
\newtheorem*{definition*}{Definition}
\DeclareMathOperator*{\argmin}{arg\,min}
\newtheorem*{assumption}{Assumption}
\usetikzlibrary {positioning}
\renewcommand{\arraystretch}{1.5}
\newcommand\hideit[1]{%
  \only<0| handout:1>{\mbox{}}%
  \invisible<0| handout:1>{#1}}
\usepackage[default]{lato}

\setbeamercolor{block body alerted}{bg=alerted text.fg!10}
\setbeamercolor{block title alerted}{bg=alerted text.fg!20}
\setbeamercolor{block body}{bg=structure!10}
\setbeamercolor{block title}{bg=structure!20}
\setbeamercolor{block body example}{bg=green!10}
\setbeamercolor{block title example}{bg=green!20}


\makeatletter
\let\save@measuring@true\measuring@true
\def\measuring@true{%
  \save@measuring@true
  \def\beamer@sortzero##1{\beamer@ifnextcharospec{\beamer@sortzeroread{##1}}{}}%
  \def\beamer@sortzeroread##1<##2>{}%
  \def\beamer@finalnospec{}%
}
\makeatother
%\usepackage {xcolor}

%----------------------------------------------------------------------------------------
%	TITLE PAGE
%----------------------------------------------------------------------------------------

\title[diss]{Lecture 5: The Risk-Incentive Trade-Off} % The short title appears at the bottom of every slide, the full title is only on the title page
\author{Compensation in Organizations} % Your name
\institute[shortinst]{Jacob Kohlhepp}
\date{\today} % Date, can be changed to a custom date

\begin{document}

\begin{frame}
\titlepage % Print the title page as the first slide

\end{frame}


\begin{frame}{Recalling Performance Pay Results}

\begin{theorem}
        When wages depend only on output, effort is $e_{p}$ which solves 
        \[c'(e_{p})= \frac{1}{1+r \sigma^2 c''(e_{p})}\]
        and $\beta_{p} =c'(e_{p}),\alpha_{p} =\bar u - \beta_{p} e_{p}+r \beta^2\sigma^2/2+c(e_{p})$.
    \end{theorem}
    \begin{wideitemize}
        \item Focus on: $c'(e_{p})= \frac{1}{1+r \sigma^2 c''(e_{p})}$
        \item As $\sigma^2$ rises, $\beta_p$ falls
        \item Incentives/performance pay/bonuses decrease with noise/luck/randomness.
        \item This is the risk-incentive trade-off.
        \item $e_p$ also generally falls, although this is harder to prove.
    \end{wideitemize}
    
\end{frame}

\begin{frame}
\centering
    \huge Discussion: Drago and Garvey (1998)
\end{frame}


% \begin{frame}{Effort-Based Pay}
% \begin{theorem}
%         When wages depend directly on effort, effort is $e^*$ which solves $c'(e^*)=1$ and $\beta^*=1, \alpha^*=\bar u+c(e^*)-1$
%     \end{theorem}

%     \begin{wideitemize}
%         \item Question: how does $r$ impact outcomes under effort-based pay?
%         \pause
%         \item Question: Why does it have no impact?
%     \end{wideitemize}
% \end{frame}

% \begin{frame}{Performance-Based Pay}
%     \begin{theorem}
%         When wages depend only on output, effort is $e_{p}$ which solves 
%         \[c'(e_{p})= \frac{1}{1+r \sigma^2 c''(e_{p})}\]
%         and $\beta_{p} =c'(e_{p}),\alpha_{p} =\bar u - \beta_{p} e_{p}+r \beta^2\sigma^2/2+c(e_{p})$.
%     \end{theorem}

%     \begin{wideitemize}
%         \item Question: How does $r$ impact outcomes under performance-based pay?
%         \pause
%         \item Question: Why does it have this impact?
%     \end{wideitemize}
% \end{frame}

\begin{frame}{Trade-Off is (in some sense) Theoretically Robust}
\begin{wideitemize}
    \item The model we solved is a very special type of principal-agent model.
    \item We could allow for nonlinear wages, general output, etc.
    \item However, even when we do this, we still find that risk aversion reduces incentives and effort.
\end{wideitemize}
    
\end{frame}

\begin{frame}{Profit}

\begin{wideitemize}
    \item The level of effort is tied to total surplus/efficiency.
    \item As effort rises and goes to $e^*$, total surplus/efficiency rises.
    \item Therefore, since $e_p$ falls with $\sigma^2$, profit should fall.
    \item To see this very clearly, let's derive profit when $c(e)=e^2/2$.
    \item Remember: $e_p=\beta_p=\frac{1}{1+r \sigma^2}$ in this case!
\end{wideitemize}

\end{frame}

\begin{frame}{Profit Under Performance Pay and $c(e)=e^2/2$}

    \begin{align*}
        \pi_p &= e_p-c(e_p)-\frac{r\beta^2_p \sigma^2}{2}-\bar u\\
        &= \frac{1}{1+r \sigma^2} - \frac{1}{2(1+r \sigma^2)} - r\sigma^2\frac{1}{2(1+r \sigma^2)} -\bar u\\
        &= \frac{1}{1+r \sigma^2} \bigg ( 1- \frac{1}{2(1+r \sigma^2)}-r\sigma^2\frac{1}{2(1+r \sigma^2)}-\bar u\\
        &= \frac{1}{1+r \sigma^2} \bigg ( 1- \frac{1+r\sigma^2}{2(1+r \sigma^2)} \bigg )-\bar u\\
        &=\frac{1}{1+r \sigma^2} \bigg ( \frac{2(1+r \sigma^2)}{2(1+r \sigma^2)}- \frac{1+r\sigma^2}{2(1+r \sigma^2)} \bigg )-\bar u\\
        &=\frac{1}{1+r \sigma^2} \bigg (  \frac{1+r\sigma^2}{2(1+r \sigma^2)} \bigg )-\bar u\\
        &=\frac{1}{2}\frac{1}{1+r \sigma^2}-\bar u
    \end{align*}
    
    
\end{frame}


\begin{frame}{Performance Pay vs. Some Alternative Method}

\begin{wideitemize}
    \item Suppose the firm can use performance pay or some alternative with profit $\pi_{alt}$.
    \item It will choose performance paif if:
    \[\pi_p = \frac{1}{2}\frac{1}{1+r \sigma^2}-\bar u \geq \pi_{alt}\]
    \item As $\sigma^2$ rises, performance pay becomes less likely.
    \item This is a testable implication of the risk-incentive trade-off.
\end{wideitemize}
\end{frame}



\section{Evidence for the Risk-Incentive Trade-Off}



\begin{frame}
\centering
    \huge Discussion: Gaynor and Gertler (1995)
\end{frame}


\begin{frame}{``Moral hazard and risk spreading in partnerships" (Gaynor and Gertler 1995)}
    \begin{wideitemize}
        \item Setting is medical group practices.
        \item At time of article, 61\% of physicians worked in group settings.
        \item In these groups, physicians are co-owners making decisions together, including fees and resource allocations.
        \item Physicians have specialties but demand can be highly variable across specialties (exmaple?)
        \item Data shows how much compensation depends on output (1-10), size of group, price of an office visit, and a measure of physician risk aversion.
    \end{wideitemize}
\end{frame}


\begin{frame}{``Moral hazard and risk spreading in partnerships" (Gaynor and Gertler 1995)}
    \begin{wideitemize}
        \item Concern: self-reported measure of risk aversion may just reflect what physician experiences rather than what they ``want."
        \item Alleviated: risk aversion measure is negatively correlated with performance pay measure.
        \item 10\% increase in incentives leads to 3.5\% more visits.
        \item Price decreases demand (why is this a good sign?)
    \end{wideitemize}
\end{frame}


\begin{frame}{``Moral hazard and risk spreading in partnerships" (Gaynor and Gertler 1995)}
    \begin{wideitemize}
        \item Varying risk aversion variable across full range alters ofice visits by over 877 per year.
        \item The most risk averse physicians make about \$11,582 less than the least risk averse.
        \item This is 10\% of mean income.
        \item It is a measure of the efficiency loss of partnerships.
    \end{wideitemize}
\end{frame}



\begin{frame}[c]{``Moral hazard and risk spreading in partnerships" (Gaynor and Gertler 1995)}
\centering
\includegraphics[width=0.7\textwidth]{pictures/demand_physicians.png}
\end{frame}

\begin{frame}[c]{``Moral hazard and risk spreading in partnerships" (Gaynor and Gertler 1995)}
\centering
\includegraphics[width=0.7\textwidth]{pictures/physicians_comp_demand.png}
\end{frame}
\begin{frame}{``Moral hazard and risk spreading in partnerships" (Gaynor and Gertler 1995)}
\centering
\includegraphics[width=0.9\textwidth]{pictures/physicians_riskspreading.png}
\end{frame}


\begin{frame}{``Is a Higher Calling Enough? Incentive Compensation in the Church" (Hartzell, Parsons, Yermack 2010)}

\begin{wideitemize}
    \item Recall the setting: Methodist ministers whoa re rotated but have their pay set by their congregation.
    \item The paper shows evidence that Methodist ministers pay is consistent with pay for performance.
    \item Ministers get more pay for adding more people to their church.
    \item Let's focus on the portions related to the risk-incentive trade-off.
\end{wideitemize}
\end{frame}

\begin{frame}{``Is a Higher Calling Enough? Incentive Compensation in the Church" (Hartzell, Parsons, Yermack 2010)}

\begin{wideitemize}
    \item The paper actually tests the risk-incentive trade-off two ways:
    \begin{wideitemize}
        \item By estimating how volatile each church location's membership is over 43 years.
        \item By comparing oil-driven locations vs. non-oil drivne locations
    \end{wideitemize}
    \item Both methods have problems (they are not the core result of the paper).
    \item But both point towards the existence of the risk-incentive trade-off.
\end{wideitemize}
\end{frame}

\begin{frame}{Hartzell, Parsons, Yermack 2010: Direct Computation of Volatility}

\begin{wideitemize}
    \item Idea: use the entire time series of 43 years to compute the standard deviation of membership changes.
    \item In our performance pay model, this is a proxy for $\sigma^2$ if we assume that effort does not change much over time.
    \item The authors classify a church as ``High Volatility" if the church's standard deviation is above the median.
    \item ``High Volatility" churches pay on average \$10.75 less per new member than the rest.
    \item This is 50\% less ``bonus" ($\beta$)
    \item One issue: perhaps at volatile churches the value of a member is just less (the coefficient on $e$ is smaller) so they use lower incentives.    
\end{wideitemize}
\end{frame}




\begin{frame}[c]{Hartzell, Parsons, Yermack 2010: Oil Booms}
\centering
\includegraphics[width=0.9\textwidth]{pictures/oilboom.png}
\end{frame}


\begin{frame}{Hartzell, Parsons, Yermack 2010: Oil Booms}
\begin{wideitemize}
    \item Idea: use oil booms as proxy for volatility.
    \item This is ``better" because oil booms are an exogenous or external factor impacting attendance.
    \item Boom and bust clearly bring people in and out (direct cyclical effect).
    \item But also, as a town gets richer from oil, attendance may also drop.
\end{wideitemize}
\end{frame}


\begin{frame}[c]{Hartzell, Parsons, Yermack 2010: Results}
\centering
\includegraphics[width=0.9\textwidth]{pictures/oil_paytable.png}
\end{frame}


\section{Evidence Against the Risk-Incentive Trade-Off}


\begin{frame}[c]{``The Tenous Trade-off Between Risk and Incentives" (Prendergast 2002)}
\centering
    \includegraphics[width=0.9\textwidth]{pictures/ceos_tenous.png}
\end{frame}


\begin{frame}[c]{``The Tenous Trade-off Between Risk and Incentives" (Prendergast 2002)}
The amount sharecroppers keep ($\beta$) is positively associated with noise ($\sigma^2$).
\centering
    \includegraphics[width=0.9\textwidth]{pictures/sharecroppers.png}
\end{frame}

\begin{frame}[c]{``Incentive Contracting and the Franchise Design.” (Lafontaine and Slade (2001))}
\centering
    \includegraphics[width=0.9\textwidth]{pictures/franchise_risk.png}
\end{frame}

\section{(Potential) Resolutions of the Controversy}


\begin{frame}{Choosing What to Do}

\begin{wideitemize}
    \item Our model assumes the worker only chooses $e$.
    \item But what if the worker also chooses what to exert effort on.
    \item For this discussion assume the firm can measure effort.
    \item What if also the worker knows the benefit of each activity, but the firm does not?
\end{wideitemize}



\small Supporting source: ``The Tenous Trade-off Between Risk and Incentives" (Prendergast 2002)

    
\end{frame}


\begin{frame}{Choosing What to Do}

\begin{wideitemize}
    \item The firm has two choices: (1) choose what the agent can do and specify an effort based wage (2) let the agent choose and specify an output based wage.
    \item We can get the right effort from (1), but the firm might choose the wrong  activity.
    \item We will get the wrong effort from (2) but the worker will choose the right activity (why?)
    \item The key observation is that the maximum of random variables generally depends on variance.
\end{wideitemize}

\small Supporting source: ``The Tenous Trade-off Between Risk and Incentives" (Prendergast 2002)

    
\end{frame}


\begin{frame}{Choosing What to Do}

\begin{wideitemize}
     \item For two normal random variables with mean 0 and variance $\sigma^2$ we have:
    \[E[\max\{\epsilon_1, \epsilon_2\}]=\frac{\sigma}{\pi} \]
    \item So ``delegation" using performance pay $w(y)$ (option 2) becomes better relative to effort-based pay and command and control (option 1) when variance is larger.
    \item This overturns our earlier result and suggests a positive relationship between risks and performance pay!
    \item CEOs, franchising, etc. requires making complex decisions with situation-specific knowledge.
\end{wideitemize}



\small Supporting source: ``The Tenuous Trade-off Between Risk and Incentives" (Prendergast 2002)

    
\end{frame}

\begin{frame}{Favoritism in Performance Evaluations}

\begin{wideitemize}
    \item It is typically supervisors, not firms, that measure output via performance evaluations.
     \item But it is is costly for supervisors to give employees they like bad performance evaluations.
     \item And the cost grows when more is on the line: that is when $\beta$ is higher!
     \item Therefore stronger incentives (higher $\beta$) makes supervisors less truthful about employee performance.
    \item Example: If you and I are good friends at work, I want you to make more money.
\end{wideitemize}



\small Source ``Uncertainty and Incentives" (Prendergast 2002)

    
\end{frame}


\begin{frame}{Favoritism in Performance Evaluations}

\begin{wideitemize}
    \item But suppose a firm (not the supervisor) uses performance evaluations for two things:
    \begin{wideitemize}
        \item[1.] To encourage effort (as in our model).
        \item[2.] To allocate the worker to the right job (not in our model).
    \end{wideitemize}
\end{wideitemize}


    
\end{frame}


\begin{frame}{Favoritism in Performance Evaluations}

\begin{wideitemize}
    \item How does favoritism impact goal 1 (encouraging effort)?
    \item Suppose a supervisor's favoritism for a person is given by $K$ that just adds to their reported output: $\tilde y = K+ y=K+e+\epsilon$.
    \item When $K>0$ there is biased towards the person, when $K<0$ they are biased against the person.
    \item The supervisor and the worker know $K$ exactly.
    \item From the firm's perspective the supervisor's favoritism for a person is given by a random variable $K$ that just adds to their reported output: $\tilde y = K+ y=K+e+\epsilon$.
    \item Suppose on average favoritism is unbiased ($E[K]=0$) and independent of everything else.
\end{wideitemize}

    
\end{frame}

\begin{frame}{Favoritism in Performance Evaluations}
    \begin{wideitemize}
    \item Then for fixed effort and wages, the worker will receive:
    \[w(\tilde y) = \alpha + \beta \tilde y = \alpha + \beta (K+e+\epsilon )\]
    \item From the perspective of the worker, favoritism is known and just shifts utility by a constant:
    \[E[w(\tilde y)] = E[\alpha + \beta (K+e)] + E[\beta \epsilon ] = \alpha + \beta e =\beta K \]
    \[Var(w(\tilde y)) = Var(\alpha + \beta (K+e)) + Var(\beta \epsilon )= 0+\beta^2 \sigma^2\]
    \item From the perspective of the firm, profit is:
    \[\pi = E[y - w(\tilde y)] = E[e +\epsilon - \alpha - \beta (K+e+\epsilon ) ] = e-\beta e-\alpha \]
    where $K$ has dropped out entirely because the firm is risk neutral.
    \item Therefore the firm can easily account for favoritism in the wage.
    \end{wideitemize}
\end{frame}


\begin{frame}{Favoritism in Performance Evaluations}

\begin{wideitemize}
      \item But when performance evaluations are used for goal 2 (allocation) there are big problems.
      \item When a supervisor reports a biased evaluation $\tilde y \neq y$, the firm cares about whether bias is positive or negative.
      \item Otherwise they may accidentally assign the worker to a task they are actually bad at doing!
      \item As $\beta$ rises, $\tilde y$ becomes more biased.
      \item Then $\tilde y$ becomes less useful for allocating talent.
        \item This is a new tension between encouraging effort and allocating people, and it is real (e.g. talent hoarding).  
\end{wideitemize}

   
\end{frame}

\begin{frame}{Favoritism in Performance Evaluations}
    \begin{wideitemize}
        \item How does this impact the risk/incentive trade-off?
        \pause
       \item In this context, we can think of $\sigma^2$ as measuring how much output reflects true talent.
     \item The higher $\sigma^2$, the less output tells us about talent.
     \item But if $\sigma^2$ is high, even if there is no bias ($\tilde y = y$) performance evaluations are useless for allocating talent!
     \item Therefore goal 2 does not matter, and we have a potential positive correlation between risk and incentives!
    \end{wideitemize}
\end{frame}

\begin{frame}{Deciding When to Investigate}

\begin{wideitemize}
     \item In our model, we assumed output is always known.
     \item In reality, output is only sometimes monitored.
     \item Further, it is monitored more often when people suspect slacking/shirking/cheating.
     \item Consider a model where a supervisors chooses whether to launch an investigation.
     \item If no investigation, the worker gets a wage equal to expected output (so a flat salary)
     \item If investigated, they get actual output.
\end{wideitemize}



\small Source ``Uncertainty and Incentives" (Prendergast 2002)

    
\end{frame}


\begin{frame}{Deciding When to Investigate}

\begin{wideitemize}
     \item When deciding to investigate, the supervisor gets a signal or impression of output.
     \item There is some cost to investigation (laying it out here is beyond the scope of this class)
     \item The supervisor investigates if the expected benefit of doing so is great enough.
     \item In this setting, greater performance pay ($\beta$) is needed to encourage effort when noise is greater, because we also need to encourage investigations.
     \item Intuitively, noise makes workers think they can get away with it.
     \item this also generates a positive link between risk and incentives.
\end{wideitemize}



\small Source ``Uncertainty and Incentives" (Prendergast 2002)

    
\end{frame}


% https://www.journals.uchicago.edu/doi/full/10.1086/341874

% https://www.journals.uchicago.edu/doi/full/10.1086/338676

% https://pubs.aeaweb.org/doi/pdf/10.1257/aer.90.2.421


% \section{(Potential) Resolutions of the Controversy}


% \begin{frame}{Command and Control}    
% \begin{wideitemize}
%     \item What can a firm do instead of performance pay to encourage a worker?
%     \pause
%     \item Tell the worker exactly what to do, and monitor them to make sure they do it.
%     \item This itself does not go against the risk-incentive trade-off.
%     \item However, command and control only works if the firm knows what the agent should be doing.
%     \item In other words, the main alternative to performance pay works best when there is little uncertainty.
%     \item This implies we should expect MORE performance pay in uncertain settings.
% \end{wideitemize}
% \end{frame}

% \begin{frame}{Choosing to Investigate}

% \begin{wideitemize}
%     \item We assumed that performance/output is always monitored by the firm.
%     \item In reality, monitoring is less frequent and usually triggered by something.
%     \item Suppose the firm sometimes knows nothing about $y$ and sometimes gets some ``signal" or impression or ``vibe" about $y$.
%     \item Importantly vibes or signals or impressions are not contractible or verifiable, so the firm cannot use them to write a compensation contract.
%     \item The firm can decide to investigate using the signal, and use the results of the investigation to alter pay.
% \end{wideitemize}
    
% \end{frame}


% \begin{frame}{Choosing to Investigate}

% \begin{wideitemize}
%     \item In more uncertain environments (higher $\sigma^2$) the worker knows they can get away with more.
%     \item So the agent 
% \end{wideitemize}
    
% \end{frame}


%  \begin{frame}{Biased Performance Evaluations}
% https://www.journals.uchicago.edu/doi/abs/10.1086/338676?casa_token=B3qWZ0JxOjEAAAAA%3ADG6U2QppgBTwOuK-1PtHU6vVUdlIzpxPekyAxv5FvjcMNJM8ik-rbM3hL8XADpPRpKoIep3mUX8&journalCode=jole
%  \begin{wideitemize}
    
% \end{wideitemize}
    
% \end{frame}


%  \begin{frame}{Bunched Performance Evaluations}
% https://www.journals.uchicago.edu/doi/abs/10.1086/338676?casa_token=B3qWZ0JxOjEAAAAA%3ADG6U2QppgBTwOuK-1PtHU6vVUdlIzpxPekyAxv5FvjcMNJM8ik-rbM3hL8XADpPRpKoIep3mUX8&journalCode=jole
%  \begin{wideitemize}
    
% \end{wideitemize}
    
% \end{frame}






\end{document}







