%%%%%%%%%%%%%%%%%%%%%%%%%%%%%%%%%%%%%%%%%
% Beamer Presentation
% LaTeX Template
% Version 1.0 (10/11/12)
%
% This template has been downloaded from:
% http://www.LaTeXTemplates.com
%
% License:
% CC BY-NC-SA 3.0 (http://creativecommons.org/licenses/by-nc-sa/3.0/)
%
%%%%%%%%%%%%%%%%%%%%%%%%%%%%%%%%%%%%%%%%%

%----------------------------------------------------------------------------------------
%	PACKAGES AND THEMES
%----------------------------------------------------------------------------------------

\documentclass[aspectratio=169,usenames,dvipsnames]{beamer}

\usepackage[utf8]{inputenc}
\usepackage{booktabs}
\usepackage{tabularx}
\usepackage[authordate,bibencoding=auto,strict,backend=biber,natbib]{biblatex-chicago}
\addbibresource{bib.bib}
\usepackage{graphicx}
% \hypersetup{
%     colorlinks,
%     %citecolor=black,
%     linkcolor=black
% }
\usepackage{array}
\usepackage{caption}
\usepackage{threeparttable}
\usepackage{epigraph} 
\usepackage{lscape}
\usepackage{adjustbox}
\newcommand*{\Scale}[2][4]{\scalebox{#1}{\ensuremath{#2}}}%
\usepackage{import}
\newenvironment{wideitemize}{\itemize\addtolength{\itemsep}{10pt}}{\enditemize}
\usepackage{amsmath}
\usepackage{csvsimple}
\usepackage{siunitx}
\usepackage{filecontents}
\usepackage{rotating}
\usepackage{multirow}
\usepackage{amsmath}
\usepackage{subcaption}
\usepackage{appendixnumberbeamer}
\usepackage{float}
\usepackage{amsmath}
\usepackage{csvsimple}
\newtheorem{proposition}{Proposition}
\usepackage{xcolor}
\def\boxit#1#2{%
    \smash{\color{red}\fboxrule=1pt\relax\fboxsep=2pt\relax%
    \llap{\rlap{\fbox{\phantom{\rule{#1}{#2}}}}~}}\ignorespaces
}
\newenvironment{variableblock}[3]{%
  \setbeamercolor{block body}{#2}
  \setbeamercolor{block title}{#3}
  \begin{block}{#1}}{\end{block}}
\usepackage{appendixnumberbeamer}
\usepackage{tikz,pgfplots}
\usepackage{tkz-fct}
\usepackage{amsthm}
\pgfplotsset{compat=1.10}
\usepgfplotslibrary{fillbetween}
\mode<presentation> {
\AtBeginSection[]
{
    \begin{frame}
        \frametitle{Table of Contents}
        \tableofcontents[currentsection]
    \end{frame}
}
% The Beamer class comes with a number of default slide themes
% which change the colors and layouts of slides. Below this is a list
% of all the themes, uncomment each in turn to see what they look like.

\usetheme{default}
%\usetheme{AnnArbor}
%\usetheme{Antibes} -
%\usetheme{Bergen}
%\usetheme{Berkeley}
%\usetheme{Berlin}
%\usetheme{Boadilla}
%\usetheme{CambridgeUS}
%\usetheme{Copenhagen} -
%\usetheme{Darmstadt}
%\usetheme{Dresden}
%\usetheme{Frankfurt}
%\usetheme{Goettingen}
%\usetheme{Hannover}
%\usetheme{Ilmenau}
%\usetheme{JuanLesPins}
%\usetheme{Luebeck}
%\usetheme{Madrid}
%\usetheme{Malmoe}
%\usetheme{Marburg}
%\usetheme{Montpellier}
%\usetheme{PaloAlto}
%\usetheme{Pittsburgh}
%\usetheme{Rochester} -
%\usetheme{Singapore}
%\usetheme{Szeged}
%\usetheme{Warsaw}

% As well as themes, the Beamer class has a number of color themes
% for any slide theme. Uncomment each of these in turn to see how it
% changes the colors of your current slide theme.

%\usecolortheme{albatross}
%\usecolortheme{beaver}
%\usecolortheme{beetle}
%\usecolortheme{crane}
%\usecolortheme{dolphin}
%\usecolortheme{dove}
%\usecolortheme{fly}
%\usecolortheme{lily}
%\usecolortheme{orchid}
%\usecolortheme{rose}
%\usecolortheme{seagull}
%\usecolortheme{seahorse}
%\usecolortheme{whale}
%\usecolortheme{wolverine}

%\setbeamertemplate{footline} % To remove the footer line in all slides uncomment this line
%\setbeamertemplate{footline}[frame number] % To replace the footer line in all slides with a simple slide count uncomment this line
\setbeamertemplate{theorems}[numbered]
\setbeamertemplate{navigation symbols}{} % To remove the navigation symbols from the bottom of all slides uncomment this line
}
\setbeamertemplate{caption}{\raggedright\insertcaption\par}
  \setbeamertemplate{enumerate items}[default]
  %\setbeamertemplate{page number in head/foot}{\insertframenumber}
\usepackage{graphicx} % Allows including images
\usepackage{booktabs} % Allows the use of \toprule, \midrule and \bottomrule in tables
%\usepackage {tikz}
\newtheorem*{theorem*}{Theorem}
\newtheorem*{lemma*}{Lemma}
\newtheorem*{proposition*}{Proposition}
\newtheorem*{corollary*}{Corollary}
\newtheorem*{definition*}{Definition}
\DeclareMathOperator*{\argmin}{arg\,min}
\newtheorem*{assumption}{Assumption}
\usetikzlibrary {positioning}
\renewcommand{\arraystretch}{1.5}
\newcommand\hideit[1]{%
  \only<0| handout:1>{\mbox{}}%
  \invisible<0| handout:1>{#1}}
\usepackage[default]{lato}

\setbeamercolor{block body alerted}{bg=alerted text.fg!10}
\setbeamercolor{block title alerted}{bg=alerted text.fg!20}
\setbeamercolor{block body}{bg=structure!10}
\setbeamercolor{block title}{bg=structure!20}
\setbeamercolor{block body example}{bg=green!10}
\setbeamercolor{block title example}{bg=green!20}


\makeatletter
\let\save@measuring@true\measuring@true
\def\measuring@true{%
  \save@measuring@true
  \def\beamer@sortzero##1{\beamer@ifnextcharospec{\beamer@sortzeroread{##1}}{}}%
  \def\beamer@sortzeroread##1<##2>{}%
  \def\beamer@finalnospec{}%
}
\makeatother

%\usepackage {xcolor}

%----------------------------------------------------------------------------------------
%	TITLE PAGE
%----------------------------------------------------------------------------------------

\title[diss]{Lecture 1: Introduction} % The short title appears at the bottom of every slide, the full title is only on the title page
\author{Econ 490: Compensation in Organizations} % Your name
\institute[shortinst]{Jacob Kohlhepp}
\date{\today} % Date, can be changed to a custom date

\begin{document}

\begin{frame}
\titlepage % Print the title page as the first slide

\end{frame}


\begin{frame}
\centering
    \huge Why does compensation matter?
\end{frame}

\begin{frame}{How can this class help you?}

\begin{wideitemize}
    \item This class will teach you a critical way to think through business problems.
    \item This class will teach you to digest empirical research.
    \item This class will help you think through your own compensation in the future.
    \item This class will make you think about why organizations are the way they are.
    \item Compensation, Benefits, and Job Analysis Specialists is a growing field (U.S. Bureau of Labor Statistics)
\end{wideitemize}

\end{frame}

\begin{frame}
\centering
    \huge Why mathematical models?
\end{frame}


\begin{frame}{Why Mathematical Models: The Monty Hall Problem}
\begin{wideitemize}
    \item The problem you just observed is equivalent to \textit{Let's Make a Deal?} or a simplified version of \textit{Deal or No Deal.}
    \item It was posed to Marilyn vos Savant in the magazine \textit{Parade} in 1990.
    \item vos Savant answered: you should always switch.
    \begin{wideitemize}
        \item 10,000 reader responses claimed this was incorrect (Krauss and Wang 2003)
        \item Some respondents had PhDs in statistics.
         \item In an experiment, only 16\% of people chose correctly (Krauss and Wang 2003).

    \end{wideitemize}
    \item Most people's intuition breaks down in the face of even moderately complex social science problems.
    \begin{wideitemize}
        \item But a simple game theoretic model reveals the correct answer!
    \end{wideitemize}
        

\end{wideitemize}

\end{frame}

\begin{frame}[c]{Three Facts About Me}
\centering
      \begin{columns}
          \column{0.45\linewidth}
            \begin{wideitemize}
                \item[1.] I am a Christian.
                \item[2.] My wife and I have two daughters.
                \item[3.] I lived in California most of my life.
            \end{wideitemize}
           \column{0.54\linewidth}
                           \centering
    \includegraphics[width=0.95\textwidth]{pictures/family_pic.jpg}
    \end{columns} 
         
\end{frame}
\begin{frame}{My Career in Economics}
    \begin{wideitemize}
        \item 2012-2016: UCLA Undergraduate
        \begin{wideitemize}
            \item BA in Economics, BA in Political Science
            \item Worked in policy, government, and politics during summers
        \end{wideitemize}
        \item 2016-2018: Associate at Welch Consulting (acquired by Charles River)
        \begin{wideitemize}
            \item economics labor litigation consulting
            \item pay equity, wage and hour violations, single plaintiff cases, etc.
        \end{wideitemize}
        \item 2017: Got married!
        \item Early 2018: Almost Joined Activision HR
        \begin{wideitemize}
            \item Compensation analyst managing bonus program for game studios
        \end{wideitemize}
    \end{wideitemize}
\end{frame}


\begin{frame}{My Career in Economics}
    \begin{wideitemize}
        \item 2018-2023: PhD in Economics
        \begin{wideitemize}
            \item Earned a master's in econ partway through.
            \item My first daughter was born in 2021!
            \item Started and closed a very small nonprofit.
            \item Worked as a freelance consultant
        \end{wideitemize}
        \item 2023: Started at UNC.
        \begin{wideitemize}
            \item Tenure-track assistant professor (after 6 years, up or out)
            \item My research focuses on job design, recruitment, and overtime assignment within organizations and what this means for the broader economy.
            \item My second daughter was born in November 2023!
        \end{wideitemize}
    \end{wideitemize}
\end{frame}

\section{Logistics}

\begin{frame}
\centering
    \huge Syllabus Review
\end{frame}

\begin{frame}{Syllabus Highlights}
\begin{wideitemize}
    \item Checkpoint
    \item Two Grading Schemes
    \item Participation
\end{wideitemize}
\end{frame}

\begin{frame}{Problem Sets}

\begin{wideitemize}
    \item Because of the grading scheme you can opt out.
    \item However they are meant to help you study for the test.
    \item I suggest working alone first then working together.
    \item You must submit individual answers.
\end{wideitemize}
    
\end{frame}



\begin{frame}{Game Theory/Risk Aversion Tools}
    \centering
    \huge We will discuss this at length in the next 1-2 lectures.
\end{frame}

\begin{frame}{Mathematical Readiness}
    \centering
    \huge See the board.
\end{frame}

\begin{frame}{Board Work}

\begin{wideitemize}
    \item In many lectures, I will switch to the board to solve models.
    \item I will post typed up versions of the ``board work" on the course website.
    \item It is still important to work through it in class.
    \item The typed up versions should help you fill in gaps that you miss in lecture, but they may not be comprehensive!
\end{wideitemize}

\end{frame}

\section{Reading Peer Reviewed Articles}

\begin{frame}{Readings}
    \begin{wideitemize}
        \item Most lectures have one or two assigned articles.
        \item Most are peer-reviewed research articles. We will discuss how to read these next.
        \item I will ask a few questions on the midterm and final about the readings.
        \item I will also randomly ask a few students each lecture questions about the readings (eahc person will be called a few times during the semester).
        \item As long as your answer reflects some engagement with the reading (even just a little) you will get full credit.
        \item If you are absent for an unapproved reason you get 0 credit (submit absences to the logistics Google Form).
        \item If you wish to pass (for 0 credit) just say pass or do not answer. 
    \end{wideitemize}
    
\end{frame}

\begin{frame}{What is Peer Review?}

\begin{wideitemize}
    \item Peer review is the independent evaluation of research by experts in a field.
    \begin{wideitemize}
        \item Author submits paper to a journal often using feedback from presentations/prior rejections.
        \item Editor quickly reviews paper and either desk rejects or sends it for review.
        \item Editor selects 2-4 referees: experts in the area.
        \item Experts read the paper and write often detailed reports about the flaws and merits of the paper.
        \item Experts usually make a recommendation: reject, revise and resubmit, accept.
        \item Editor, based on expert reports, makes decision. if revise, repeat all steps until reject or accept.
    \end{wideitemize}
    \item If a paper passes peer review at a journal, it is published.
\end{wideitemize}
    
\end{frame}


\begin{frame}{Journal Quality}

    \begin{wideitemize}
        \item Not all journals have the same level of rigor or quality
        \item Number of citations and impact factors can help you understand what people are reading.
        \item But just because a journal or an article is cited often does not mean it is reliable/rigorous.
        \item Journal quality is subjective, but 5 journals have the best ``reputations" in economics:
        \begin{wideitemize}
            \item \textit{American Economic Review},\textit{Econometrica}, \textit{Quarterly Journal of Economics},\textit{Journal of Political Economy}, \textit{Review of Economic Studies} 
        \end{wideitemize}
        \item Below these 5, there are reputable journals in each subfield. Relevant ones for our class:
        \begin{wideitemize}
            \item \textit{Journal of Labor Economics}, \textit{Journal of Economic Theory}, \textit{AEJ: Applied}
        \end{wideitemize}
    \end{wideitemize}
\end{frame}

\begin{frame}{Reading an Academic Article}

\begin{wideitemize}
    \item You do not need to read each line to get main idea.
    \item Articles are organized so you can go to the section you want to understand.
    \item If we want to understand the theory, we can read that section.
    \item If we want empirical results, we can go there.
    \item The introduction and abstract give us the high-level findings and main approach.
\end{wideitemize}
\end{frame}

\begin{frame}
\centering
    \huge Practice with ``Performance Pay and Productivity" (Lazear AER 2000)
\end{frame}


\end{document}







