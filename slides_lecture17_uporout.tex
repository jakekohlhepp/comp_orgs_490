%%%%%%%%%%%%%%%%%%%%%%%%%%%%%%%%%%%%%%%%%
% Beamer Presentation
% LaTeX Template
% Version 1.0 (10/11/12)
%
% This template has been downloaded from:
% http://www.LaTeXTemplates.com
%
% License:
% CC BY-NC-SA 3.0 (http://creativecommons.org/licenses/by-nc-sa/3.0/)
%
%%%%%%%%%%%%%%%%%%%%%%%%%%%%%%%%%%%%%%%%%

%----------------------------------------------------------------------------------------
%	PACKAGES AND THEMES
%----------------------------------------------------------------------------------------

\documentclass[aspectratio=169,usenames,dvipsnames]{beamer}

\usepackage[utf8]{inputenc}
\usepackage{booktabs}
\usepackage{tabularx}
\usepackage[authordate,bibencoding=auto,strict,backend=biber,natbib]{biblatex-chicago}
\addbibresource{bib.bib}
\usepackage{graphicx}
% \hypersetup{
%     colorlinks,
%     %citecolor=black,
%     linkcolor=black
% }
\usepackage{array}
\usepackage{caption}
\usepackage{threeparttable}
\usepackage{epigraph} 
\usepackage{lscape}
\usepackage{adjustbox}
\newcommand*{\Scale}[2][4]{\scalebox{#1}{\ensuremath{#2}}}%
\usepackage{import}
\newenvironment{wideitemize}{\itemize\addtolength{\itemsep}{10pt}}{\enditemize}
\usepackage{amsmath}
\usepackage{csvsimple}
\usepackage{siunitx}
\usepackage{filecontents}
\usepackage{rotating}
\usepackage{multirow}
\usepackage{amsmath}
\usepackage{subcaption}
\usepackage{appendixnumberbeamer}
\usepackage{float}
\usepackage{amsmath}
\usepackage{csvsimple}
\usepackage{hyperref}
\newtheorem{proposition}{Proposition}
\usepackage{xcolor}
\def\boxit#1#2{%
    \smash{\color{red}\fboxrule=1pt\relax\fboxsep=2pt\relax%
    \llap{\rlap{\fbox{\phantom{\rule{#1}{#2}}}}~}}\ignorespaces
}
\newenvironment{variableblock}[3]{%
  \setbeamercolor{block body}{#2}
  \setbeamercolor{block title}{#3}
  \begin{block}{#1}}{\end{block}}
\usepackage{appendixnumberbeamer}
\usepackage{tikz,pgfplots}
\usepackage{tkz-fct}
\usepackage{amsthm}
\pgfplotsset{compat=1.10}
\usepgfplotslibrary{fillbetween}
\mode<presentation> {
\AtBeginSection[]
{
    \begin{frame}
        \frametitle{Table of Contents}
        \tableofcontents[currentsection]
    \end{frame}
}
% The Beamer class comes with a number of default slide themes
% which change the colors and layouts of slides. Below this is a list
% of all the themes, uncomment each in turn to see what they look like.

\usetheme{default}
%\usetheme{AnnArbor}
%\usetheme{Antibes} -
%\usetheme{Bergen}
%\usetheme{Berkeley}
%\usetheme{Berlin}
%\usetheme{Boadilla}
%\usetheme{CambridgeUS}
%\usetheme{Copenhagen} -
%\usetheme{Darmstadt}
%\usetheme{Dresden}
%\usetheme{Frankfurt}
%\usetheme{Goettingen}
%\usetheme{Hannover}
%\usetheme{Ilmenau}
%\usetheme{JuanLesPins}
%\usetheme{Luebeck}
%\usetheme{Madrid}
%\usetheme{Malmoe}
%\usetheme{Marburg}
%\usetheme{Montpellier}
%\usetheme{PaloAlto}
%\usetheme{Pittsburgh}
%\usetheme{Rochester} -
%\usetheme{Singapore}
%\usetheme{Szeged}
%\usetheme{Warsaw}

% As well as themes, the Beamer class has a number of color themes
% for any slide theme. Uncomment each of these in turn to see how it
% changes the colors of your current slide theme.

%\usecolortheme{albatross}
%\usecolortheme{beaver}
%\usecolortheme{beetle}
%\usecolortheme{crane}
%\usecolortheme{dolphin}
%\usecolortheme{dove}
%\usecolortheme{fly}
%\usecolortheme{lily}
%\usecolortheme{orchid}
%\usecolortheme{rose}
%\usecolortheme{seagull}
%\usecolortheme{seahorse}
%\usecolortheme{whale}
%\usecolortheme{wolverine}

%\setbeamertemplate{footline} % To remove the footer line in all slides uncomment this line
%\setbeamertemplate{footline}[frame number] % To replace the footer line in all slides with a simple slide count uncomment this line
\setbeamertemplate{theorems}[numbered]
\setbeamertemplate{navigation symbols}{} % To remove the navigation symbols from the bottom of all slides uncomment this line
}
\setbeamertemplate{caption}{\raggedright\insertcaption\par}
  \setbeamertemplate{enumerate items}[default]
  %\setbeamertemplate{page number in head/foot}{\insertframenumber}
\usepackage{graphicx} % Allows including images
\usepackage{booktabs} % Allows the use of \toprule, \midrule and \bottomrule in tables
%\usepackage {tikz}
\newtheorem*{theorem*}{Theorem}
\newtheorem*{lemma*}{Lemma}
\newtheorem*{proposition*}{Proposition}
\newtheorem*{corollary*}{Corollary}
\newtheorem*{definition*}{Definition}
\DeclareMathOperator*{\argmin}{arg\,min}
\newtheorem*{assumption}{Assumption}
\usetikzlibrary {positioning}
\renewcommand{\arraystretch}{1.5}
\newcommand\hideit[1]{%
  \only<0| handout:1>{\mbox{}}%
  \invisible<0| handout:1>{#1}}
\usepackage[default]{lato}

\setbeamercolor{block body alerted}{bg=alerted text.fg!10}
\setbeamercolor{block title alerted}{bg=alerted text.fg!20}
\setbeamercolor{block body}{bg=structure!10}
\setbeamercolor{block title}{bg=structure!20}
\setbeamercolor{block body example}{bg=green!10}
\setbeamercolor{block title example}{bg=green!20}


\makeatletter
\let\save@measuring@true\measuring@true
\def\measuring@true{%
  \save@measuring@true
  \def\beamer@sortzero##1{\beamer@ifnextcharospec{\beamer@sortzeroread{##1}}{}}%
  \def\beamer@sortzeroread##1<##2>{}%
  \def\beamer@finalnospec{}%
}
\makeatother
%\usepackage {xcolor}

%----------------------------------------------------------------------------------------
%	TITLE PAGE
%----------------------------------------------------------------------------------------

\title[diss]{Lecture 17: Up or Out} % The short title appears at the bottom of every slide, the full title is only on the title page
\author{Compensation in Organizations} % Your name
\institute[shortinst]{Jacob Kohlhepp}
\date{\today} % Date, can be changed to a custom date

\begin{document}

\begin{frame}
\titlepage % Print the title page as the first slide

\end{frame}


\begin{frame}
\centering
    \huge Discussion: MacLeod and Urquiola (2021)

\end{frame}


\begin{frame}
\begin{definition}
    An up or out contract consists of an initial period of employment followed by an evaluation after which the worker is either promoted or terminated.
\end{definition}
\end{frame}

\begin{frame}
\centering
    \huge What are some occupations where up or out is used?

\end{frame}

\begin{frame}{Up or Out in the Wild}
\begin{wideitemize}
    \item Law firms: Cravath system says you must achieve partner in 10 years or leave.
    \item Management consulting: Bain, BCG, Mckinsey 
    \item Military (until recently): discharged if passed over for rank promotion twice.
    \item Universities: tenure-track assistant professors
\end{wideitemize}
    
\end{frame}
\begin{frame}{Tenure as Up or Out}

\begin{wideitemize}
    \item Tenure is a unique form of up or out.
    \item If you receive tenure, you receive some form of additional job security.
    \item This can be viewed as a form of compensation!
    \item Recall that if people are risk averse, lowering the variance of pay improves their utility.
\end{wideitemize}
    
\end{frame}

\begin{frame}{Tenure at Research Universities}
    \begin{wideitemize}
        \item Six year period as a tenure-track assistant professor
        \item During this period, the AP does research and teaches
        \item At the end, they submit a packet documenting their accomplishments
        \item Publication record is usually the main metric of success
        \item The department votes, the college and dean also sign off on tenure decision
        \item If given tenure, promoted to associate professor and typically not subject to termination except in extreme circumstances
        \item If not, AP is terminated (all or nothing).
    \end{wideitemize}
\end{frame}


\begin{frame}
\centering
    \huge What are some benefits of tenure in the academic setting?
\end{frame}

\begin{frame}{Academic Freedom}
\begin{wideitemize}
    \item After tenure, tenure may promote academic freedom.
    \item For example, a professor can study a topic or provide an answer that is controversial.
    \item But before tenure, the need to get department support could weaken academic freedom.
    \item For example, a professor may only study topics which interest a large number of colleagues.
    \item There are many professions where truth is important, but where up or out/tenure is not used.
\end{wideitemize}
    
\end{frame}

\begin{frame}{Riskier Projects}
\begin{wideitemize}
    \item Brogaard, Engelberg, and Van Wesep (2018) study economists after tenure.
    \item They find total publications and ``home run publications" fall.
    \item No evidence that economists try for riskier projects after tenure is granted.
\end{wideitemize}
    
\end{frame}

\begin{frame}{Selecting and Sorting}

\begin{wideitemize}
    \item Industry pays (much) better for some academic disciplines at the PhD level.
    \item For example, consulting offers for econ PhDs are often more than 2x an assistant professor salary
    \item Tenure allows universities to compete for talent without higher salaries
    \item It also sorts people: up or out is only attractive if you think you have a chance at moving up!
\end{wideitemize}
    
\end{frame}

%https://pubs.aeaweb.org/doi/pdfplus/10.1257/jep.32.1.179

\begin{frame}
\centering
    \huge What are some costs of tenure in the academic setting?
\end{frame}

\begin{frame}{High to Low Powered Incentives}

\begin{wideitemize}
    \item Prior to tenure, professors face high powered incentives.
    \item Produce research or lose your job!
    \item But after, they face low powered incentives.
    \item So there is incentive to reduce effort after tenure.
\end{wideitemize}
    
\end{frame}


\begin{frame}{Why Up or Out?}

    \begin{wideitemize}
        \item We will review two prominent theories for why up or out contracts exist.
        \item Both have to do with encouraging the worker to work hard/invest:
        \begin{wideitemize}
            \item[1.] Kahn and Huberman (1988): firm-specific human capital
            \item[2.] Waldman (1990): general human capital, known only to the firm
        \end{wideitemize}
        \item We will only verbally discuss the logic of these theories.
        \item You are not responsible for understanding the derivations or the model setup.
    \end{wideitemize}
\end{frame}


\begin{frame}{Kahn and Huberman (1988): firm-specific human capital}

\begin{wideitemize}
    \item Suppose there is a worker who can invest (at a cost) to increase their productivity at a particular firm only.
    \item This increase in productivity is random: sometimes investing increases productivity sometimes not.
    \item Further, suppose this investment is observed only by the worker, and output only by the firm.
    \item How can the firm get the worker to invest?
    \item Problem: the investment does not increase the worker's market wage.
    \item Discussion: Why?
\end{wideitemize}
    
\end{frame}

\begin{frame}{Kahn and Huberman (1988): firm-specific human capital}

\begin{wideitemize}
    \item One option: the firm can offer a high wage for high productivity.
    \item But output is not seen by a court.
    \item So when it comes time to pay, the firm can just claim the worker had low output.
    \item The worker knows this, and so will not invest!
    \item There are two key issues:
    \begin{wideitemize}
        \item[1.] The benefits of training go to the firm, and the costs are on the worker.
        \item[2.] The firm paying the worker to invest/work hard is not credible.
    \end{wideitemize}
\end{wideitemize}
    
\end{frame}

\begin{frame}{Kahn and Huberman (1988)}

\begin{wideitemize}
    \item One solution is an up or out contract.
    \item If output is high ``promote" and pay a higher wage.
    \item If output is low, terminate.
    \item The firm does not want to claim output is low when it is not.
    \item This is because it has to then fire the employee, and get 0.
    \item So up or out compensates the worker for hard work, and is credible for the firm!
\end{wideitemize}
    
\end{frame}


\begin{frame}{General vs. Firm-Specific Human Capital}
    \begin{wideitemize}
        \item In Kahn and Huberman (1988), investing in human capital only improves productivity at one firm.
        \item This is called firm-specific human capital.
        \item Because the worker incurs the cost and does not directly benefit this causes problems.
        \item But many jobs do not have firm-specific human capital.
        \item Example: most professors have discipline-specific knowledge that can be used at any university.
    \end{wideitemize}
\end{frame}


\begin{frame}{General vs. Firm-Specific Human Capital}
    \begin{wideitemize}
        \item Skills that can be used anywhere are called general human capital.
        \item They raise productivity at any workplace.
        \item As a consequence they raise a workers wage.
        \item In this case the worker bares the cost but gets the benefit.
        \item Issue: we still see up or out among professions with only general human capital!
    \end{wideitemize}
\end{frame}

\begin{frame}{Waldman (1990): Result}
\begin{wideitemize}
    \item Keep almost everything the same as Kahn and Huberman (1988).
    \item Except: now investment increases productivity at many firms.
    \item Allow firms to make offers and counteroffers to the worker.
    \item Result: firms offer up or out contracts that get workers to invest.
    \item This is surprising: firms pay workers to invest even though their skills can be stolen by other firms!
\end{wideitemize}
    
\end{frame}

\begin{frame}{Waldman (1990): The Logic}
\begin{wideitemize}
    \item Investment either increases productivity or does not.
    \item The firm currently employing the worker observes which case happens.
    \item The firm retains if productivity is high (up) and terminates if not (out).
    \item Other firms do not see productivity.
    \item But they do see whether the worker is retained or terminated.
    \item And they know the termination decision was made with full knowledge of the worker's productivity!
\end{wideitemize}
    
\end{frame}


\begin{frame}{Waldman (1990): Retention As a Signal}
\begin{wideitemize}
    \item When other firms see a worker is retained, they know that productivity is high.
    \item They try to lure the worker away with high wage offers.
    \item To keep the worker, the original firm must counteroffer.
    \item The counteroffer must be equal to the worker's high productivity (why?)
\end{wideitemize}
    
\end{frame}



\begin{frame}{Waldman (1990): Retention As a Signal}
\begin{wideitemize}
    \item When other firms see a worker is terminated, they know that productivity is low.
    \item They make offers to the worker.
    \item The equilibrium wage offer is exactly the worker's low productivity (low wage).
    \item The original firm is not tempted to try to hire the worker back because the wage is equal to productivity.
\end{wideitemize}
    
\end{frame}

\begin{frame}{Waldman (1990): Incentives to Invest/Work Hard}
\begin{wideitemize}
    \item The worker invests in their human capital (or works hard) initially.
    \item They do this for the chance of being retained (up) rather than being terminated (out)
    \item The fact that the firm terminates the worker makes up or out contracts credible.
\end{wideitemize}
    
\end{frame}

\begin{frame}{Waldman (1990) and Academics}

\begin{wideitemize}
    \item The model we verbally walked through explains why up or out might be used even with general human capital.
    \item This gives a reason why up or out is used for academics and lawyers who often have general human capital.
    \item It also explains why tenure does not usually come with a large pay increase.
    \item Instead, professors often go out and get counteroffers from other universities even when they make tenure!
    \item Salary then increases to match these counteroffers.
    \item This is especially true at state universities (why?).
\end{wideitemize}
    
\end{frame}

\end{document}




