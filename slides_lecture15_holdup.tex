%%%%%%%%%%%%%%%%%%%%%%%%%%%%%%%%%%%%%%%%%
% Beamer Presentation
% LaTeX Template
% Version 1.0 (10/11/12)
%
% This template has been downloaded from:
% http://www.LaTeXTemplates.com
%
% License:
% CC BY-NC-SA 3.0 (http://creativecommons.org/licenses/by-nc-sa/3.0/)
%
%%%%%%%%%%%%%%%%%%%%%%%%%%%%%%%%%%%%%%%%%

%----------------------------------------------------------------------------------------
%	PACKAGES AND THEMES
%----------------------------------------------------------------------------------------

\documentclass[aspectratio=169,usenames,dvipsnames]{beamer}

\usepackage[utf8]{inputenc}
\usepackage{booktabs}
\usepackage{tabularx}
\usepackage[authordate,bibencoding=auto,strict,backend=biber,natbib]{biblatex-chicago}
\addbibresource{bib.bib}
\usepackage{graphicx}
% \hypersetup{
%     colorlinks,
%     %citecolor=black,
%     linkcolor=black
% }
\usepackage{array}
\usepackage{caption}
\usepackage{threeparttable}
\usepackage{epigraph} 
\usepackage{lscape}
\usepackage{adjustbox}
\newcommand*{\Scale}[2][4]{\scalebox{#1}{\ensuremath{#2}}}%
\usepackage{import}
\newenvironment{wideitemize}{\itemize\addtolength{\itemsep}{10pt}}{\enditemize}
\usepackage{amsmath}
\usepackage{csvsimple}
\usepackage{siunitx}
\usepackage{filecontents}
\usepackage{rotating}
\usepackage{multirow}
\usepackage{amsmath}
\usepackage{subcaption}
\usepackage{appendixnumberbeamer}
\usepackage{float}
\usepackage{amsmath}
\usepackage{csvsimple}
\usepackage{hyperref}
\newtheorem{proposition}{Proposition}
\usepackage{xcolor}
\def\boxit#1#2{%
    \smash{\color{red}\fboxrule=1pt\relax\fboxsep=2pt\relax%
    \llap{\rlap{\fbox{\phantom{\rule{#1}{#2}}}}~}}\ignorespaces
}
\newenvironment{variableblock}[3]{%
  \setbeamercolor{block body}{#2}
  \setbeamercolor{block title}{#3}
  \begin{block}{#1}}{\end{block}}
\usepackage{appendixnumberbeamer}
\usepackage{tikz,pgfplots}
\usepackage{tkz-fct}
\usepackage{amsthm}
\pgfplotsset{compat=1.10}
\usepgfplotslibrary{fillbetween}
\mode<presentation> {
\AtBeginSection[]
{
    \begin{frame}
        \frametitle{Table of Contents}
        \tableofcontents[currentsection]
    \end{frame}
}
% The Beamer class comes with a number of default slide themes
% which change the colors and layouts of slides. Below this is a list
% of all the themes, uncomment each in turn to see what they look like.

\usetheme{default}
%\usetheme{AnnArbor}
%\usetheme{Antibes} -
%\usetheme{Bergen}
%\usetheme{Berkeley}
%\usetheme{Berlin}
%\usetheme{Boadilla}
%\usetheme{CambridgeUS}
%\usetheme{Copenhagen} -
%\usetheme{Darmstadt}
%\usetheme{Dresden}
%\usetheme{Frankfurt}
%\usetheme{Goettingen}
%\usetheme{Hannover}
%\usetheme{Ilmenau}
%\usetheme{JuanLesPins}
%\usetheme{Luebeck}
%\usetheme{Madrid}
%\usetheme{Malmoe}
%\usetheme{Marburg}
%\usetheme{Montpellier}
%\usetheme{PaloAlto}
%\usetheme{Pittsburgh}
%\usetheme{Rochester} -
%\usetheme{Singapore}
%\usetheme{Szeged}
%\usetheme{Warsaw}

% As well as themes, the Beamer class has a number of color themes
% for any slide theme. Uncomment each of these in turn to see how it
% changes the colors of your current slide theme.

%\usecolortheme{albatross}
%\usecolortheme{beaver}
%\usecolortheme{beetle}
%\usecolortheme{crane}
%\usecolortheme{dolphin}
%\usecolortheme{dove}
%\usecolortheme{fly}
%\usecolortheme{lily}
%\usecolortheme{orchid}
%\usecolortheme{rose}
%\usecolortheme{seagull}
%\usecolortheme{seahorse}
%\usecolortheme{whale}
%\usecolortheme{wolverine}

%\setbeamertemplate{footline} % To remove the footer line in all slides uncomment this line
%\setbeamertemplate{footline}[frame number] % To replace the footer line in all slides with a simple slide count uncomment this line
\setbeamertemplate{theorems}[numbered]
\setbeamertemplate{navigation symbols}{} % To remove the navigation symbols from the bottom of all slides uncomment this line
}
\setbeamertemplate{caption}{\raggedright\insertcaption\par}
  \setbeamertemplate{enumerate items}[default]
  %\setbeamertemplate{page number in head/foot}{\insertframenumber}
\usepackage{graphicx} % Allows including images
\usepackage{booktabs} % Allows the use of \toprule, \midrule and \bottomrule in tables
%\usepackage {tikz}
\newtheorem*{theorem*}{Theorem}
\newtheorem*{lemma*}{Lemma}
\newtheorem*{proposition*}{Proposition}
\newtheorem*{corollary*}{Corollary}
\newtheorem*{definition*}{Definition}
\DeclareMathOperator*{\argmin}{arg\,min}
\newtheorem*{assumption}{Assumption}
\usetikzlibrary {positioning}
\renewcommand{\arraystretch}{1.5}
\newcommand\hideit[1]{%
  \only<0| handout:1>{\mbox{}}%
  \invisible<0| handout:1>{#1}}
\usepackage[default]{lato}

\setbeamercolor{block body alerted}{bg=alerted text.fg!10}
\setbeamercolor{block title alerted}{bg=alerted text.fg!20}
\setbeamercolor{block body}{bg=structure!10}
\setbeamercolor{block title}{bg=structure!20}
\setbeamercolor{block body example}{bg=green!10}
\setbeamercolor{block title example}{bg=green!20}


\makeatletter
\let\save@measuring@true\measuring@true
\def\measuring@true{%
  \save@measuring@true
  \def\beamer@sortzero##1{\beamer@ifnextcharospec{\beamer@sortzeroread{##1}}{}}%
  \def\beamer@sortzeroread##1<##2>{}%
  \def\beamer@finalnospec{}%
}
\makeatother
%\usepackage {xcolor}

%----------------------------------------------------------------------------------------
%	TITLE PAGE
%----------------------------------------------------------------------------------------

\title[diss]{Lecture 15: Hold-Up} % The short title appears at the bottom of every slide, the full title is only on the title page
\author{Compensation in Organizations} % Your name
\institute[shortinst]{Jacob Kohlhepp}
\date{\today} % Date, can be changed to a custom date

\begin{document}

\begin{frame}
\titlepage % Print the title page as the first slide

\end{frame}

\begin{frame}
\centering
    \huge Discussion: Klein, Crawford, Alchian (1978)

    %https://www.thethings.com/-elijah-wood-lord-of-the-rings-salary-underpaid/
\end{frame}


\begin{frame}
\centering
    \huge How much was Elijah Wood's base salary for all three LOTR movies?
    
    \pause
\large (Screen-time: 2 hours, 4 minutes)

\centering
\pause \huge \textcolor{red}{\$250,000}

    %https://www.thethings.com/-elijah-wood-lord-of-the-rings-salary-underpaid/
    %\
\end{frame}


\begin{frame}
\centering
    \huge How much was Elijah Wood paid for \textit{The Hobbit} trilogy?

    \pause
\large (Screen-time: less than 4 minutes)

\pause 
\huge \textcolor{red}{\$1,000,000}
    %https://www.thethings.com/-elijah-wood-lord-of-the-rings-salary-underpaid/
\end{frame}


\begin{frame}
\centering
    \huge Discussion: Why this difference?
\end{frame}

\begin{frame}{Human Asset Specificity}
\begin{wideitemize}
    \item \textit{The Fellowship of the Ring} (movie 1) was enormously successful.
    \item Frodo is the ring bearer (the closest to a main character)
    \item Elijah Wood essentially becomes Frodo after the first movie.
    \item Elijah Wood gets the specific asset of being Frodo.
    \item He cannot be replaced, and he is extremely valuable after the first movie.
\end{wideitemize}
    
\end{frame}

\begin{frame}{From Specificity to Holdup}
\begin{wideitemize}
    \item This alone is not a problem.
    \item It becomes a problem when:
    \begin{wideitemize}
        \item There are things that must be done after Elijah Wood becomes merged with Frodo.
          \begin{wideitemize}
            \item They needed to make \textit{The Two Towers} and \textit{Return of the King}.
        \end{wideitemize}
        \item This generates a problem called hold-up.
    \end{wideitemize}

\end{wideitemize}
    
\end{frame}


\begin{frame}{The Model}

\begin{wideitemize}
    \item Players: Elijah Wood (EW), New Line Cinema (NLC) 
    \item NLC decides whether or not to start LOTR, which entails simultaneously announcing EW as Frodo, and paying fixed cost of production $c>0$.
    \item After this, EW proposes a wage $w$ for all movies.
    \item If NLC rejects, EW gets outside option $\bar u$ and NLC makes 0 box office revenue.
    \item If they accept, NLC receives box office revenue from LOTR $b$.
    \item Assume indifference is broken in favor of making LOTR.
\end{wideitemize}
    
\end{frame}


\begin{frame}{Diagram of the Model}

    \huge See the board!

    
\end{frame}

\begin{frame}{Solving the Model}

    \huge See the board!

    
\end{frame}

\begin{frame}{Hold-Up Solution}

\begin{theorem}
    For all values of fixed costs $c$, outside option $\bar u$, and box office revenue $b$, LOTR is never made.
\end{theorem}

\begin{wideitemize}
    \item Elijah Wood ``holds up" NLC and gets all box office revenue.
    \item Therefore LOTR is never made!
    \item Whenever box office revenue exceeds the real costs ($b \geq \bar u +c$) this is inefficient.
    \item \textit{Fellowship} revenue was approx. \$900 million, and total budget of all 3 movies was less than \$300 million
\end{wideitemize}
    
\end{frame}


\begin{frame}{Making \textit{The Lord of the Rings Trilogy}}

\begin{wideitemize}
    \item Suppose you are deciding how to make the three LOTR movies.
    \item You have two options:\pause
    \begin{wideitemize}
        \item[1.] Make all at once and then release.\pause
        \item[2.] Make them sequentially, with releases in between.
    \end{wideitemize}
    \item What are the main benefits of option 1?
    \begin{wideitemize}
        \item No aging of actors between movies.
        \item Conserve on fixed costs (setting up infrastructure, negotiating contracts, etc.)
    \end{wideitemize}
    \item What are the main benefits of option 2?
    \begin{wideitemize}
        \item Can see if the first does well before making the second.
        \item Can learn from the first when making the second.
    \end{wideitemize}
    
\end{wideitemize}
    
\end{frame}




\begin{frame}{Hold-Up with Back-to-Back Filming}
\begin{wideitemize}
    \item Let's modify the original game.
    \item Specifically: EW proposes a wage to NLC before production starts.
    \item After accepting the wage, NLC starts production and incurs the fixed cost.
\end{wideitemize}
\end{frame}


\begin{frame}{New Diagram}

    \huge See the board!

    
\end{frame}


\begin{frame}{Solving the Model}

    \huge See the board!

    
\end{frame}


\begin{frame}{Fixing Hold-Up: Back-to-Back Filming}

\begin{theorem}
Under back-to-back filming, LOTR is made whenever $b\geq \bar u +c$.
\end{theorem}

    \begin{wideitemize}
        \item Now EW proposes an acceptable wage: $w=b-c$
        \item Changing timing ``fixes" hold-up!
        \item EW now internalizes the fixed cost of production.
        \item EW still gets all of the surplus, but...
        \item By changing NLC's outside option we can fix this to be more realistic.
    \end{wideitemize}
\end{frame}



\begin{frame}{Discussion}

    \huge Is this a realistic solution?

    
\end{frame}


\begin{frame}{Back-to-Back Has Drawbacks}
\begin{wideitemize}
    \item \textit{Superman} and \textit{Superman II} were shot back to back.
    \item Actors signed on to both upfront.
    \item However there was conflict between the producers and the original director.
    \item So the producers hired a new director.
    \item As a result, most ``concurrent" footage was not used.
\end{wideitemize}
\end{frame}

\begin{frame}{Renegotiation/Slacking}

\begin{wideitemize}
    \item We ignored the fact that filming is a process.
    \item We also ignored that most employment in the US is at will.
    \item EW could potentially threaten to slack or quit part way through.
    \item EW could then propose a contract partway through when he has a lot of leverage.
    \item This means our solution is vulnerable in some sense.
\end{wideitemize}
    
\end{frame}


\end{document}







