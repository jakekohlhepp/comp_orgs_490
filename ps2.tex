\documentclass{article}
\usepackage{graphicx} % Required for inserting images
\usepackage{amsmath} 
\newtheorem{lemma}{lemma}
\usepackage{geometry}
\geometry{margin=1in}
\title{Problem Set 2}
\author{Jacob Kohlhepp}
\date{\today}

\begin{document}

\maketitle


The purpose of this homework is to work through a multitasking problem and a relative performance evaluation problem. There are only minor differences between these problems and the ones we did in class, so your notes should be very helpful in completing this problem set.

\section{Meaning and Performance}
\subsection{Setup}
    \begin{itemize}
        \item The cost of effort is: $c_{ij}(e) = \frac{e^2}{2(1+\lambda_i m_{ij})} $.

    \item Output is given by: $y_{ij}=\theta_i e +\epsilon_{ij}$.
    \item The contract is based on output: $w_{ij}(y_{ij})= \alpha_{ij} + \beta_{ij} y_{ij}=\alpha_{ij} + \beta_{ij} (\theta_i e_{ij} +\epsilon_{ij})$
\item The timing is as usual: the firm designs the contract and the worker then decides how much effort to exert.

\end{itemize}

\subsection{Questions}



\begin{enumerate}
    \item Show the model captures the idea that ``if you enjoy your job you will never work a day in your life."

    \item Explain how we can return to our standard effort-based pay model by setting primitives to certain values.

    \item Taking the contract as given (and assuming the worker takes the job), solve for the worker's effort.

    \item Explain why the ``take the job" step from our standard performance pay or effort based-pay model does not make sense in this setting. Make sure to connect it to the reading.

    \item Given your earlier answers, maximize firm $j$'s profit with respect to $\beta_{ij}$, and solve for the firm's profit maximizing bonus.

    \item Write down and simplify the worker's utility from taking the job assuming some given base pay $\alpha_{ij}$.\footnote{Remember we do not allow the firm to choose the base pay in this model so it is just a primitive.}

    \item Write down and simplify an inequality that represents the worker choosing between two jobs $p$ and $a$.

    \item The experiment in the paper changed $\lambda_i$ from 0 to 1. Use the model to show how this should impact effort of some fixed worker $i$ who stays. Hint: Verification 2 in the paper will help.


    
\end{enumerate}


\section{Multitasking}
There are two key differences in the setup of this problem vs. what we did in class:

\begin{enumerate}
    \item Task 1 effort is measured in different ``units."
    \item I am telling you specific values for $a,b$ in the last part of the problem.
\end{enumerate}

\subsection{Setup}
\begin{itemize}
    \item Output is $y=a e_1+b e_2$
    \item Cost of effort is:
       \[c(e_1, e_2) = \begin{cases}
            0 & \text{ if }  e_1+e_2 \leq 2 \bar e \\
            (e_1+e_2-2\bar e)^2/2 & \text{ if } e_1+e_2 > 2 \bar e 
        \end{cases}\]
      \item We assume that without incentives the worker supplies all 0 cost effort and splits effort evenly:
      \[e_1=e_2=\bar e\]
    \item Only task 1 effort is measured: $m=k \cdot e_1$, where $k>0$.
    \item The firm can only pay based on task 1: $w(m)=\alpha + \beta  m =\alpha + \beta k e_1$
    \item The firm's and worker's outside options are 0.
\end{itemize}

\subsection{Questions}
\begin{enumerate}
    \item Setup the firm's problem in the first-best, that is when the firm can just choose effort directly and we do not care about wages.

    \item Solve for the first-best $e_1,e_2$ when $a>b, a>0$. Only assume that $a>b$ for this problem.

 \item From now on we are solving for equilibrium, meaning the firm cannot choose effort directly but just chooses a compensation scheme. Setup the worker's effort choice problem.

 \item Solve for worker's choice of effort assuming for now until told otherwise that $\beta>0$.

 \item Write down the inequality that determines whether the worker takes the job. Argue that it must be an equality.

 \item Setup the firm's profit maximization problem. Substitute past work in so that it is only a function of $\beta$.
 \item Solve for $\beta, e_1,e_2$.

 \item Now, solve for $e_1, e_2$ when $\beta=0$. You may use the same steps we just did or do it your own way.
 
    \item From now until I say otherwise assume that $a=-1, b=2, \bar e=1$. Provide an interpretation for $a$ being negative.

    \item Using the work you have already done, should the firm set $\beta=0$ or $\beta>0$? Find $\beta, e_1, e_2$.

    \item Now assume that $a=2, b=1,\bar e=1$. Using the work you have already done, should the firm set $\beta=0$ or $\beta>0$? Find $\beta, e_1, e_2$.

    \item Do your answers to any of these questions depend on $k$? Interpret your answer.
\end{enumerate}

\section{Relational Contracts}


\subsection{Setup}
\begin{itemize}
    \item A firm and a worker both have discount rate $\delta$ and interact for many periods ($t=1,...,\infty$)
    \item At each period $t$ the following occur:
    \begin{itemize}
        \item First the firm offers a flat wage $w_t$
        \item Second the worker chooses high (H) or low (L) effort $e_t$
    \end{itemize}
    \item High effort has cost $c$, low effort has cost 0.
    \item High effort yields revenue $v$, low effort yields revenue 0.
    \item Firm outside option is 0, worker outside option is $\bar u>0$.
    \item Assume the firm wants to motivate high effort.
\end{itemize}

\subsection{Questions}

\begin{enumerate}
    \item Guess an equilibrium strategy for the firm in words. Guess an equilibrium strategy for the worker in words. (Hint: guess the same strategy as in class)

    

        \item Call the high wage $w_H$ and the low wage $w_L$. Assume the strategy we guessed is being played. What value of $w_L$ will the firm choose and why?

 
    \item What is the worker's payoff in any period where the firm posts a wage of $w_L$? Justify your answer.

   \item Consider the case when trust was already broken in the past. Write down the worker's present value utility from not deviating, that is following our guessed strategy. Write down the worker's present value utility from deviating from a one shot deviation from our guessed strategy that involves taking the job and exerting low effort.

     \item Consider the case when trust was already broken in the past. Write down an inequality for when there are no incentives for the worker to deviate in this case. Make sure to simplify. When does this inequality hold?

    
    \item Consider the case when trust has never been broken. Write down the worker's present value utility from not deviating and following our guessed strategy. Write down the worker's utility from a one shot deviation of exerting low effort today.


\item Consider the case when trust has never been broken. Argue that the worker would always prefer to take the job and exert low effort rather than not take the job.

    \item Consider the case when trust has never been broken. Write down (and simplify) an inequality for when there is no incentive for the worker to deviate. When is it satisfied?

   Assume $w_H$ in equilibrium is the lowest possible wage such that the inequality is satisfied. 
   
    \item Suppose $\delta = 0.4, v=3, \bar u = 1, c=1$. Is the relational contract we derived profitable for the firm?
    \item Suppose $\delta = 0.6, v=3, \bar u= 1,c=1$. Is the relational contract we derived profitable for the firm?

    \item Interpret the difference between your prior two answers.
\end{enumerate}



\end{document}