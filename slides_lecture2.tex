%%%%%%%%%%%%%%%%%%%%%%%%%%%%%%%%%%%%%%%%%
% Beamer Presentation
% LaTeX Template
% Version 1.0 (10/11/12)
%
% This template has been downloaded from:
% http://www.LaTeXTemplates.com
%
% License:
% CC BY-NC-SA 3.0 (http://creativecommons.org/licenses/by-nc-sa/3.0/)
%
%%%%%%%%%%%%%%%%%%%%%%%%%%%%%%%%%%%%%%%%%

%----------------------------------------------------------------------------------------
%	PACKAGES AND THEMES
%----------------------------------------------------------------------------------------

\documentclass[aspectratio=169,usenames,dvipsnames]{beamer}

\usepackage[utf8]{inputenc}
\usepackage{booktabs}
\usepackage{tabularx}
\usepackage[authordate,bibencoding=auto,strict,backend=biber,natbib]{biblatex-chicago}
\addbibresource{bib.bib}
\usepackage{graphicx}
% \hypersetup{
%     colorlinks,
%     %citecolor=black,
%     linkcolor=black
% }
\usepackage{array}
\usepackage{caption}
\usepackage{threeparttable}
\usepackage{epigraph} 
\usepackage{lscape}
\usepackage{adjustbox}
\newcommand*{\Scale}[2][4]{\scalebox{#1}{\ensuremath{#2}}}%
\usepackage{import}
\newenvironment{wideitemize}{\itemize\addtolength{\itemsep}{10pt}}{\enditemize}
\usepackage{amsmath}
\usepackage{csvsimple}
\usepackage{siunitx}
\usepackage{filecontents}
\usepackage{rotating}
\usepackage{multirow}
\usepackage{amsmath}
\usepackage{subcaption}
\usepackage{appendixnumberbeamer}
\usepackage{float}
\usepackage{amsmath}
\usepackage{csvsimple}
\usepackage{hyperref}
\newtheorem{proposition}{Proposition}
\usepackage{xcolor}
\def\boxit#1#2{%
    \smash{\color{red}\fboxrule=1pt\relax\fboxsep=2pt\relax%
    \llap{\rlap{\fbox{\phantom{\rule{#1}{#2}}}}~}}\ignorespaces
}
\newenvironment{variableblock}[3]{%
  \setbeamercolor{block body}{#2}
  \setbeamercolor{block title}{#3}
  \begin{block}{#1}}{\end{block}}
\usepackage{appendixnumberbeamer}
\usepackage{tikz,pgfplots}
\usepackage{tkz-fct}
\usepackage{amsthm}
\pgfplotsset{compat=1.10}
\usepgfplotslibrary{fillbetween}
\mode<presentation> {
\AtBeginSection[]
{
    \begin{frame}
        \frametitle{Table of Contents}
        \tableofcontents[currentsection]
    \end{frame}
}
% The Beamer class comes with a number of default slide themes
% which change the colors and layouts of slides. Below this is a list
% of all the themes, uncomment each in turn to see what they look like.

\usetheme{default}
%\usetheme{AnnArbor}
%\usetheme{Antibes} -
%\usetheme{Bergen}
%\usetheme{Berkeley}
%\usetheme{Berlin}
%\usetheme{Boadilla}
%\usetheme{CambridgeUS}
%\usetheme{Copenhagen} -
%\usetheme{Darmstadt}
%\usetheme{Dresden}
%\usetheme{Frankfurt}
%\usetheme{Goettingen}
%\usetheme{Hannover}
%\usetheme{Ilmenau}
%\usetheme{JuanLesPins}
%\usetheme{Luebeck}
%\usetheme{Madrid}
%\usetheme{Malmoe}
%\usetheme{Marburg}
%\usetheme{Montpellier}
%\usetheme{PaloAlto}
%\usetheme{Pittsburgh}
%\usetheme{Rochester} -
%\usetheme{Singapore}
%\usetheme{Szeged}
%\usetheme{Warsaw}

% As well as themes, the Beamer class has a number of color themes
% for any slide theme. Uncomment each of these in turn to see how it
% changes the colors of your current slide theme.

%\usecolortheme{albatross}
%\usecolortheme{beaver}
%\usecolortheme{beetle}
%\usecolortheme{crane}
%\usecolortheme{dolphin}
%\usecolortheme{dove}
%\usecolortheme{fly}
%\usecolortheme{lily}
%\usecolortheme{orchid}
%\usecolortheme{rose}
%\usecolortheme{seagull}
%\usecolortheme{seahorse}
%\usecolortheme{whale}
%\usecolortheme{wolverine}

%\setbeamertemplate{footline} % To remove the footer line in all slides uncomment this line
%\setbeamertemplate{footline}[frame number] % To replace the footer line in all slides with a simple slide count uncomment this line
\setbeamertemplate{theorems}[numbered]
\setbeamertemplate{navigation symbols}{} % To remove the navigation symbols from the bottom of all slides uncomment this line
}
\setbeamertemplate{caption}{\raggedright\insertcaption\par}
  \setbeamertemplate{enumerate items}[default]
  %\setbeamertemplate{page number in head/foot}{\insertframenumber}
\usepackage{graphicx} % Allows including images
\usepackage{booktabs} % Allows the use of \toprule, \midrule and \bottomrule in tables
%\usepackage {tikz}
\newtheorem*{theorem*}{Theorem}
\newtheorem*{lemma*}{Lemma}
\newtheorem*{proposition*}{Proposition}
\newtheorem*{corollary*}{Corollary}
\newtheorem*{definition*}{Definition}
\DeclareMathOperator*{\argmin}{arg\,min}
\newtheorem*{assumption}{Assumption}
\usetikzlibrary {positioning}
\renewcommand{\arraystretch}{1.5}
\newcommand\hideit[1]{%
  \only<0| handout:1>{\mbox{}}%
  \invisible<0| handout:1>{#1}}
\usepackage[default]{lato}

\setbeamercolor{block body alerted}{bg=alerted text.fg!10}
\setbeamercolor{block title alerted}{bg=alerted text.fg!20}
\setbeamercolor{block body}{bg=structure!10}
\setbeamercolor{block title}{bg=structure!20}
\setbeamercolor{block body example}{bg=green!10}
\setbeamercolor{block title example}{bg=green!20}


\makeatletter
\let\save@measuring@true\measuring@true
\def\measuring@true{%
  \save@measuring@true
  \def\beamer@sortzero##1{\beamer@ifnextcharospec{\beamer@sortzeroread{##1}}{}}%
  \def\beamer@sortzeroread##1<##2>{}%
  \def\beamer@finalnospec{}%
}
\makeatother
%\usepackage {xcolor}

%----------------------------------------------------------------------------------------
%	TITLE PAGE
%----------------------------------------------------------------------------------------

\title[diss]{Lecture 2: The Toolkit} % The short title appears at the bottom of every slide, the full title is only on the title page
\author{Compensation in Organizations} % Your name
\institute[shortinst]{Jacob Kohlhepp}
\date{\today} % Date, can be changed to a custom date

\begin{document}

\begin{frame}
\titlepage % Print the title page as the first slide

\end{frame}

\begin{frame}
\centering
    \huge Discussion: Hartzell, Parsons, Yermack (2010)
\end{frame}


\begin{frame}{What Should You Know Already?}

\begin{wideitemize}
    \item Single variable derivatives.
    \item Inequalities.
    \item Very basic probability.
\end{wideitemize}
\end{frame}

\begin{frame}{What Will I Teach You Today?}

\begin{wideitemize}
    \item The concept of risk aversion and a useful formula.
    \item A decision problem.
    \item Simple game theory, static.
    \item Simple game theory, dynamic.
\end{wideitemize}
\end{frame}


\begin{frame}{Caveats}

\begin{wideitemize}
    \item I will try to cover just what is needed for the class.
    \item I will not cover Nash equilibrium in depth.
    \item I will not cover risk aversion in depth.
    \item If you want to go beyond this or want more practice, see my notes and practice problems from an old course: \url{https://github.com/jakekohlhepp/Econ101}.
\end{wideitemize}
\end{frame}

\section{Risk Aversion}

\begin{frame}{Compensation as Lotteries}

\begin{wideitemize}
    \item Your performance in a job is often effort + noise/luck/chance/circumstance
    \item Effort is in your control, but the rest is random.
    \item If you get paid based on performance, your compensation is a lottery!
\end{wideitemize}
    
\end{frame}
\begin{frame}{A Survey}
    Suppose a company offered you three compensation schemes:
    \begin{wideitemize}
        \item[a.] $X_a$: \$100,000 with probability 50\%, \$0 with probability 50\%.
        \item[b.] $X_b$: \$49,000 with probability 100\%
        \item[c.] $X_c$: \$200,000 with probability 24\%, \$0 with probability 76\%.
    \end{wideitemize}
    Which would you choose? (will tally on board)
\end{frame}

\begin{frame}{Risk Attitudes}
    \begin{wideitemize}
        \item Before looking at the results, notice some facts:
        \begin{itemize}
            \item b has lower average than a: $E[X_a]=50>49=E[X_b]$. 
            \item But a has higher variance than b: $Var(X_a)=2500>0=Var(X_b)$.
            \item So if you dislike variance or uncertainty, you will prefer a.
            \item $E[X_c]=48<49<50$: in terms of expected money, c is worse than both a and b.
            \item However, the variance of c is higher: $Var(X_c) \approx 7296>2500>0$.
            \item You will only choose c if you like uncertainty/risk.
        \end{itemize}
        \item A rough interpretation of the results:
        \begin{itemize}
            \item If you chose b you are \textit{risk averse}: you dislike uncertainty/risk, and are willing to pay to reduce it
            \item If you chose a you are either \textit{risk neutral}
            \item If you chose c you are \textit{risk loving}
        \end{itemize}
    \item We will assume people are risk averse or risk neutral.
    \end{wideitemize}
\end{frame}

\begin{frame}{Expected Utility Theory}
 \begin{wideitemize}
     \item We analyze uncertainty using \textbf{Expected Utility Theory.}
     \item This theorem justifies the tools we will use in this class:
    
     \begin{theorem}
Under a set of axioms (which you do not need to know), we can represent an individual's preferences over lotteries using an \textbf{expected utility function} $u$ where $E[u(X_a)] \geq E[u(X_b)]$ means that lottery $a$ is preferred to lottery $b$.
\end{theorem}

 \end{wideitemize}
 \end{frame}

\begin{frame}{Risk Attitudes As Functions}
A person with expected utility function $u$ is...
\begin{wideitemize}
    \item \textbf{risk averse} if $u$ is concave.
    \item \textbf{risk neutral} if $u$ is linear.
    \item \textbf{risk loving} if $u$ is convex.
\end{wideitemize}

\end{frame}

\begin{frame}{Certainty Equivalent}

\begin{wideitemize}
    \item We often want to rank lotteries/gambles. The following concept is useful for ranking lotteries:
    \begin{definition}
    The amount of money for sure a decision maker is willing to pay for lottery $a$ is the \textbf{certainty equivalent} ($d_a$). Mathematically:
    \[u(d_a) = E[u(X_a)]\]
    \end{definition}
    \item Given a lottery that gives me $d$ dollars for sure and $X_a$, it is the value of d where I am indifferent.
    \pause \item Gut check: What is the certainty equivalent of a lottery with $E[X_a]=10$ when the decision maker is risk neutral?
\end{wideitemize}
\end{frame}

\begin{frame}{Certainty Equivalent}

\begin{wideitemize}
    \item We often want to rank lotteries/gambles. The following concept is useful for ranking lotteries:
    \begin{definition}
    The amount of money for sure a decision maker is willing to pay for lottery $a$ is the \textbf{certainty equivalent} ($d_a$). Mathematically:
    \[u(d_a) = E[u(X_a)]\]
    \end{definition}
    \item Given the choice between $d$ dollars for sure or $X_a$, it is the value of d where I am indifferent.
    \pause \item Gut check: What is the certainty equivalent of a lottery with $E[X_a]=10$ when the decision maker is risk neutral?
\end{wideitemize}
\end{frame}

\begin{frame}{Exponential Utility}
\begin{wideitemize}
    \item We will use the exponential utility function in this class:
    \[u(x) = \frac{1-e^{-r x}}{r} \]
    \item $r$ captures risk aversion:
    \begin{wideitemize}
        \item When $r>0$ the decision maker is risk averse
    
        \item When $r<0$ they are risk loving.
    
    \end{wideitemize}

           \item What happens when $r \to 0$? Using L'Hopsital's rule:
       \[\lim_{r \to 0}  \frac{1-e^{-r x}}{r} = x\]
       So we have risk neutrality!
       
\end{wideitemize}

\end{frame}
\begin{frame}{Exponential Utility}
\begin{wideitemize}
    \item We will use the exponential utility function in this class:
    \[u(x) = \frac{1-e^{-r x}}{r} \]
    \begin{theorem}
        When a person has risk preference given by an exponential utility function $u(x) = \frac{1-e^{-r x}}{r}$, the certainty equivalent of a normal lottery with mean $\mu$ and variance $\sigma^2$ is given by:
        \[d = \mu -r \frac{\sigma^2}{2}\]
    \end{theorem}
    \item In this class, you can apply this formula directly.
    \item Talk to me if you are interested in the derivation!
\end{wideitemize}
\end{frame}
\section{Decision Problem}

\begin{frame}{Making a Decision}

\begin{wideitemize}
    \item Suppose there is a single person ($A$) making a decision.
    \item $A$ takes an action which we will call $e$.
    \item $e$ can be a \textbf{discrete action}:
    \begin{wideitemize}
        \item accept or reject
        \item work hard or slack off
    \end{wideitemize}
    \item $e$ can be a \textbf{continuous action}:
    \begin{wideitemize}
        \item exert $e$ units of effort
        \item drive $e$ miles
    \end{wideitemize}
    \item The utility or payoff of an action $e$ is $u(e)$
    \end{wideitemize}
\end{frame}

\section{Game Theory}

\begin{frame}{Game Theory in This Class}
\begin{wideitemize}
    \item Game theory lets us model strategic interaction.
    \item Therefore it is a tool, but not the point, of this class.
    \item I do not require you to learn the definitions of Nash equilibrium, best responses, etc.
    \item However, if you want to go to econ. grad school this can be useful.
    \item I do require you to make either mathematical and/or verbal arguments.
    \item For tests, I will only ask you to solve models we solved in class (sometimes with slight modifications).
\end{wideitemize}
    
\end{frame}

\begin{frame}{Competing for a Worker (Bertrand Game)}
We will use the basic ideas of this game often:
\begin{wideitemize}
    \item \textbf{Players.} Two identical firms, numbered $i=1,2$, and one worker.
    \item \textbf{Actions.} Firms choose wages \textit{continuously}: $0\leq w_i < \infty$
    \item \textbf{Payoffs.} 
    \begin{enumerate}
        \item Worker gets the wage of the firm they choose.
        \item When firms set the worker chooses randomly.
        \item The firm which hires the worker gets productivity $p$ and pays the wage.
        \item If a firm does not hire they get 0.
    \end{enumerate}
\end{wideitemize}
\end{frame}

\begin{frame}{Competing for a Worker: Solution}
\centering
    \huge See the board (we will solve this almost entirely verbally)!
\end{frame}

\begin{frame}{The Company Call List (Tragedy of the Commons)}
\begin{wideitemize}
    \item \textbf{Players.} Two sales workers, $i=1,2$, share a common list of company sales contacts.
    \item \textbf{Actions.} Each chooses a number of people on the list to call, $q_i$
    \item \textbf{Payoffs.} 
    \begin{enumerate}
        \item The cost to the worker of making a call on the list is 0.
        \item The amount of commission the worker makes per call is $120-q_1-q_2$
    \end{enumerate}
\end{wideitemize}


\end{frame}

\begin{frame}{The Company Call List (Tragedy of the Commons) - Solution}
\centering
    \huge See the board!
\end{frame}




\begin{frame}{Sequential Games}
\begin{wideitemize}
    \item We will consider several models in this class where a firm moves first (usually to set up a compensation plan)
    \item Then a worker reacts to this compensation plan.
    \item Unlike the last examples these are sequential games.
    \item Two key differences between static and dynamic:
    \begin{wideitemize}
        \item In sequential games, future players take past player actions as fixed.
        \item Earlier players anticipate future players will react to their current choices.
    \end{wideitemize}
    \item To deal with this, we use subgame perfect Nash equilibrium.
    \item For this class, that just means we use backwards induction.
\end{wideitemize}
    
\end{frame}
\begin{frame}{The Pirate Riddle}
There is a very famous and tricky riddle that we will use to illustrate backwards induction.
\begin{quote}
    Five pirates, numbered 1 through 5, must decide how to divide 100 gold coins. Their decision process is as follows. Starting with pirate 1, each pirate proposes a split consisting of a number of coins for each of the pirates on the ship. Then all pirates vote. If a strict majority approve, the allocation happens. If it does not the proposer is thrown off the ship and the remaining pirates repeat the process. Assume pirates value 2 coins more than 1, etc and that getting thrown off is worse than getting 0 coins. Assume pirates vote no when indifferent (they get a little bit of enjoyment from watching someone walk the plank). What is the maximum number of coins P1 can obtain and not get thrown off?
\end{quote}
    
\end{frame}


\begin{frame}{Pirate Riddle: Verbal Solution}

\begin{wideitemize}
    \item Start from the end of the game. If P5 gets to make a proposal, then they are the last pirate left. They propose 100 coins for themselves and get it!\pause
    \item Rolling back, P4 needs pirate 5's vote for a strict majority. There is no way to get it since P5 knows they get 100 coins if they throw off 4. Thus P4 can propose anything, and P5 always rejects.\pause
    \item Roll back. P3 needs 1 other vote to get a majority. The easiest person to convince is pirate 4, since pirate 4 gets thrown off if the game continues. To get P4's vote P3 can get away with giving him/her 0 coins. P3 proposes\pause
    \item Roll back. Pirate 2 needs to get two votes. P4 and P5 are the cheapest to convince because they get 0 next round. So P2 gives P4 and P5 1, P3 0, and keeps 98.\pause
    \item Roll back. P1 needs two other votes. P3 is the cheapest to convince. P4 and P5 are next cheapest, and P1 need only convince one. So P1 proposes 0 for P2 and P5, P3 1, and P4 2 and keeps 97!
\end{wideitemize}
\end{frame}

\begin{frame}{Pirate Riddle: Open Discussion}
\centering
    \huge What can we learn from the pirate riddle? What is surprising about the outcome/solution?
\end{frame}

\begin{frame}{The Company Call List - Sequential Version}
We now modify our static company call list game to be sequential.
\begin{wideitemize}
    \item \textbf{Players.} Two sales workers, $i=1,2$, share a common list of company sales contacts.
    \item \textbf{Actions.} Each chooses a number of people on the list to call, $q_i$
    \item \textbf{Timing.} Worker 1 calls first, then worker 2.
    \item \textbf{Payoffs.} 
    \begin{enumerate}
        \item The cost to the worker of making a call on the list is 0.
        \item The amount of commission the worker makes per call is $120-q_1-q_2$
    \end{enumerate}
\end{wideitemize}


\end{frame}

\begin{frame}{The Company Call List - Sequential Version Solution}
\centering
    \huge See the board!
\end{frame}
\end{document}







