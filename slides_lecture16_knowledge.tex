%%%%%%%%%%%%%%%%%%%%%%%%%%%%%%%%%%%%%%%%%
% Beamer Presentation
% LaTeX Template
% Version 1.0 (10/11/12)
%
% This template has been downloaded from:
% http://www.LaTeXTemplates.com
%
% License:
% CC BY-NC-SA 3.0 (http://creativecommons.org/licenses/by-nc-sa/3.0/)
%
%%%%%%%%%%%%%%%%%%%%%%%%%%%%%%%%%%%%%%%%%

%----------------------------------------------------------------------------------------
%	PACKAGES AND THEMES
%----------------------------------------------------------------------------------------

\documentclass[aspectratio=169,usenames,dvipsnames]{beamer}

\usepackage[utf8]{inputenc}
\usepackage{booktabs}
\usepackage{tabularx}
\usepackage[authordate,bibencoding=auto,strict,backend=biber,natbib]{biblatex-chicago}
\addbibresource{bib.bib}
\usepackage{graphicx}
% \hypersetup{
%     colorlinks,
%     %citecolor=black,
%     linkcolor=black
% }
\usepackage{array}
\usepackage{caption}
\usepackage{threeparttable}
\usepackage{epigraph} 
\usepackage{lscape}
\usepackage{adjustbox}
\newcommand*{\Scale}[2][4]{\scalebox{#1}{\ensuremath{#2}}}%
\usepackage{import}
\newenvironment{wideitemize}{\itemize\addtolength{\itemsep}{10pt}}{\enditemize}
\usepackage{amsmath}
\usepackage{csvsimple}
\usepackage{siunitx}
\usepackage{filecontents}
\usepackage{rotating}
\usepackage{multirow}
\usepackage{amsmath}
\usepackage{subcaption}
\usepackage{appendixnumberbeamer}
\usepackage{float}
\usepackage{amsmath}
\usepackage{csvsimple}
\usepackage{hyperref}
\newtheorem{proposition}{Proposition}
\usepackage{xcolor}
\def\boxit#1#2{%
    \smash{\color{red}\fboxrule=1pt\relax\fboxsep=2pt\relax%
    \llap{\rlap{\fbox{\phantom{\rule{#1}{#2}}}}~}}\ignorespaces
}
\newenvironment{variableblock}[3]{%
  \setbeamercolor{block body}{#2}
  \setbeamercolor{block title}{#3}
  \begin{block}{#1}}{\end{block}}
\usepackage{appendixnumberbeamer}
\usepackage{tikz,pgfplots}
\usepackage{tkz-fct}
\usepackage{amsthm}
\pgfplotsset{compat=1.10}
\usepgfplotslibrary{fillbetween}
\mode<presentation> {
\AtBeginSection[]
{
    \begin{frame}
        \frametitle{Table of Contents}
        \tableofcontents[currentsection]
    \end{frame}
}
% The Beamer class comes with a number of default slide themes
% which change the colors and layouts of slides. Below this is a list
% of all the themes, uncomment each in turn to see what they look like.

\usetheme{default}
%\usetheme{AnnArbor}
%\usetheme{Antibes} -
%\usetheme{Bergen}
%\usetheme{Berkeley}
%\usetheme{Berlin}
%\usetheme{Boadilla}
%\usetheme{CambridgeUS}
%\usetheme{Copenhagen} -
%\usetheme{Darmstadt}
%\usetheme{Dresden}
%\usetheme{Frankfurt}
%\usetheme{Goettingen}
%\usetheme{Hannover}
%\usetheme{Ilmenau}
%\usetheme{JuanLesPins}
%\usetheme{Luebeck}
%\usetheme{Madrid}
%\usetheme{Malmoe}
%\usetheme{Marburg}
%\usetheme{Montpellier}
%\usetheme{PaloAlto}
%\usetheme{Pittsburgh}
%\usetheme{Rochester} -
%\usetheme{Singapore}
%\usetheme{Szeged}
%\usetheme{Warsaw}

% As well as themes, the Beamer class has a number of color themes
% for any slide theme. Uncomment each of these in turn to see how it
% changes the colors of your current slide theme.

%\usecolortheme{albatross}
%\usecolortheme{beaver}
%\usecolortheme{beetle}
%\usecolortheme{crane}
%\usecolortheme{dolphin}
%\usecolortheme{dove}
%\usecolortheme{fly}
%\usecolortheme{lily}
%\usecolortheme{orchid}
%\usecolortheme{rose}
%\usecolortheme{seagull}
%\usecolortheme{seahorse}
%\usecolortheme{whale}
%\usecolortheme{wolverine}

%\setbeamertemplate{footline} % To remove the footer line in all slides uncomment this line
%\setbeamertemplate{footline}[frame number] % To replace the footer line in all slides with a simple slide count uncomment this line
\setbeamertemplate{theorems}[numbered]
\setbeamertemplate{navigation symbols}{} % To remove the navigation symbols from the bottom of all slides uncomment this line
}
\setbeamertemplate{caption}{\raggedright\insertcaption\par}
  \setbeamertemplate{enumerate items}[default]
  %\setbeamertemplate{page number in head/foot}{\insertframenumber}
\usepackage{graphicx} % Allows including images
\usepackage{booktabs} % Allows the use of \toprule, \midrule and \bottomrule in tables
%\usepackage {tikz}
\newtheorem*{theorem*}{Theorem}
\newtheorem*{lemma*}{Lemma}
\newtheorem*{proposition*}{Proposition}
\newtheorem*{corollary*}{Corollary}
\newtheorem*{definition*}{Definition}
\DeclareMathOperator*{\argmin}{arg\,min}
\newtheorem*{assumption}{Assumption}
\usetikzlibrary {positioning}
\renewcommand{\arraystretch}{1.5}
\newcommand\hideit[1]{%
  \only<0| handout:1>{\mbox{}}%
  \invisible<0| handout:1>{#1}}
\usepackage[default]{lato}

\setbeamercolor{block body alerted}{bg=alerted text.fg!10}
\setbeamercolor{block title alerted}{bg=alerted text.fg!20}
\setbeamercolor{block body}{bg=structure!10}
\setbeamercolor{block title}{bg=structure!20}
\setbeamercolor{block body example}{bg=green!10}
\setbeamercolor{block title example}{bg=green!20}


\makeatletter
\let\save@measuring@true\measuring@true
\def\measuring@true{%
  \save@measuring@true
  \def\beamer@sortzero##1{\beamer@ifnextcharospec{\beamer@sortzeroread{##1}}{}}%
  \def\beamer@sortzeroread##1<##2>{}%
  \def\beamer@finalnospec{}%
}
\makeatother
%\usepackage {xcolor}

%----------------------------------------------------------------------------------------
%	TITLE PAGE
%----------------------------------------------------------------------------------------

\title[diss]{Lecture 16: Knowledge as Compensation} % The short title appears at the bottom of every slide, the full title is only on the title page
\author{Compensation in Organizations} % Your name
\institute[shortinst]{Jacob Kohlhepp}
\date{\today} % Date, can be changed to a custom date

\begin{document}

\begin{frame}
\titlepage % Print the title page as the first slide

\end{frame}

\begin{frame}{Discussion}
\centering
    \huge  Garicano and Rayo (2025)

    \normalsize (Not Yet Published)
\end{frame}


\begin{frame}{Discussion}
\centering
    \huge  Garicano (2025)

    \normalsize (Blog Post on Silicon Continent)
\end{frame}

\begin{frame}{Medieval Apprentices and Generative AI?}

\begin{wideitemize}
    \item Smart people  are willing to do grunt work. What are examples?

    \item It is hard to get firms to provide general training to workers. Why?

    \item How does this connect to medieval apprenticeships?

    \begin{wideitemize}
        \item Entry-level workers reap the gains of training through future wages.
        \item Entry-level workers do not have the money to pay for training today.
        \item So they pay for training via grunt work.
    \end{wideitemize}
    
\end{wideitemize}
\end{frame}


\begin{frame}{Transferring Knowledge}
    \begin{wideitemize}
        \item Time is continuous and infinite: $t\geq 0$.
        \item Both the firm and the worker exponentially discount the future at rate $r$.\footnote{Do not worry I will show what this means.}
        \item Expert E commits to wage path $\{w_t\}$ and knowledge transfer path $\{x_t\}$.
        \begin{wideitemize}
            \item Starting knowledge is $0$, maximum is $1$, knowledge cannot decrease.
        \end{wideitemize}
        \item The Apprentice's output at time t is $0$ if $x_t<\theta $, and $(x_t-\theta)$ otherwise.
        \item While employed, the Apprentice A gets the discounted flow of wages.
            \item Apprentice decides a time to quit $\tau\geq 0$, they work for themselves and receive the discounted flow of their output forever after.
           \item The expert gets the discounted flow of output less wages while the worker is employed, and 0 after.

    \end{wideitemize}
\end{frame}


\begin{frame}{Transferring Knowledge: Adding Generative AI}

\begin{wideitemize}
    \item Where was AI?
    \item The parameter $\theta$ represents the level of generative AI.
    \item As $\theta$ rises AI can perform more of the basic knowledge tasks.
    \item The expert and apprentice cannot sell tasks that AI can do. Why?
\end{wideitemize}
    
\end{frame}

\begin{frame}{The Apprentice Becomes the Expert!}

\begin{theorem}
    During a training period that lasts from $0\leq t \leq \frac{1}{r}$, the apprentice receives 0 wages ($w_t=0$) and knowledge is transferred:
    \[x_t = \theta + (1-\theta)e^{-1} exp(rt).\]
    From $\frac{1}{r}$, all knowledge has been transferred ($x_t=1$) and the apprentice is paid all output $w_t=y_t=1-\theta$.
\end{theorem}
    \begin{wideitemize}
        \item The apprentice stays for 0 wages initially! Why?
        \item The apprentice is given an immediate burst of knowledge. Why?
    \end{wideitemize}
\end{frame}


\begin{frame}{Generative AI and the Old Training Model}

\begin{wideitemize}
    \item Profit for the expert from training the apprentice is given by:
    \[\Pi = \frac{1-\theta}{r \cdot e}\]
    \item As $\theta$ rises, profit falls, eventually converging to 1 when AI reaches the expert.
    \item Entry-level workers benefit through wages from general training/knowledge.
    \item The old ``deal" was that entry-level workers did menial work for cheap in exchange for training.
    \item This grunt work compensates the expert for the training.
    \item AI destroys this paradigm.
    
\end{wideitemize}
\end{frame}


\begin{frame}{Employment Growth and AI Exposure: Early Career}

\centering

 \includegraphics[width=0.7\textwidth]{pictures/canaries_early.png}

\raggedright\hfill \footnotesize ``Canaries in the Coal Mine?" by Brynjolfsson, Chandar, Chen (2025)
\end{frame}

\begin{frame}{Employment Growth and AI Exposure: Mid-Late Career}

\centering

 \includegraphics[width=0.7\textwidth]{pictures/canaries_mid.png}

\raggedright \hfill \footnotesize ``Canaries in the Coal Mine?" by Brynjolfsson, Chandar, Chen (2025)
\end{frame}

\end{document}







